% -*-coding:UTF-8-*-
% 毕业论文.tex
% 基于VxWorks的调试通道的设计与实现

\documentclass[UTF-8]{ctexart}

% 插图功能由graphicx宏包提供的。
\usepackage{graphicx}


%*******************设置页边距*****************************
\usepackage{geometry}
%假设将纸张设置为长度20cm,宽15cm,左边距1cm,右边距2cm,上边距3cm,下边距4cm
%\geometry{papersize={20cm,15cm}}
\geometry{a4paper}
\geometry{left=1cm,right=2cm,top=3cm,bottom=4cm}
%************************************************


%*******************设置页眉页尾*****************************
\usepackage{fancyhdr}
\pagestyle{fancy}
%页眉
\lhead{}
\chead{华中科技大学硕士学位论文}
\rhead{}
%页尾
\lfoot{}
\cfoot{\thepage}
\rfoot{}
%页眉和正文之间设置一道宽度为0.4pt的横线分割
\renewcommand{\headrulewidth}{0.4pt}
\renewcommand{\headwidth}{\textwidth}
\renewcommand{\footrulewidth}{0pt}
%************************************************


\title{基于Vxworks的调试通道的设计与实现}
\author{Joe}
\date{\today}

% 设置参考文献的格式
\bibliographystyle{plain}

\begin{document}

% maketitle控制序列能够将导言区中定义的标题、作者、日期按照预定的格式展现出来。
\maketitle

%直接使用tableofcontents生成的目录中不会包含有参考文献那一章。
\tableofcontents


\section{绪论} 



\subsection{课题的背景以及意义}

\paragraph{Tian'anmen square}
is the center of beijing
\subparagraph{chairman mao}
is in the center of 天安门

1. 分析嵌入式实时操作系统VxWorks的特点和其市场定位。

2. USB接口和串口的特点及使用场景。

3. 提出为什么要在VxWorks这个操作系统上实现USB口转串口,并给出本课题的实际来源和市场需求。

\subsection{国内外概况}


\subsection{论文的主要内容和组织结构}




\section{实时操作系统VxWorks}




\section{驱动程序的开发}



\section{应用层程序接口封装}


\section{Windows客户端接收及管理程序的设计}


\section{系统的功能测试}


\section{总结与展望}


% 引入参考文献,需要有reference.bib文件
\bibliography{reference}

\end{document}