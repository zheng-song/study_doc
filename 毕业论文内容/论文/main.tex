
% !TeX program = xelatex

%使用hustthesis这个模板。
% draft版本的正文页包括页眉(“华中科技大学xx学位论文”)、页眉修饰线(双线)、页脚(页码)和页脚修饰线(单线)。
% final版本的正文页不包括页眉、页眉修饰线和页脚修饰线,仅包 含页脚(页码)。如果不指定,默认设置为final。

% degree用来指定论文的种类
% language用来指定论文语言。特别的,如果设定为english-draft,将会剔除论文中的所有中文内容,这有利于在未安装中文字体的环境中使用。如果不指定,默认设置为 chinese。


\documentclass[format=draft,language=chinese,degree=master]{hustthesis}
%\documentclass[format=final,language=chinese,degree=master]{hustthesis}

%使用下面的这两个宏包来生成带书签和超链接的PDF文件。
\usepackage[pdftex,CJKbookmarks=true,colorlinks=true]{hyperref}

\stuno{M201672905}
\schoolcode{10487}
\title{基于Vxworks的调试通道的设计与实现}{A design and implementation of debug channel based on Vxworks.}
\author{郑松}{Joe}
\major{计算机应用技术}{Computer Applications Technology}
\supervisor{张杰\hspace{1em}讲师}{Instructor JieZhang}
\date{2018}{3}{26}
%\date{\today}

\zhabstract{
    数据传输是现代通讯过程中的一个重要环节,在数据的传输过程中,不仅仅要求数据传输的准确率要高,而且要求速度快,连接方便。传统的RS232串口通讯和并口通讯都存在传输速度低,扩展性性差、安装麻烦等缺点,而基于USB接口的数据传输系统能够较好的解决这些问题。目前USB接口以其传输速率高、即插即用、支持热插拔等优点,逐步成为PC机的标准接口。

	本文中的数据传输系统采用了USB接口进行上位机与下位机之间的数据通讯。下位机采用的是VxWorks实时操作系统。

}
\zhkeywords{实时操作系统,设备驱动,USB口转串口,VxWorks,CP2103}

\enabstract
{
    This is a \LaTeX{} template example file. This template is used in written thesis for Huazhong Univ. of Sci. \& Tech.

    This template is published under LPPL v1.3 License.

}
\enkeywords
{\LaTeX{}, Huazhong Univ. of Sci. \& Tech., Thesis, Template}

\begin{document}

% frontmatter用于设定论文的状态、改变样式,其具体使用见简单示例。
% frontmatter用在文档最开始,表明文档的前言部分(如封面,摘要,目录等)的开 始。
\frontmatter

% maketitle的作用和makecover的作用相同,用于生成封面和版权页面
\maketitle

% makeabtract用于生成中英文的摘要页面。
\makeabstract

% tableofcontents用于生成目录
\tableofcontents

% listoffigures 和 listoftables 分别用于生成图片和表格索引,可以根据要求在论文的前言中使用或者是在附录中使用
%\listoffigures
%\listoftables

% mainmatter表示论文正文的开始。
\mainmatter
\clearpage

\chapter{绪论}\label{chapter:1}

VxWorks操作系统是美国WindRiver公司于1983年推出的一种嵌入式实时操作系统(RTOS),是一个运行在目标机上的高性能、可裁剪的实时操作系统,是一个专门为嵌入式实时系统设计开发的操作系统,其具有良好的持续发展能力、高性能的内核以及友好的用户开发环境,为开发人员提供了高效的实时多任务调度、中断管理、实时的系统资源以及实时的任务间通信。在嵌入式实时操作系统领域当中占有一席之地。VxWorks支持X86、PowerPC、ARM等众多主流的处理器,且在各种CPU平台上提供了统一的编程接口和一致的运行环境。在军事、航空航天、工业控制、通信等高精尖以及实时性要求极高的领域当中,有着更加广泛而深入的应用。应用实例包括1997年火星表面登入的火星探测器、爱国者导弹、飞机导航、F-16、FA-18战斗机等。自从对我国的销售解禁之后,VxWorks也大量的应用于我国的军事、国防工业当中,通常在进行VxWorks应用程序的开发或者是将Linux下的应用程序移植到

\section{课题背景以及意义}\label{sec:1}
\section{国内外概况}\label{sec:2}

\section{论文的主要内容和组织结构}\label{sec:3}
\subsection{第二层}\label{sec:1}
\subsubsection{第三层}\label{sec:1}
测试测试测试测试测试测试测试测试测试测试测试测试。
\footnote{\label{footnote:1}脚注}

\section{字体}

普通\textbf{粗体}\emph{斜体}

\hei{黑体}\kai{楷体}\fangsong{仿宋}

\section{公式}

单个公式,公式引用:\autoref{eq:1}。
\begin{equation}
 c^2 = a^2 + b^2 \label{eq:1}
\end{equation}

多个公式,公式引用:\autoref{eq:2},\autoref{eq:3}。

\begin{subequations}
\begin{equation}
  F = ma \label{eq:2}
\end{equation}
\begin{equation}
  E = mc^2 \label{eq:3}
\end{equation}
\end{subequations}

\section{罗列环境}

\begin{enumerate}
    \item 第一层\label{item:1}
    \item 第一层
    \begin{enumerate}
        \item 第二层\label{item:2}
        \item 第二层
        \begin{enumerate}
            \item 第三层\label{item:3}
            \item 第三层
        \end{enumerate}
    \end{enumerate}
\end{enumerate}

\begin{description}
    \item[解释环境]  解释内容
\end{description}



\clearpage
\chapter{实时操作系统VxWorks}
\section{概述}
\subsection{实时操作系统}
实时操作系统(Real Time Operation System)是整个实时系统的核心。POSIX1003.1标准为RTOS下了一个简单的定义:RTOS是能够在有限的响应时间内为应用提供所要求级别服务的操作系统\cite{Renard20081003}。这个系统能够对任何时间要求苛刻的事件服务,能够在正确的时间内做正确的事情。实时系统按照实时的效果可以分为软实时和硬实时,一个好的实时操作系统能够使你的系统时钟满足要实现的需求,即使在系统的负荷很重的情况下。

现代实时操作系统基于多任务和任务间通信的互补概念。多任务环境允许将实时应用程序构建为一组独立任务,每一个任务都有自己的执行线程和一组系统资源。任务间通信设施允许这些任务同步并进行协调其活动。在VxWorks中,任务间通信工具从快速信号量到消息队列,从管道到网络透明套接字。

	实时系统的另外一个关键设施是硬件中断处理,因为中断处理时通知系统发生外部时间的常用机制。为了尽可能快地响应中断,VxWorks中的中断服务例程(ISR)运行自己的特定上下文中,这个上下文在任何其他的任务上下文之外\cite{Wind2003VxWorks}。

操作系统的核心是内核。内核控制着计算机系统上的所有硬件和软件资源,在必要的时候给应用程序分配硬件资源,并执行相应的操作命令。

	内核的主要功能为以下的四种:

\bullet 系统的内存管理:不仅可以管理服务器上的可用物理内存,还可以创建和管理虚拟内存。

\bullet 软件程序管理:内核控制着系统上运行着的所有程序。

\bullet 硬件设备管理:任何需要操作系统与之进行通信的设备都需要在内核的代码当中加入其驱动程序代码。驱动程序代码相当于应用程序和硬件设备间的中间人,允许内核和设备之间交换数据。

\bullet 文件系统管理:不同于其他的一些操作系统,Linux内核支持通过不同类型的文件系统从硬盘当中读写数据。除了自有的诸多文件系统之外,Linux还支持从其他的操作系统(如windows)采用的文件系统中读写数据。内核必须在编译时就加入对所有可能用到的文件系统的支持。Linux内核采用虚拟文件系统(Virtual File System,VFS)作为和每一个文件系统交互的接口。这为Linux内核同任何类型文件系统通信提供了一个标准的接口,当每个文件系统都被挂载和使用时,VFS将信息都缓存在内存当中。

VxWorks是一种基于微内核技术的实时操作系统。


\section{代码环境}

\begin{lstlisting}[language=python]
import os

def main():
    '''
    doc here
    '''
    print 'hello, world' # Abc
    print 'hello, 中文' # 中文
\end{lstlisting}

\section{定律证明环境}

\begin{definition}\label{def:1}
这是一个定义。
\end{definition}
\begin{proposition}\label{proposition:1}
这是一个命题。
\end{proposition}
\begin{axiom}\label{axiom:1}
这是一个公理。
\end{axiom}
\begin{lemma}\label{lemma:1}
这是一个引理。
\end{lemma}
\begin{theorem}\label{theorem:1}
这是一个定理。
\end{theorem}
\begin{proof}\label{proof:1}
这是一个证明。
\end{proof}

\section{算法环境}

\begin{algorithm}[H]
\SetAlgoLined
\KwData{this text}
\KwResult{how to write algorithm with \LaTeX2e }
initialization\;\label{alg_line:1}
\While{not at end of this document}{
read current\;
\eIf{understand}{
go to next section\;
current section becomes this one\;
}{
go back to the beginning of current section\;
}
}
\caption{How to write algorithms}\label{alg:1}
\end{algorithm}

\section{表格}
表格见\autoref{tab:1}。

\begin{table}[!h]
\centering
\caption{一个表格}\label{tab:1}
\begin{tabular}{|c|c|}
\hline
a & b \\
\hline
c & d \\
\hline
\end{tabular}
\end{table}
\section{图片}
图片见\autoref{fig:1}。图片格式支持eps,png,pdf等。多个图片见\autoref{fig:2},分开引用:\autoref{fig:2-1},\autoref{fig:2-2}。

\begin{figure}[!h]
\centering
\includegraphics[width=.4\textwidth]{fig-example.pdf}
\caption{一个图片}\label{fig:1}
\end{figure}

\begin{figure}[!h]
\centering
  \begin{subfigure}[b]{0.3\textwidth}
  \includegraphics[width=\textwidth]{fig-example.pdf}
  \caption{图片1}\label{fig:2-1}
  \end{subfigure}
  ~
  \begin{subfigure}[b]{0.3\textwidth}
  \includegraphics[width=\textwidth]{fig-example.pdf}
  \caption{图片2}\label{fig:2-2}
  \end{subfigure}
\caption{多个图片}\label{fig:2}
\end{figure}

\section{参考文献示例}
这是一篇中文参考文献\cite{徐媛媛2003嵌入式实时操作系统的设备驱动};这是一篇英文参考文献\cite{9787508342894};同时引用\cite{9780124467422,bamboosilk}。

\section[\textbackslash{}autoref 测试]{\texttt{\textbackslash{}autoref} 测试}

\begin{description}
  \item[公式] \autoref{eq:1}
  \item[脚注] \autoref{footnote:1}
  \item[项] \autoref{item:1},\autoref{item:2},\autoref{item:3}
  \item[图] \autoref{fig:1}
  \item[表] \autoref{tab:1}
  \item[附录] \autoref{appendix:1}
  \item[章] \autoref{chapter:1}
  \item[小节] \autoref{sec:1},\autoref{sec:2},\autoref{sec:3}
  \item[算法] \autoref{alg:1},\autoref{alg_line:1}
  \item[证明环境] \autoref{def:1},\autoref{proposition:1},\autoref{axiom:1},\autoref{lemma:1},\autoref{theorem:1},\autoref{proof:1}
\end{description}

% backmatter用于表示论文的正文结束
\backmatter

%ack 环境用于致谢页面
\begin{ack}
致谢正文。
\end{ack}

% bibliography用于生成参考文献。
\bibliography{ref/myref.bib}

% appendix环境用于附录环境。即可以将附录置于appendix环境当中。如:
% \begin{appendix}
%  <content>
% \end{appendix}

% 直接使用\appendix 则表明后文均为附录。如:
% \appendix
%  <content> 
\appendix

% publications环境用于已经发表了的论文的页面,一般用于附录当中,使用上同enumerate环境
\begin{publications}
    \item 论文1
    \item 论文2
\end{publications}

\chapter{这是一个附录}\label{appendix:1}
附录正文。


\end{document}

\endinput
%%
%% End of file `hustthesis-zh-example.tex'.
