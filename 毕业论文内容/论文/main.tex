
% !TeX program = xelatex

%使用hustthesis这个模板。
% draft版本的正文页包括页眉(“华中科技大学xx学位论文”)、页眉修饰线(双线)、页脚(页码)和页脚修饰线(单线)。
% final版本的正文页不包括页眉、页眉修饰线和页脚修饰线,仅包 含页脚(页码)。如果不指定,默认设置为final。

% degree用来指定论文的种类
% language用来指定论文语言。特别的,如果设定为english-draft,将会剔除论文中的所有中文内容,这有利于在未安装中文字体的环境中使用。如果不指定,默认设置为 chinese。


\documentclass[format=draft,language=chinese,degree=master]{hustthesis}
%\documentclass[format=final,language=chinese,degree=master]{hustthesis}

%使用下面的这两个宏包来生成带书签和超链接的PDF文件。
%\usepackage[bookmarks,bookmarksopen,bookmarksdepth=2]{hyperref}
%\usepackage[pdftex,CJKbookmarks=true,colorlinks=true]{hyperref} %LaTeX Error: Option clash for package hyperref

\stuno{M201672905}
\schoolcode{10487}
\title{基于Vxworks的调试通道的设计与实现}{A design and implementation of debug channel based on Vxworks.}
\author{郑松}{Joe}
\major{计算机应用技术}{Computer Applications Technology}
\supervisor{张杰\hspace{1em}讲师}{Instructor JieZhang}
\date{2018}{3}{26}
%\date{\today}

\zhabstract{数据传输是现代通讯过程中的一个重要环节,在数据的传输过程中,不仅仅要求数据传输的准确率要高,而且要求速度快,连接方便。传统的RS232串口通讯和并口通讯都存在传输速度低,扩展性性差、安装麻烦等缺点,而基于USB接口的数据传输系统能够较好的解决这些问题。目前USB接口以其传输速率高、即插即用、支持热插拔等优点,逐步成为PC机的标准接口。
	
本文中的数据传输系统采用了USB接口进行上位机与下位机之间的数据通讯。下位机采用的是VxWorks实时操作系统。

}
\zhkeywords{实时操作系统,设备驱动,USB口转串口,VxWorks,CP2103}

\enabstract
{
    This is a \LaTeX{} template example file. This template is used in written thesis for Huazhong Univ. of Sci. \& Tech.

    This template is published under LPPL v1.3 License.

}
\enkeywords
{\LaTeX{}, Huazhong Univ. of Sci. \& Tech., Thesis, Template}

\begin{document}

% frontmatter用于设定论文的状态、改变样式,其具体使用见简单示例。
% frontmatter用在文档最开始,表明文档的前言部分(如封面,摘要,目录等)的开 始。
\frontmatter

% maketitle的作用和makecover的作用相同,用于生成封面和版权页面
\maketitle

% makeabtract用于生成中英文的摘要页面。
\makeabstract

% tableofcontents用于生成目录
\tableofcontents

% listoffigures 和 listoftables 分别用于生成图片和表格索引,可以根据要求在论文的前言中使用或者是在附录中使用
%\listoffigures
%\listoftables

% mainmatter表示论文正文的开始。
\mainmatter
\clearpage

\chapter{绪论}\label{chapter:1}

VxWorks操作系统是美国WindRiver公司于1983年推出的一种嵌入式实时操作系统(RTOS),是一个运行在目标机上的高性能、可裁剪的实时操作系统,是一个专门为嵌入式实时系统设计开发的操作系统,其具有良好的持续发展能力、高性能的内核以及友好的用户开发环境,为开发人员提供了高效的实时多任务调度、中断管理、实时的系统资源以及实时的任务间通信。在嵌入式实时操作系统领域当中占有一席之地。VxWorks支持X86、PowerPC、ARM等众多主流的处理器,且在各种CPU平台上提供了统一的编程接口和一致的运行环境。在军事、航空航天、工业控制、通信等高精尖以及实时性要求极高的领域当中,有着更加广泛而深入的应用。应用实例包括1997年火星表面登入的火星探测器、爱国者导弹、飞机导航、F-16、FA-18战斗机等。自从对我国的销售解禁之后,VxWorks也大量的应用于我国的军事、国防工业当中,通常在进行VxWorks应用程序的开发或者是将Linux下的应用程序移植到

\section{课题背景以及意义}\label{sec:1}
\section{国内外概况}\label{sec:2}

\section{论文的主要内容和组织结构}\label{sec:3}

\clearpage
\chapter{调试通道总体设计与关键技术}

\section{总体设计}
	VxWorks的集成开发调试环境为Tornado,它使用串口和网口结合的方式来对目标机进行控制和数据传输,而目前对于大多数的设备而言都已经抛弃了串口,很多用于军事上的嵌入式设备都是专用设备,没有联网的需求,并不会配备网口,但是对于设备上产生的各种调试信息、日志信息都需要传输到我们的windows PC上来进行一个事后分析工作,因此我们需要设计一个新的调试通道来满足这些信息的传输要求,我们本次设计一个基于USB口转串口的底层驱动来实现该调试通道。
	调试通道的总体结构如\autoref{fig:debug-system-diagram}所示。
\begin{figure}[!h]
\centering
\includegraphics[width=.9\textwidth]{./graphics/debug-system-diagram.pdf}
\caption{调试通道整体结构图}\label{fig:debug-system-diagram}
\end{figure}
	
	整个调试通道主要分为两个模块:应用层的接口模块、USB口转串口模块。
	
	提供给应用层的接口模块负责将系统应用层的输出通过我们的USB口转串口驱动程序传输到windows PC机,输出的形式包括特定内容的格式化的输出和普通的重定向的输出,格式化的输出我们会使用自定义的Log接口进行格式控制,为此我们设计了一个自定义的Log协议格式,其中的内容包含有调试级别、调试信息所在的文件、调试信息所处的行号、输出该条调试信息的时间等;重定向的输出包括RTP模式下的重定向和task模式下的重定向,VxWorks中对于这两种模式需要使用不同的重定向方式。
	
	USB口转串口模块用于在VxWorks上实现一个USB口转串口驱动程序,负责将上层应用的信息传输到windows PC,包括一个特定需求的驱动程序一个普通的驱动程序。特殊需求的驱动程序相对于普通的驱动程序在流程和结构上进行了修改,以使其达到要求,具体的实现我们会在第三节进行介绍。两种实现方式中都会包含有驱动程序加载、卸载模块,设备的打开、关闭、读、写、控制模块。同时在驱动程序中还需要一个数据的管理模块,我们会使用循环缓冲区来管理数据。


\section{关键技术}

\subsection{VxWorks驱动开发}
	
	在VxWorks当中使用I/O子系统来管理设备驱动,I/O 子系统在整个VxWorks当中起着承上启下的作用,各种类型的设备都必须要向I/O子系统进行注册才能够被内核访问/cite{VxWorks内核解读}\cite{曹桂平2011VxWorks}。I/O子系统负责维护设备驱动中三个非常重要的数据结构:系统设备表、系统驱动表、系统文件描述符表,设备驱动程序初始化时会对硬件完成初始化的配置,同时会向I/O子系统注册自己,注册之后I/O子系统才能找到该驱动。我们从VxWorks 下的 I/O 系统和驱动程序的关系入手,分析 VxWorks 下 I/O 系统调用和驱动程序的实现过程,在VxWorks中设备驱动的访问过程如下:
\begin{enumerate}
\item 调用open()函数打开一个设备(假设设备名为/ttyUsb),I/O 系统会在系统的设备表中寻找这个名为/ttyUsb的设备项,并找到相应的驱动号; 
\item I/O系统在文件描述符当中保留一个文件描述符,然后在系统的设备驱动表中找到该设备注册的设备打开函数,调用这个函数,并返回设备描述符的指针;
\item I/O系统将设备描述符的指针存储在文件描述符列表的Device ID中,同时将对应的设备驱动号存储在文件描述符的Driver Num项。最后I/O系统返回该描述符的索引(即fd);
\item 应用程序当中使用这个fd来调用read()、write()函数。系统会根据fd自动找到相应的设备驱动号,进而找到相应的驱动例程\cite{解月江2004VxWorks设备驱动技术研究}。 
\end{enumerate}

\subsubsection{VxWorks I/O 系统}
	通常操作系统为了平台的无关性都会为应用程序提供一套标准的接口,VxWorks也不例外,它为应用层的提供了接口函数creat()、open()、unlink()、remove()、close()、rename()、read()、write()、ioctl()、lseek()、readv()、writev()等\cite{陈洋2007VxWorks}\cite{Wu2008Implementation}\cite{Zhang2010Design},我们将其称作为标准I/O 库。
	这样就可以通过调整底层驱动或者是接近驱动那部分的操作系统中间层来实现应用程序的通用性,提高应用层的开发效率,避免重复编码。在Mac OS、Linux、或Windows当中会把这套接口以标准库的形式呈现,但是在VxWorks中它们是由系统的内核实现的,都位于ioLib.c文件下。VxWorks与通用操作系统有很大的一个不同点是:VxWorks不区分用户态和内核态,用户层可以直接对内核函数进行调用,而无需使用陷阱指令之类的机制,以及存在使用权限上的限制。因此VxWorks提供给应用层的接口无需通过外围库的方式,而是直接以内核文件的形式提供\cite{VxWorks内核解读}。对于一般的设备而言,remove()接口是不需要实现的,I/O调用结构如\autoref{fig:I/O调用}所示。
	\begin{figure}[!h]
\centering
\includegraphics[width=1.0\textwidth]{./graphics/IOCall.pdf}
\caption{I/O调用}\label{fig:I/O调用}
\end{figure}

	
\subsubsection{系统设备表}
	系统设备表是VxWorks中为了管理系统上的所有设备而使用的一个链表。系统设备表在系统中的连接方式如\autoref{fig:VxWorks系统设备示意图}所示。wind内核规定每一个设备都必须使用一个DEV\_ HDR的结构来表示该设备,其定义如下\cite{VxWorks内核解读}:
\lstset{language=C}
\begin{lstlisting}
/*h/iosLib.h*/
typedef struct
{ 
  DL_NODE node; /* 设备链表节点 */ 
  short drvNum; /* 设备的驱动号 */ 
  char * name;/* 指向设备名 */ 
}DEV_HDR; /* 这个结构是所有设备自定义结构体的第一个成员 */
\end{lstlisting}
VxWorks系统提供了一个设备的注册函数iosDevAdd( DEV\_ HDR *pDevHdr, char *name, int drvnum)来将设备添加到系统设备表当中,系统设备表在每次添加设备时就会增加一个节点,删除设备时就会减少一个节点,它会为open()、close()、remove()这三个函数提供文件与设备的连接,当应用程序执行这三个函数中的一个时,IO系统会通过文件名对设备链表中的项进行匹配\cite{刘小军2008基于},VxWorks中使用最佳匹配的方式进行设备名匹配,匹配成功之后就使用这个设备驱动进行其他的文件操作。


\begin{figure}[!h]
\centering
\includegraphics[width=1.0\textwidth]{./graphics/vxworks-device-link.pdf}
\caption{VxWorks系统设备示意图}\label{fig:VxWorks系统设备示意图}
\end{figure}

\subsubsection{系统驱动表}	
	系统驱动表用于管理当前注册到I/O子系统下的所有驱动程序,这些驱动可以是直接驱动硬件工作的驱动层,如一般的字符驱动,也可以是驱动中间层,如文件系统中间层,TTY 中间层,USB IO 中间层等。对于中间层驱动,下层硬件驱动将由这些中间层自身负责管理,而不再通过 IO 子系统。如串口底层驱动将通过 TTY 中间层进行管理,而不再通过IO子系统\cite{VxWorks内核解读}\cite{罗国庆2003VxWorks}。
	
	VxWorks中使用数组来实现系统驱动表,表中的每一个表项都是一个 DRV\_ ENTRY 类型的结构,该结构定义在 h/private/iosLibP.h文件当中,其定义如下\cite{VxWorks内核解读}:
\lstset{language=C}
\begin{lstlisting}
typedef struct  
{ 
  FUNCPTR de_create; /* 指向驱动的create()函数 */
  FUNCPTR de_delete; /* 指向驱动的delete()函数 */
  FUNCPTR de_open; 	 /* 指向驱动的open()函数 */
  FUNCPTR de_close;  /* 指向驱动的close()函数 */
  FUNCPTR de_read;   /* 指向驱动的read()函数 */
  FUNCPTR de_write;  /* 指向驱动的write()函数 */
  FUNCPTR de_ioctl;  /* 指向驱动的ioctl()函数 */
  BOOL de_inuse;     /* 用于指示表项是否空闲*/
} DRV_ENTRY;/* 系统驱动表中的条目 */ 
\end{lstlisting}
DEV\_ ENTRY结构体类型实际上就是一个函数指针结构,结构中每个成员都指向一个完成特定功能的函数,这些函数与用户层提供标准函数接口一一对应\cite{VxWorks内核解读}\cite{VxWorksDriverAPI}\cite{Wind2003VxWorks}。若成员 de\_ inuse 空闲则表示该表项未被使用。这个结构体中的函数指针实际指向的内容由驱动调用iosDrvInstall()函数来提供。 该函数的作用就是将驱动注册到系统驱动表当中,并根据注册的内容来填充DRV\_ ENTRY表项的内容。


\subsubsection{系统文件描述符表}
	系统描述符表用于管理当前系统打开的所有文件描述符。其底层实现也是一个数组,文件描述符表的表项索引被用作文件描述符的ID(即open()函数返回值),系统中每次使用open()调用就会占用一个系统文件描述符的表项。对于文件描述符有一点需要注意:标准输入,标准输出,标准错误输出虽然使用 0,1,2 三个文件描述符,但是可能在系统文件描述附表中只占用一个表项,即都使用同一个表项\cite{基于VxWorks的嵌入式实时系统设计}\cite{VxWorks内核解读}\cite{An2003Implementation}。
		
	
系统文件描述符表中每一个表项都使用 FD\_ ENTRY 这个结构来表示,这个结构定义在h/private/iosLibP.h 中,其定义如下\cite{VxWorks内核解读}:
\lstset{language=C}
\begin{lstlisting}
typedef struct  
{ 
  DEV_HDR * pDevHdr;/* 该设备的设备头 */ 
  int value; /* 设备的驱动号 */ 
  char * name; /* 设备名指针 */ 
  int taskId; /* 设备的任务ID */ 
  BOOL inuse;  
  BOOL obsolete; /* 底层驱动是否仍然存在 */ 
  void * auxValue;/* 驱动自定指针,根据驱动的需求实现 */ 
  void * reserved; /* 保存未用 */ 
} FD_ENTRY; /* 系统文件描述符表的表项 */
\end{lstlisting}


用户的应用程序每次使用open()系统调用都会在系统文件描述符表中就增加一个有效表项,该表项的FD\_ ENTRY结构体会根据open()调用的内容来进行填充,每一个文件能够进行的open()调用是有限制的,每个驱动的FD\_ ENTRY结构数组满了之后就无法再对这个设备进行open()操作,此时 open()函数将会失败返回。系统会在表中的索引偏移 3 之后找一个最先找到的未使用的id作为文件描述符返回给用户,之后用户对设备的其他操作只需要使用这个文件描述符作为文件句柄即可。	
	
\subsection{VxWorks 中的通信机制}
	
	wind内核提供了三种任务间通信机制:信号量,消息队列,管道。这三种机制都是本质上都是使用的共享物理内存机制,这块共享的内存是由内核进行管理的,任务必须通过内核提供的接口函数进行访问,这种保护和管理机制使任务间通信安全有序的进行\cite{胡明民2012基于实时操作系统}。

\begin{enumerate}
	\item \textbf{信号量}
	
	信号量是一种在程序的设计当中需要使用的通信机制,其主要用于线程间的互斥和同步。,VxWorks的信号量机制提供了三种具体的具体的实现,分别是通用信号量、互斥信号量、资源计数信号量,他们有各自不同的特点,适用于不同的场景。通用信号量既可用于同步也可用于资源计数,此时资源数通常为 1(当资源数为 1 时,也可以称之为互斥)。互斥信号量针对在使用过程中一些具体问题(如优先级反转)做了优化,更好的服务于任务间互斥需求;资源计数信号量用于资源数较多,同时可供多个任务使用的场合\cite{冯云贺2014基于}。
	
	VxWorks中信号量是一个指向semaphore类型的结构指针,其定义如下所示\cite{胡明民2012基于实时操作系统}:
\lstset{language=C}
\begin{lstlisting}
typedef struct semaphore
{
  OBJ_CORE objCore;/*对象管理*/
  UINT8 semType; /*信号量类型*/
  UINT8 options; /*信号量选择*/
  UINT16 recurse; /*信号量重复获取计数器*/
  Q_HEAD qHead; /*阻塞的任务队列头*/
  union{
	UNIT count;/*当前状态*/
	struct windTcb *owner;  
  }state;
  EVENTS_RSRC events;/*VxWorks事件*/
}SEMAPHORE;
\end{lstlisting}
	
	\item \textbf{消息队列}
	
	消息队列是一种在消息传输的过程中保存消息的容器,在wind内核中使用一个结构数组来实现消息队列,这就使得其在任务间传递较多信息时存在的很大的局限性,因为数组大小和数组中元素的容量是确定的,即每个消息的最大长度是固定的,而且消息队列必须将信息分批打包。Vxworks内核提供的消息机制在创建消息队列时就必须指定单个消息的最大长度以及消息的数量,在消息队列成功创建后,这些参数都是固定不变的\cite{冯云贺2014基于}。
	
	\item \textbf{管道}
	
	管道也是一种基本的进程间通信机制,包括命名管道和匿名管道。VxWorks内核当中使用环形队列的方式来实现管道,管道提供了比消息队列更流畅的信息传递机制,可以像文件一样进行读写。命名管道具有一个与之关联的路径名,因此任何的进程间都可以用它进行通信,命名管道是双工的数据可以双向流动;非命名管道一般用于父子或兄弟进程间通信,非命名管道是半双工的,数据只能向一个方向流动。
\end{enumerate}\\		
\textbf{任务间特殊的通信机制--信号:} 信号通常用于通知一个进程发生了异步事件,也被称为软中断。通常收到信号的进程通常可以选择三种方式来处理:一是使用一个信号处理函数处理;二是选择忽略该信号;三是使用系统默认的处理方式处理。
在Vxworks中的信号处理机制有些特别之处,对于SIGKILL,SIGSTOP这类的信号,在通用操作系统上是不允许用户修改其默认处理函数的,但是在 VxWorks 操作系统中可以对任何信号的处理函数都可以进行更换的。



\subsection{USB技术}
	USB(Universial Serial Bus)作为PC领域的最新型的接口技术,目前已被各个PC厂家所支持,并且在各类外设当中都广泛的采用USB接口。USB的开发技术也已经很成熟,通用串行总线开发者论坛(USB Implementers Forum,USB IF)目前制定了三种USB接口标准:USB1.1,USB2.0和USB3.0。USB采用菊花链的形式连接所有的设备,最多可以连接127个设备,USB的总线拓扑结构如\autoref{fig:USB体系结构}所示
\begin{figure}[!h]
\centering
\includegraphics[width=1.0\textwidth]{./graphics/usb-structure.pdf}
\caption{USB总线拓扑结构}\label{fig:USB体系结构}
\end{figure}


USB的体系结构包括三个部分:USB主机(Host)、USB集线器(Hub)、USB设备(Device)。
	
\begin{enumerate}
\item \textbf{USB主机}\\
	USB主机是USB体系中的核心,且系统中只允许一个USB主机存在。USB主机上的USB接口是USB主控制器,其控制着总线上所有USB设备数据通信。对于USB的体系结构而言,其数据的传输都是USB主机端发起的,非主机端(设备端)只能够被动的进行响应。USB主机需要完成的功能包括检测设备的热插拔、管理主机和设备之间的信息(控制和数据)流\cite{李雪红2004USB}\cite{莫宏伟2001USB}。

\item \textbf{USB集线器}\\	
	USB集线器的作用是扩展USB的通讯能力,他可以提供更多的接入口,USB主机上的集线器被称为根集线器,集线器大大的简化了USB的复杂性,而且以很低的价格和易用性提供了设备的健壮性\cite{李雪红2004USB},集线器的最大的连接能力是127。


\item \textbf{USB设备}

	USB设备指的是提供具体功能的而外部USB设备,是相对USB主机而言的,它们受USB主机的控制,只能对主机的请求进行被动响应。USB主机端会通过协议和USB设备通信,对设备进行配置,并给设备提供驱动程序,USB设备通过以下的属性来完成主机的配置要求:
	\begin{itemize}
	\item \textbf{描述符(Descriptor)}\\
	USB协议中定义了一套描述USB设备的功能和属性的固定结构的描述符,包括设备描述符、配置描述符、接口描述符、端点描述符、字符串描述符\cite{张杰2008基于}\cite{边海龙2004USB}除此之外,设备还可以提供自己专用的描述符,分为设备类描述符和供应商自定义描述符,我们使用的USB口转串口设备就不属于一个标准的USB设备,它会为我们提供供应商自定义的描述符,我们之后需要用它来对设备进行识别。
	
	\item \textbf{类(Class)}\\
	USB协议支持许多的外围设备,为了正确的驱动这些设备,USB主机端要为这些设备提供符合USB协议的驱动程序,称为客户端驱动。同时为了避免客户端程序过多,协议通过归纳将设备划分为不同的设备类,把功能相近的设备归为一类,主机端只需要提供类驱动程序便可以驱动大多数的USB设备\cite{李雪红2004USB}。	
	
	\item \textbf{功能(Function)/接口(Interface)}\\
	USB协议中将功能或接口定义为具有某种能力的设备,FUnction是从功能角度来说的,从设备的角度来说,被称为Interface。一个接口负责完成设备的一个特定的功能,并且是可替换的,当USB设备处于可配置状态是能够改变其功能,对于USB设备的每一个接口都必须要有一个接口描述符来描述。
	
	\item \textbf{端点(Endpoint)}\\
	端点是USB设备与USB主机逻辑上的数据传输的通信流的终点,每个设备都拥有一个可独立进行操作的端点集合,且每个端点在使用时都要先初始化其数据传输方向(IN/OUT),即使端点号相同但是传输方向不同的通信点也是不同的端点。端点0被USB规范保留用作设备枚举和配置过程中的数据传输端点,与端点0对应的管道是默认管道,设备的所有端点共享端点0\cite{李雪红2004USB}。
	
	\item \textbf{管道}\\
	管道是设备上的一个端点和主机上的软件的联合体,是一个具有特定数据传输特性(如格式、带宽、方向等)的数据通道,设备和主机间的数据传输要基于管道进行。对USB设备进行配置时就需要建立传输管道,在我们的USB口转串口驱动中会为每一个设备建立两个管道,一个批量输出管道和一个批量输入管道。另外端点0会自动的拥有一个缺省管道。
	\item \textbf{设备地址}\\
	设备地址用于区分USB系统中的一个USB设备,设备地址由主机分配且是唯一的。设备地址共有7位,地址0是缺省地址,在设备初始化的时候使用,理论上系统可以区分127个USB设备,实际中由于USB总线带宽的限制,无法同时支持这么多设备工作\cite{李雪红2004USB}。
	\end{itemize}	
	
\end{enumerate}



\noindent USB规范规定了USB主从设备之间的四种传输方式,每种方式有各自的用途\cite{USB总线接口开发指南}:
\begin{itemize}
\item \hei{控制传输}:控制传输USB传输方式中最重要、最复杂的一种,它适用于少量、对时间和速率无要求的场合,一个USB设备插入主机之后就是使用这种传输方式来读取设备的地址和描述符等信息。所有的设备都会在其0号端点的缺省管道当中支持控制传输\cite{张杰2008基于}。
\item \hei{批量传输}:批量传输有两种最基本的事物类型:BULK\_ IN和BULK\_ OUT,其主要用于处理对数据传输速率不是很高的情况,批量传输使我们的USB口转串口设备所使用的主要传输传输方式,每次有数据需要传输时我们都会构建一个IRP使用批量传输将其传出或传入。
\item \hei{中断传输}:中断传输也有两种基本的事务:IN和OUT,其主要是为那些要快速实现主机和设备的交互,但是数据量很小、对服务时间有要求的情况而准备的。
\item \hei{等时传输}:等时传输也是由基本的IN和OUT两种事务组成,主要用于处理大量、恒速、对时间周期有要求的数据。等时传输只有全速和高速设备才支持,低速设备不支持\cite{张杰2008基于}。
\end{itemize}


	

\section{本章小结}
	本章重点介绍了本次的VxWorks调试通道的整体架构,并介绍了介绍了各个部分的设计方案,最后介绍了在本次的设计当中所需要使用关键技术和所需了解的重要知识,主要包括VxWorks下的驱动开发必须的结构、驱动中所需使用的VxWorks的通信机制、缓冲区技术、USB技术。下面将要讨论VxWorks下的调试通道的详细的设计细节和具体的实际机制。



























\clearpage
\chapter{驱动程序的设计和实现}
	
\section{USB口转串口驱动的设计}

	USB口转串口驱动程序的编写与操作系统的关系密不可分。设备驱动程序在操作系统中如何存在、如何与操作系统的其它部分相联系、如何与操作系统的其他部分相联系、如何为用户提供服务都是操作系统的设计人员在设计操作系统时制定的,系统已经为驱动程序制定好了一个框架,无论驱动程序的开发人员以何种方式控制设备,他们所开发的驱动程序都是以预先设计好的方式存在、与操作系统其他部分相联系和为用户提供服务的\cite{徐媛媛2003嵌入式实时操作系统的设备驱动}。我们此次驱动程序需要实现的框架部分主要包括cp210xDrvInit()、cp210xDevOpen()、cp210xDevClose()、cp210xDevIoctl()、cp210xDevWrite()、cp210xDevRead()、cp210xDrvUnInit()等函数,如\autoref{lab:驱动程序的关键模块}所示 ,以及用于处理数据的缓冲区、用于进行同步和互斥操作的信号量。
\begin{table}[!h]
\centering
\begin{tabular}{|c|c|}
\hline
{模块} & {作用} \\
\hline
{cp210xDrvInit()} & \tabincell{c}{这个模块用来初始化驱动程序,主要是与设备无关的一些\\全局变量并向系统注册该驱动} \\
\hline
{cp210xDevOpen()} & \tabincell{c}{这个模块用来转接I/O子系统分发过来的open()操作,实现设备的打开,\\返回文件描述符} \\
\hline
{cp210xDevClose()} & \tabincell{c}{这个模块用来转接I/O子系统分发过来的close()操作,实现设备的关闭,\\对设备占用资源进行清理} \\
\hline
{cp210xDevIoctl()} & \tabincell{c}{这个模块用来转接I/O子系统分发过来的ioctl()操作,\\实现对设备的一些特定的控制操作} \\
\hline
{cp210xDevWrite()} & \tabincell{c}{这个模块用来转接I/O子系统分发过来的write()操作,\\实现对设备进行数据的写入} \\
\hline
{cp210xDevRead()} & \tabincell{c}{这个模块用来转接I/O子系统分发过来的read()操作,\\用于从设备读取数据。} \\
\hline
{cp210xDrvUnInit()} & \tabincell{c}{这个模块用来卸载驱动程序,将驱动从系统驱动表中删除,\\并清理该驱动程序所占用的全部资源} \\
\hline
\end{tabular} 
\caption{驱动程序的关键模块}\label{lab:驱动程序的关键模块}
\end{table}
	
	另外一个方面,USB口转串口的驱动程序需要硬件来作为支撑,PC机上本身并没有USB/RS-232的转换器,
	对于USB/RS-232转换器的设计通常有两类实现方式:一类是采用全面系统的设计,使用包含有USB单元的微处理器,要求它具有内置的通用异步收发器(UART)在USB和RS-232之间进行信号转换,这样的控制器包括PCI16C745、68HC705JB4和C541U系列等\cite{USB与RS232接口转换器的设计},也可以采用USB接口芯片如PDIUSBD12、USBN9604等与微控制器组合工作;第二类方法是采用专用的USB/RS-232双向转换芯片,如CP2102、FT232BM等,我们在此处的设计即使用了CP2102芯片作为USB/RS-232转换器,这样设计的好处是不需要编写转换器芯片的固件,节约开发时间,由于这个技术已经很成熟,大多的USB口转串口的解决方案都会采用这种已经设计好的集成芯片来作为转换器。我们本次使用的设备如\autoref{fig:cp2102模块正反面}所示。
\begin{figure}[h]
\centering
  \begin{subfigure}[b]{0.4\textwidth}
  \includegraphics[width=\textwidth]{./graphics/cp2102Front.pdf}
  \caption{CP2102模块正面}\label{fig:cp2102Front}
  \end{subfigure}
  ~
  \begin{subfigure}[b]{0.4\textwidth}
  \includegraphics[width=\textwidth]{./graphics/cp2102Rear.pdf}
  \caption{CP2102模块反面}\label{fig:cp2102Rear}
  \end{subfigure}
\caption{CP2102模块正反面}\label{fig:cp2102模块正反面}
\end{figure}




\subsection{CP2102开发}
	
	CP2102是SILICON LABORATORIES推出的USB与RS232接口转换芯片,是一种专门用来进行USB转UART的高度集成的桥接器,和其他同类型的芯片相比具有功耗更低、体积更小、集成度更高(仅需少量外部元件)、价格更低等优点。因此我们此次选择这个芯片作为调通道的设计当中使用的芯片。CP2102提供一个使用最小化的元件和PCB空间实现RS232转USB的简便的解决方案。
	CP2102芯片包含有一个USB2.0全速功能控制器,EEPROM,USB收发器,振荡器和带有全部的调制解调器控制信号的异步串行数据总线(UART),CP2102将全部的部件集成在一个5mm*5mm MLP-28封装的IC当中\cite{CP2102},CP2102的电路框图如\autoref{fig:cp2102电路框图}所示。

\begin{figure}[!h]
\centering
\includegraphics[width=1.0\textwidth]{./graphics/cp2102-circuit-diagram.pdf}
\caption{cp2102电路框图}\label{fig:cp2102电路框图}
\end{figure}

	使用CP2102进行串口扩展的时候所需要的外部器件是非常少的,仅仅需要2-3个去耦电容即可,在SILICON给出的文档当中已经帮我们给出了一个最简单的连接电路图,如\autoref{fig:cp2102电路框图}所示。电路使用CP2102UART总线上的TXD/RXD两个引脚,其余的引脚都悬空。CP2102可以完成USB/RS-232双向转换(需要外接一个TTL电平到RS-232电平的芯片),一方面可以从主机接收USB数据并将其转换为RS232信息格式流发送给外设,另外一方面可以从RS232外设接收数据转换为USB数据格式传送回主机。使用时我们只需要将数据通过USB的数据包发送给CP2102芯片即可,芯片会自动进行解析和控制。
		
\begin{enumerate}
\item CP2102的USB功能控制器和收发器:CP2102的USB功能控制器是一个符合USB2.0协议的全速器件,这个器件负责管理USB和UART之间的所有数据传输以及由USB主控制器发出的命令请求和用于控制UART功能的命令。
\item 异步串行数据总线(UART)接口:CP2102的UART接口包括TXD(发送)和RXD(接收)数据信号以及RTS,CTS,DSR,DTR,DCD和RI控制信号。ART支持RTS/CTS,DSR/DTR和X-on/X-Off握手。且支持编程使UART支持各种数据格式和波特率。ART的数据格式和波特率的编程是在PC的COM口配置期间进行的。可以使用的数据格式和波特率见\autoref{CP2102可配置参数}。
\item 内部EEPROM:CP2102内部集成了一个EEPROM用于存储设备原始制造商定义的USB供应商ID、产品ID、产品说明、电源参数、器件版本号和器件序列号等信息\cite{CP2102}。USB配置数据的定义是可选的,如果EEPROM没有被OEM的数据所填充的话,则设备会自动的使用一组默认的数据如\autoref{CP2102DefaultConfigure}所示。
\end{enumerate}

\begin{table}[!h]
\centering
\begin{tabular}{|c|c|}
\hline
{数据位} & {5,6,7,8} \\
\hline
{停止位} & {1,1.5,2} \\
\hline
{校验位} & {无校验,偶校验,奇校验,标志校验,间隔校验} \\
\hline
{波特率} & \tabincell{c}{600,1200,2400,4800,7200,9600,14400,16000,19200,28800,\\ 38400,51200,56000,57600,64000,76800,115200,128000,158600,\\ 230400,250000,256000,4608000,576000,921600}\\
\hline
\end{tabular} 
\caption{CP2102可配置参数}\label{CP2102可配置参数}
\end{table}

\begin{table}[!h]
\centering
\begin{tabular}{|c|c|}
\hline
{\hei{Name}} & {\hei{Value}} \\
\hline
{Vendor ID} & {10C4h} \\
\hline
{Product ID} & {EA60h} \\
\hline
{Power Descriptor(attributes)} & \tabincell{c}{80h}\\
\hline 
{Power Descriptor(Max Power)} & {32h} \\
\hline
{Release Number} & {0100h} \\
\hline
{Serial Number} & {0001(63 characters maximum)} \\
\hline
{Product Description String} & \tabincell{c}{"CP2102 USB to UART Bridge Controller”(126 characters maximum)"} \\
\hline
\end{tabular}
\caption{CP2102默认配置表}\label{CP2102DefaultConfigure}
\end{table}
	
	CP2102当中的协议控制单元会通过接受USB接口的命令,对UART接口进行配置(如配置通信的波特率、数据位、校验位、起始/停止位、流控信号等)。CP2102当中的接收和发送缓冲区用来临时保存双方在数据传输过程中的数据。以从计算机到外设的数据传输为例。当USB转串口设备连接到PC的USB总线上时,PC在检测到设备连接之后会对设备进行初始化并启动相关的客户端驱动程序;之后会由驱动程序给设备发送配置命令,设置设备的数据传输特性;最后,在数据传输的时候,计算机上的驱动程序会将数据包传输给USB接口(通常使用批量传输的方式),设备从USB接口提取出数据并保存在数据缓冲区中,UART接口再从数据缓冲区中将数据取走并发送出去,从外设传输数据到计算机的方式则相反。





\subsection{VxWorks上的USB开发}
	VxWorks 的I/O框架由ioLib.c 文件提供,但ioLib.c文件提供的函数仅仅是一个最上层的接口,并不能完成具体的用户请求,而是将请求进一步向其他内核模块进行传递,位于ioLib.c模块之下的模块就是iosLib.c。我们将ioLib.c 文件称为上层接口子系统,将iosLib.c文件称为I/O 子系统,注意二者的区别。上层接口子系统直接对用户层可见,而I/O 子系统则一般不可见(当然用户也可以直接调用iosLib.c 中定义的函数,但一般需要做更多的封装,且违背了内核提供的服务层次),其作为上层接口子系统与下层驱动系统的中间层而存在。VxWorks的内核驱动层次结构如\autoref{fig:VxWorks内核驱动层次结构}所示。

\begin{figure}[!h]
\centering
\includegraphics[width=0.9\textwidth]{./graphics/vxworks-kernel-diagram.pdf}
\caption{VxWorks驱动内核层次结构}\label{fig:VxWorks内核驱动层次结构}
\end{figure}	
			
	VxWorks的USB主机驱动程序堆栈满足USB协议规定的要求,提供了一整套服务来操作USB以及一些预置USB类驱动程序,以处理特定类型的USB设备。在Wind River的VxWorks中USB驱动程序堆栈的开发符合的是通用串行总线规范2.0版,USB系统是一种主从结构,系统的所有动作都是由USB主机发起,并协调不同的设备动作,设备端软件在系统中只需要对主机的命令做出响应即可,USB的主机端由于在系统中的地位比较特殊,因而其软件结构比较复杂,USB协议在主机端是分层实现的,其通信的逻辑结构和PC端的软硬件结构如\autoref{fig:usb通信结构}所示。USB协议由上至下可以分为三层:客户端驱动程序(Client Driver)、USB驱动(USBD)、主机控制器驱动(HCD),每一层完成不同的功能。

\begin{figure}[h]
\centering
  \begin{subfigure}[b]{0.4\textwidth}
  \includegraphics[width=1.0\textwidth]{./graphics/USB-device-structure-diagram.pdf}
  \caption{USB通信的逻辑结构}\label{fig:usb通信逻辑结构}
  \end{subfigure}
  ~
  \begin{subfigure}[b]{0.5\textwidth}
  \includegraphics[width=1.0\textwidth]{./graphics/USB-PC-structure.pdf}
  \caption{USB主机端软硬件结构}\label{fig:usb-PC}
  \end{subfigure}
\caption{USB通信结构}\label{fig:USB通信结构}
\end{figure}
	
	客户端驱动程序完成对不同的设备类的设备的功能驱动。本次设计中所要完成的USB口转串口的驱动要完成的就是客户端驱动。为了和设备进行正常的通信,他通过USB的I/O请求包(I/O request,IRP)向USBD层发出数据接收或者发送的请求。此外,USB的传输机制对于客户端的驱动程序而言是完全透明的,客户端驱动程序所看到的仅仅是具体的设备类,不管设备采用的何种的数据传输方式。另外IRP是USB协议定义的抽象概念,其结构需要根据协议的具体来实现。
	
	USBD是USB的核心驱动,其提供的功能包括USB总线的枚举、总线带宽的分配、传输控制等操作。向上的接口负责处理客户端驱动程序提出的I/O请求,他通过IRP了解此设备的属性和本次数据通信的具体要求,将此IRP转换成USB能够识别的一系列的事物处理,交给HCD层或直接交给主机控制器处理。USBD还负责新设备的动态插拔、USB电源管理和对客户端驱动程序的维护等操作。
	
	HCD层的主要功能是与主机控制器合作完成USB的各种事物处理。它根据一定的规则调度所有奖杯广播发送到USB上的事物处理。调度方法是首先将数据传输类型组成不同的链表,每一种链表包括来自不同的设备驱动程序的同一种类型的数据,然后定义不同的数据类型在传输中所占的带宽比例,交给主机控制器处理,控制器根据规则从立案表上摘下数据块,根据大小为它创建一个或者多个事物处理,完成与设备的数据传输,当事物处理完成时HCD将结果交给USBD层。此外它还完成对主机控制器和根集线器的配置和驱动等操作。WindRiver提供了两种类型的主机控制器驱动:usbHcdUhciLib(UHCI主机控制器驱动)和usbHcdOhciLib(OHCI主机控制器驱动)。VxWorks的USBD和HCD之间的接口允许操过一个的底层主控制器,并且USBD能够同时连接多个USB HCD。这样的设计特点可以让开发者建立复杂的USB系统。
	
	在VxWorks系统当中UBSD层的驱动和HCD层的驱动都是已经实现好的,我们所需要实现的是客户端的驱动程序,用以驱动特定的USB设备。典型的USB设备的描述符一般由USB标准描述符和USB类描述符组成,或者由USB标准描述符和USB厂商特定描述符组成。任何一个USB设备都必须包含USB标准描述符,他提供了设备的基本信息和通信方式。为了简化USB设备的开发过程,通常会将具有相同的或者是相识的功能的设备归为一类,并指定相关的类规范,这样就能够保证只要按照同样的规范标准,即使是不同的厂商开发的USB设备也能够使用相同的驱动程序。针对不同类型的USB设备,USB-IF规定了相关的类描述符,他在标准描述符的基础上进一步说明了特定类型的设备共能以及相关的数据传输方式。但是USB-IF规定的设备类描述符并不能够覆盖所有的电子设备,对于没有相关的类描述符的USB接口,生产厂商需要利用自己提供的厂商特定功能的类描述符和设备命令对其通信特性做出说明,这些特定功能的描述符和命令的定义和操作完全取决于厂商,要想驱动此类设备就必须要参考厂商提供的这些专有命令。CP2102模块就属于这种没有相关的类描述符的设备,他不但支持USB的标准描述符和USB标准命令,还支持自己特定的描述符和命令,我们称这样的设备为非标准类型的USB设备。非标准的设备命令和描述符的结构和处理方式与标准设备命令和描述符是一样的,但是它只对特定的功能设备有效。

	
	


\subsection{环形缓冲区的设计}
	在我们的USB口转串口驱动程序当中 ,我们需要解决低速设备和处理器之间、USB接口和RS-232串口之间的速度匹配问题 ,如果每次数据传送或接收时系统任务都去操作数据 ,这样必会造成任务频繁切换 ,系统效率降低。解决办法之一就是在设备驱动和应用程序之间建立环形数据缓冲 ,当数据到达时 ,系统会产生中断 ,USBD层会处理中断,然后调用我们注册的输入IRP来处理数据,将接受的数据放到我们设计的缓冲中 ,应用程序在需要接收数据时 ,从环形缓冲中的另一端(即最先放入数据的一端)读取数据,是一种先进先出缓冲结构。
		
	
  应用程序、设备驱动和环形缓冲区的关系如\autoref{fig:设备数据缓冲}所示,通过环形数据缓冲,我们可以解决速度匹配的问题,大大降低系统的开销。
	我们使用循环队列实现的来实现缓冲区,并设置对头指针和队尾指针,其特点是当一个数据元素被取走以后,其余的数据元素不需要移动其存储位置。在设备初始化时,将设备的环形缓冲区清空,队头指针和队尾指针均设为 0,之后通过操作队头指针与队尾指针来填充数据、移除数据。当驱动中接收到一个数据时,将此数据保存到当前位置 tail++,
	缓冲区当中需要考虑的关键问题是缓冲区的满和空的判断,因为缓冲区满或者是空都有可能出现读指针或者是写指针指向同一个位置。 我们处理这个问题的策略是在缓冲区中保持一个存储单元保存为未使用的状态。即一个大小为N的缓冲区中最多只能够存入N-1个数据。如果读写指针指向同一个位置那么缓冲区即为空。如果写指针位于读指针的相邻后一个位置,那么缓冲区为满。在测试缓冲区是否满时做取余运算。在缓冲区已满但还是有数据需要写入的时候,此时我们选择的策略是将最开始的数据进行覆盖,这样操作的好处是可以防止最新的数据丢失,而老的数据可能已经失去了时效性。

\begin{figure}[!h]
\centering
\includegraphics[width=.9\textwidth]{./graphics/Dev-Data-Buf.pdf}
\caption{设备数据缓冲}\label{fig:设备数据缓冲}
\end{figure}


\section{USB转串口驱动的实现}
	在VxWorks中驱动程序对上需要匹配操作系统提供的一套规范接口,对下必须驱动硬件设备进行工作,其起着一个关键的中间转换角色,将操作系统的具体请求转换为对硬件的某种操作,让所有的硬件对操作系统的一套内部规范接口进行响应,屏蔽了硬件的所有复杂性,应用层对于某个设备的操作通过操作系统提供的一套标准接口完成,操作系统最终将这些操作请求传递给驱动程序,驱动硬件完成这些请求。	
	VxWorks调试通道当中运行的最主要的软件平台是嵌入式实时操作系统VxWorks,作为系统的最底层的软件,要想进行数据的传输,驱动程序是必不可少的。本系统中的硬件设备是基于USB总线的,USB口转串口设备的驱动程序在Windows和Linux下都有现成可用的,但是在VxWorks下需要自己来实现这部分。
	
	在VxWorks I/ O 当中通常应该经过以下的三个基本步骤来实现一个设备驱动:
\begin{enumerate}
\item 实现对实际物理设备的数据结构抽象(即设备的自定义数据结构);
\item 完成 I/ O 系统所需要的各类接口及自身的特殊接口(open、read、write等);
\item 将驱动集成到操作系统中。
\end{enumerate}

\subsection{特定需求单设备驱动的实现}

	由于对于仅支持单设备驱动程序是基于特定的需求而具体定制的,所以该设备的驱动程序的实现流程与通常的支持多设备的驱动的初始化流程存在差异。具体的需求为:\\
\hei{1. 驱动中支持的设备名是固定的,无论具体的设备是否连接上,都可以往这个设备中写入数据。}\\
\hei{2. 驱动程序中要有缓存一定数据的能力,一旦设备连接之后就能够检查缓冲区中是否有数据,有数据则将其发送出去。}

	由于需要在设备未连接时就能够往设备中写入数据,且设备名为固定的,那么就必须调整驱动的初始化流程,使得其能够支持这一特性,通常驱动都是在设备加载之后再将其加入到系统设备表和系统驱动表当中,那么此时我们就需要先将一个固定的设备名加入到系统设备表当中。对于该特定需求的单设备驱动的流程图如\autoref{fig:SDev-Drv-diagram}和所示。
\begin{figure}[h]
\centering
  \begin{subfigure}[b]{1.0\textwidth}
  \includegraphics[width=\textwidth]{./graphics/SDev-Drv-Diagram-a.pdf}
  \caption{}\label{fig:SDevice-Driver-diagram-a}
  \end{subfigure}
  ~
  \begin{subfigure}[b]{1.0\textwidth}
  \includegraphics[width=\textwidth]{./graphics/SDev-Drv-Diagram-b.pdf}
  \caption{}\label{fig:SDevice-Driver-diagram-b}
  \end{subfigure}
\caption{特定需求单设备驱动运行流程图}\label{fig:SDev-Drv-diagram}
\end{figure}




\subsubsection{设备的自定义结构}
	底层驱动都要对其驱动的设备维护一个自定义的数据结构,这个结构用来保存所驱动的设备的关键参数,这些信息将随着设备类型的不同而有所差别。对于我们的USB口转串口设备需要保存的关键参数有:USB配置、读写缓冲区的指针、接口配置、端点地址等等。关键数据结构的定义如下: 
	
\lstset{language=C}
\begin{lstlisting}
typedef struct cp210x_dev
{
  DEV_HDR cp210xDevHdr; 
  
  UINT16 numOpen;
  USBD_NODE_ID nodeId;
  UINT16 configuration;	
  UINT16 interface; 
  UINT16 interfaceAltSetting;
  UINT16 vendorId;
  UINT16 productId;

  BOOL connected;  
  int trans_len;
  USBD_PIPE_HANDLE outPipeHandle; 
  USB_IRP	outIrp; 
  BOOL outIrpInUse; 
  UINT16  outEpAddr;
  UINT8 trans_buf[64];

  USBD_PIPE_HANDLE inPipeHandle;
  USB_IRP inIrp;
  BOOL inIrpInUse;
  UINT8 inBuf[64];
  UINT16 	inEpAddr;
} CP210X_DEV, *pCP210XDEV;
\end{lstlisting}
\noindent 部分成员的含义如下:

\begin{itemize}
\item DEV\_ HDR:这一成员必须是自定义设备结构的第一个成员,VxWorks 的I/O子系统会把所有的设备结构都看作是这个类型的结构,wind只会识别到DEV\_ HDR结构并对其进行管理,在系统的设备列表中,内核只使用DEV\_ HDR结构当中的成员。
\item numOpen:用来记录设备被打开的次数,每次调用open()函数打开该设备则numOpen加一,调用close()函数关闭设备则减一。
\item nodeId:用来保存该设备在系统中的唯一ID号。
\item configure、interface、interfaceAltsetting:用来保存设备的描述符中的配置、接口、可变接口信息。
\item vendorId、productId:保存该设备的厂商ID和产品ID,用来识别该设备是否适合我们的驱动程序。
\item outPipeHandle、inPipeHandle:设备的输入/输出端点的管道句柄,每次传输数据时都需要使用该句柄来表明数据传到的哪一个端点。
\end{itemize}






\subsubsection{驱动注册和设备创建} 
	
	底层的驱动程序都要提供一个函数给系统来调用,以便进行驱动程序的注册和初始化,这些工作的完成通常是在内核启动过程中进行的(即内核启动的时候会调用我们的驱动初始化函数)。对于此处的USB口转串口驱动我们定义cp210xDrvInit() 初始化函数,其主要完成驱动所需要的资源申请和系统的初始化,包括创建信号量、向系统注册驱动、创建设备、向USBD层注册。cp210xDevInit模块主要代码如下所示:
\lstset{language=C}
\begin{lstlisting}
STATUS cp210xDrvInit(void)
{
  ...
  if(OSS_MUTEX_CREATE(&cp210xWriteMutex) != OK || OSS_MUTEX_CREATE(&cp210xReadMutex) != OK || OSS_MUTEX_CREATE(&cp210xMutex) != OK || (blockReadSem = semBCreate(SEM_Q_FIFO, SEM_EMPTY)) == NULL )
  ... 	
  cp210xDrvNum = iosDrvInstall(NULL,NULL,cp210xDevOpen,cp210xDevClose,
			cp210xDevRead,cp210xDevWrite,cp210xDevIoctl);
  ...
  if( iosDevAdd(&pCp210xDev->cp210xDevHdr,CP210X_NAME,cp210xDrvNum) != OK)
  ...  
  if(usbdClientRegister (CP210X_CLIENT_NAME, &cp210xHandle) != OK)
  ...  
  if(usbdDynamicAttachRegister(cp210xHandle,USBD_NOTIFY_ALL,USBD_NOTIFY_ALL,USBD_NOTIFY_ALL,TRUE,(USBD_ATTACH_CALLBACK)cp210xAttachCallback)!= OK)
  ...
}
\end{lstlisting}\\

	驱动会在进入初始化的时候首先检查该驱动是否已经安装,若已经安装了则无需再次安装,直接退出即可,cp210xDrvInit()通常是在usrRoot(usrConfig.c)中调用,但是你也可以手动调用这个函数对该驱动进行初始化操作。然后会进行一些驱动所需要的全局资源的初始化,如信号量、全局变量、看门狗等,接着调用iosDrvInstall()函数安装驱动的I/O函数,将其添加到驱动表当中,在我们的USB口转串口驱动当中不需要实现delete函数和create函数,直接将其指针置为NULL即可。
	
	注册完成之后还要向系统将该驱动程序添加到IO子系统当中,添加成功后会在系统的设备列表中显示该命名为CP210X\_ NAME的设备(CP210X\_ NAME只是一个宏定义,设备名可以自己更改)。此处即是我们的驱动程序中的一个特殊的地方,在没有识别到设备之前就已经创建好设备文件,只不过此时该设备还只是一个“假”的,只有软件实现,没有硬件支撑,即使此时已经可以打开该设备,向该设备写入数据,但是也只是写入到了系统的缓冲区当中而已,数据并没有发送到任何的硬件上。

	接下来USB客户端驱动还需要向USBD层注册,注册完成之后会返回一个用于操作USBD的客户端handle,我们将其保存在cp210xHandle这个变量当中。然后还要注册一个动态注册的回调函数,当USBD层发现有USB设备的插拔动作时,就会根据我们注册时选定的设备类和接口类来判断是否需要调用我们注册的回调函数,由于我们的设备是一个特殊的设备,并不符合任何标准的USB设备类和接口类,于是我们将这两个参数置为USBD\_ NOTIFY\_ ALL,即任何USB设备的插拔都调用我们的注册的回调函数。之后我们在回调函数中根据设备的v设备ID和厂商ID来判断该设备是否能被我们的驱动所支持。
	在回调函数中我们会发送标准的USB设备请求命令来获取该设备的设备描述符,设备描述符当中包含有设备的类、子类、协议、厂商ID等信息,详细的USB设备描述符的信息如\ 所示;
\lstset{language=C}
\begin{lstlisting}
typedef struct usb_device_descr
    {
    UINT8 length;	    /* bLength */
    UINT8 descriptorType;	    /* bDescriptorType */
    UINT16 bcdUsb;		    /* bcdUSB - USB release in BCD */
    UINT8 deviceClass;		    /* bDeviceClass */
    UINT8 deviceSubClass;	    /* bDeviceSubClass */
    UINT8 deviceProtocol;	    /* bDeviceProtocol */
    UINT8 maxPacketSize0;	    /* bMaxPacketSize0 */
    UINT16 vendor;		    /* idVendor */
    UINT16 product;		    /* idProduct */
    UINT16 bcdDevice;		    /* bcdDevice - dev release in BCD */
    UINT8 manufacturerIndex;	    /* iManufacturer */
    UINT8 productIndex; 	    /* iProduct */
    UINT8 serialNumberIndex;	    /* iSerialNumber */
    UINT8 numConfigurations;	    /* bNumConfigurations */
    } WRS_PACK_ALIGN(4) USB_DEVICE_DESCR, *pUSB_DEVICE_DESCR;
...
    usbdDescriptorGet (cp210xHandle, nodeId,USB_RT_STANDARD | USB_RT_DEVICE,USB_DESCR_DEVICE,0, 0, sizeof(pBfr), pBfr, &actLen)) != OK)
\end{lstlisting}
    \end{lstlisting}
使用usbdDescriptorGet()函数来获取设备描述符,其中第二、第三个参数是USB的标准请求命令,表示我们请求的是设备的标准描述符,获取到设备标准描述之后提取其中的厂商ID和产品ID,我们将该驱动支持的设备的厂商ID和设备ID存储在一个二维数组当中,然后通过遍历当前插入的设备的VID和PID是否在我们的数组当中来判断这个设备是否能够被我们的驱动所支持。设备识别的流程如\autoref{fig:device-recognize}所示,目前支持的设备的设备ID、产品ID的组合如\autoref{tab:目前支持的设备列表}所示。
\begin{figure}[!h]
\centering
\includegraphics[width=.9\textwidth]{./graphics/device-recognize.pdf}
\caption{设备识别流程图}\label{fig:device-recognize}
\end{figure}

\begin{table}[!h]
\centering
\begin{tabular}{|c|c|c|c|c|c|c|}
\hline
{\hei{PID}}&{0x045B}&{0x0471}&{0x0489}&{0x0489}&{0x10C4}&{0x10C4}\\ 
\hline
{\hei{VID}}&{0x0053}&{0x066A}&{0xE000}&{0xE003}&{0x80F6}&{0x8115}\\
\hline 
{\hei{PID}}&{0x10C4}&{0x10C4}&{0x10C4}&{0x10C4}&{0x10C4}&{0x2405}\\
\hline
{\hei{VID}}&{0xEA60}&{0x813D}&{0x813F}&{0x814A}&{0x814B}&{0x0003}\\
\hline
\end{tabular}
\caption{目前支持的设备列表}\label{tab:目前支持的设备列表}
\end{table}


\subsubsection{驱动缓冲的实现}
	在驱动的内部我们会建立一个循环缓冲区来接收上层程序写入、从设备接收的数据,若系统中没有USB口转串口的设备,那么所有的上层应用往外输出的数据就只会写入到该循环缓冲区当中,当循环缓冲区中的数据已经满了之后,遵循先进先出的原则来覆盖数据。当USB口转串口设备连接上时,若循环缓冲区当中已经有数据存在,则会立即启动发送数据的过程,若没有数据,则什么也不做。
	
	在每一个设备初始化的时候我们都会为该设备分配读缓存和写缓存,缓存空间的大小作为以宏的方式定义在头文件当中,方便以后改动,分别定义为:WRITE_BUFFER_SIZE、READ_BUFFER_SIZE。
	由于计算机的内存是线性地址空间,因此环形缓冲区需要特别的设计才能够从逻辑上实现,通常环形缓冲区需要4个指针:
	 \begin{itemize}
	 \item 在内存中实际开始位置的指针;
	 \item 在内存中实际结束位置的指针,或者缓冲区的长度;
	 \item 存储在缓冲区中的有效数据的开始位置(读指针);
	 \item 存储在缓冲区中的有效数据的结束位置的指针(写指针)。
	 \end{itemize}


通过比较我们选择第三种范式方式作为我们底层驱动中缓冲区空/满的判断方式,同时我们会将缓冲区的大小设置成2的幂,这样我们就可以不需要镜像指示位和取余操作,只需要通过简单的条件表达式就可以判断缓冲区的空/满,这减少了我们的驱动程序本身的运算量和存储空间的消耗,节省了数据传输过程中的处理时间。


\subsubsection{设备的初始化}

	设备的初始化包括获取该USB设备的各种描述符信息。包括配置描述符信息、接口描述符信息、端点描述符信息。再通过所获得的这些信息来创建到输出端点的管道和对设备进行设置,我们在此处将设备的波特率初始化为115200,数据位为8位,1个停止位,没有奇偶校验,没有流控。设备的初始化流程图如\autoref{fig:device-init}所示。
\begin{figure}[!h]
\centering
\includegraphics[width=1.0\textwidth]{./graphics/Dev-Init.pdf}
\caption{设备初始化流程图}\label{fig:device-init}
\end{figure}

这些描述符同样是通过usbdDescriptorGet()函数来发送设备的标准描述符命令来获取,在获取到
这些信息之后,通过usbdConfigurationSet()、usbdInterfaceSet()来配置设备,通过usbdPipeCreate()函数来设置设备的输入、输出管道,通过管道来连接设备的输入、输出端点。部分关键代码如下:
\lstset{language=C}
\begin{lstlisting}
...  
  usbdDescriptorGet (cp210xHandle, pCp210xDev->nodeId,USB_RT_STANDARD | USB_RT_DEVICE, USB_DESCR_CONFIGURATION,0, 0, USB_MAX_DESCR_LEN, pBfr, &actLen) 
  pCfgDescr = usbDescrParse (pBfr, actLen,USB_DESCR_CONFIGURATION)
  pScratchBfr = pBfr;
  ifNo = 0;
  while ((pIfDescr = usbDescrParseSkip (&pScratchBfr,&actLen,USB_DESCR_INTERFACE))!= NULL){
    if (ifNo == pCp210xDev->interface)
		break;
	ifNo++;
  }
  pOutEp = findEndpoint(pScratchBfr, actLen, USB_ENDPOINT_OUT)
  pInEp = findEndpoint(pScratchBfr,actLen,USB_ENDPOINT_IN))
...  
  usbdConfigurationSet (cp210xHandle, pCp210xDev->nodeId,pCfgDescr->configurationValue,pCfgDescr->maxPower * USB_POWER_MA_PER_UNIT)
...  
  usbdInterfaceSet(cp210xHandle,pCp210xDev->nodeId,pCp210xDev->interface,pIfDescr->alternateSetting);
...  
  usbdPipeCreate(cp210xHandle,pCp210xDev->nodeId,pOutEp->endpointAddress,pCfgDescr->configurationValue,pCp210xDev->interface,USB_XFRTYPE_BULK,USB_DIR_OUT,maxPacketSizeOut,0,0,&pCp210xDev->outPipeHandle)
...  
  usbdPipeCreate(cp210xHandle,pCp210xDev->nodeId,pInEp->endpointAddress,pCfgDescr->configurationValue,pCp210xDev->interface,USB_XFRTYPE_BULK,USB_DIR_IN,maxPacketSizeIn,0,0,&pCp210xDev->inPipeHandle)
...
\end{lstlisting}


\subsubsection{设备打开/关闭函数}

	用户在使用一个设备之前必须先对这个设备进行打开,在这个过程当中底层驱动的响应函数通常会将进行中断注册和使能设备工作配置等操作。但是对于我们的USB口转串口驱动而言,我们不需要自己管理中断,USBD层会为我们进行中断的管理工作,而配置工作我们在设备的初始化工作中已经完成,因此此处我们的设备打开工作,只需要简单的记录设备被打开的次数,并返回即可。设备的打开代码如下所示:
	
\lstset{language=C}
\begin{lstlisting}
LOCAL CP210X_DEV * cp210xDevOpen(DEV_HDR *pDevHdr, char *name, int flags,int mode)
{
	CP210X_DEV *pCp210xDev;
	pCp210xDev = (CP210X_DEV *)pDevHdr;
	if(!pCp210xDev->connected)	
		return (ERROR);
		
	(pCp210xDev->numOpen)++;
	return (pCp210xDev);
}
\end{lstlisting}

	第一个DEV\_ HDR类型的参数即是我们之前所说的必须是设备自定义结构的第一个成员,在此时它会由 IO 子系统自动提供,IO 子系统会根据驱动号寻址到对应驱动函数时,然后将对应的系统设备列表中存储的设备结构作为第一个参数来调用 cp210xDevOpen()。我们之后在使用的时候需要首先将这个 DEV\_ HDR 结构转换成我们的自定义结构 CP210X\_ DEV。但是我们也可以直接将第一个参数的类型设置为自定义结构类型,那么对于我们USB口转串口驱动,以上  cp210xDevOpen() 函数的调用原型就变为:LOCAL int cp210xDevOpen(CP210X\_ DEV *pCp210xDev, char *name, int flags,int mode)这并不会造成什么影响。
		
	第二个参数是设备名匹配后的剩余部分,注意不是我们传过来的设备名。VxWorks中使用最佳匹配的原则来匹配设备名,如我们打开一个设备"/ttyUsb/xyz",但是系统中并没有这个设备,只有设备"/ttyUsb/x",那么此处的name就是匹配之后剩余的字符串"yz"。我们的应用中 open() 函数调用时的路径名与系统设备列表中的设备名是完全匹配的,此处的 name 就会是一个空字符串。对于在文件系统层下的块设备来说,此处指向的应该是块设备节点名后的子目录和文件名。

	第三,四个参数就是用户 open() 调用时传入的第二,三个参数,IO 子系统不会对他们进行更改,只是原封不动的转发给了 cp210xDevOpen() 函数。
	
	该函数的返回值有如下两种值:一个有效的CP210X\_ DEV结构指针表示 cp210xDevOpen 调用成功,ERROR 则表示 cp210xDevOpen()调用失败,IO 子系统会根据返回的指针是否有效来决定返回一个文件描述符还是返回一个错误。cp210xDevOpen() 函数的返回值非常重要,这个指针将被 IO 子系统保存,用于其后对驱动中读写,控制函数的调用。这个返回的指针也会作为这些函数的第一个参数。
	
	设备的关闭和设备的打开操作是相反的操作,在关闭操作当中我们只需要对设备记录的打开次数进行减法操作即可。

\subsubsection{设备的读写}
	在成功打开一个设备后,用户程序将得到一个open()返回的文件描述符,此后用户就可以使用这个文件描述符对设备进行读写,控制操作。我们的USB口转串口驱动的读写函数原型如下:

\lstset{language=C}
\begin{lstlisting}
LOCAL int cp210xDevWrite(CP210X_DEV *pCp210xDev, char *buffer,UINT32 nBytes)
LOCAL int cp210xDevRead(CP210X_DEV *pCp210xDev, char *buffer,UINT32 nBytes)
\end{lstlisting}

此处我们将这两个函数的第一个参数类型直接设置为 CP210X\_ DEV 结构类型而不是DEV\_ HDR,只是为了向大家展示两种方式都是允许的,在我们的实际编程中应该一种方式一以贯之。

	由于我们的驱动程序的初始化方式比较特殊,所以此处对数据的输出操作的流程也需要有相应的改变。在驱动程序中我们设置了一个4K的数据缓冲区来接收数据,设备的初始化之后就先查询缓冲区中是否存在数据,若存在数据则将其发送,若不存在则等待其他程序调用write写入数据来触发数据的发送操作,系统调用write()在该驱动程序中对应的函数为cp210xDevWrite()函数,这个函数会接受write()发送过来的数据,并将其存入缓冲区中,并判断是否需要触发数据的发送操作。发送数据的触发操作只会在当前没有数据在发送时完成,若调用write时设备已经在发送数据,则只是将write的数据存入缓冲区,设备发送完当前正在发送的数据之后会去判断缓冲区中是否还有数据,有数据则会继续发送。其流程图如\autoref{fig:outData-diagram} 所示。

\begin{figure}[!h]
\centering
\includegraphics[width=.7\textwidth]{./graphics/data-send-diagram.pdf}
\caption{数据发送逻辑图}\label{fig:outData-diagram}
\end{figure}


cp210xDevRead 和 cp210xDevWrite 的实现非常相似,只是更换了一下数据的传输方向。cp210xDevRead在底层也有一个循环缓冲区用来接收从USB口转串口设备发送来的数据,当需要读取 nbytes 个字节而缓冲区内的字节不够时,read就会阻塞,直到USBD层通知你有新的数据到来时才会继续进行读操作。同时也在驱动中启动了一个计时器,如果在计时器时间到了之后,还未能满足需要读取的字节数,则退出本次读写操作,返回当前已处理的字节数。


\subsubsection{设备的控制操作}
	设备控制操作用于对设备的某一些工作行为进行再配置,可进行的在配置类别随着设备类型的不同而不同,操作系统当中通常会一种类型的设备的某一组共同属性作为一个配置选项,比如波特率再配置就是串口的一个标准属性,而一般的USB设备是不具有该属性的。但是这只是一个约定,并不是所有的设备都必须要完全对照这一准则,底层驱动也可以根据自己的实际需要来对这些再配置属性进行选择,我们可以选择只实现某一些再配置参数,也可以根据具体情况对某一个再配置选项进行响应,设备控制函数给用户控制设备提供方便的同时也对底层设备的实现提供了极大的方便性。
	
	对于我们的USB口转串口驱动而言,其属于一个特殊的设备,没法归入操作系统的已经分好类的设备当中,我们需要实现一些非约定的配置属性,如配置波特率、流控、数据位等等非USB所属的配置选项。
	我们将USB口转串口驱动特定的参数定义在一个头文件当中,而后将这个头文件提供给用户程序,当用户对设备进行操作时,其包含这个头文件,使用其中定义的特定参数对设备进行控制。IO 子系统不会对用户调用的ioctl()函数做任何的改变,只会将用户使用的选项参数或者控制命令传递给我们的cp210xDevIoctl()函数,然后由这个函数完成对选项参数或控制命令的解释和使用。
设备控制函数原型如下:
\lstset{language=C}
\begin{lstlisting}
LOCAL int cp210xDevIoctl(CP210X_DEV *pCp210xDev, int request, void *someArg )
\end{lstlisting}

对于我们的 USB口转串口驱动,在实际使用中,再配置参数和命令有很多,但是目前我们只提供设备的波特率、数据位、校验位、流控的参数和命令。这些控制对于普通的USB设备而言是没有的,他们在定义上属于USB的厂商自定义请求,在我们的驱动程序的cp210xDevIoctl()函数当中最后都要使用usbdVendorSpecific()函数来发送USB的厂商自定义请求,该函数的原型如下:
\lstset{language=C}
\begin{lstlisting}
STATUS usbdVendorSpecific
(
  USBD_CLIENT_HANDLE clientHandle,/* Client handle */
  USBD_NODE_ID nodeId,	/* Node Id of device/hub */
  UINT8 requestType,	/* bmRequestType in USB spec. */
  UINT8 request,		/* bRequest in USB spec. */
  UINT16 value,			/* wValue in USB spec. */
  UINT16 index,			/* wIndex in USB spec. */
  UINT16 length,		/* wLength in USB spec. */
  pUINT8 pBfr,			/* ptr to data buffer */
  pUINT16 pActLen		/* actual length of IN */
)
\end{lstlisting}
其中第二个参数是请求类型,表示该类型是从主机到设备的,还是从设备到主机的,有四种类型,分别是REQTYPE\_ HOST\_ TO\_ INTERFACE(0x41),REQTYPE\_ INTERFACE\_ TO\_ HOST(0xC1),REQTYPE\_ HOST\_ TO\_ DEVICE(0x40),REQTYPE\_ DEVICE\_ TO\_ HOST((0xC0)),第三个参数是具体的请求,对于我们的设备而言能响应的部分主要请求如\autoref{tab:厂商指定请求}所示。

\begin{table}[!h]
\centering
\begin{tabular}{|c|c|c|c|}
\hline
{\hei{宏名}}&{\hei{代码}}&{\hei{宏名}}&{\hei{代码}}\\
\hline
{CP210X\_ IFC\_ ENABLE}&{0x00}&{CP210X\_ SET\_ BAUDDIV}&{0x01}\\
\hline
{CP210X\_ GET\_ BAUDDIV}&{0x02}&{CP210X\_ SET\_ LINE\_ CTL}&{0x03}\\
\hline
{CP210X\_ GET\_ LINE\_ CTL}&{0x04}&{CP210X\_ SET\_ BREAK}&{0x05}\\
\hline
{CP210X\_ IMM\_ CHAR}&{0x06}&{CP210X\_ SET\_ MHS}&{0x07}\\
\hline
{CP210X\_ SET\_ XOFF}&{0x0A}&{CP210X\_ SET\_ XON}&{0x09}\\
\hline
{CP210X\_ SET\_ FLOW}& 0x13 & CP210X\_ GET\_ FLOW & 0x14\\
\hline
CP210X\_ EMBED\_ EVENTS	& 0x15 & CP210X\_ GET\_ EVENTSTATE & 0x16\\
\hline
CP210X\_ SET\_ CHARS & 0x19 & CP210X\_ GET\_ BAUDRATE & 0x1D\\
\hline
CP210X\_ SET\_ BAUDRATE & 0x1E & CP210X\_ VENDOR\_ SPECIFIC & 0xFF\\
\hline
\end{tabular}
\caption{厂商指定请求}\label{tab:厂商指定请求}
\end{table}

\subsubsection{驱动卸载}
	在驱动的初始化函数cp210xDrvInit()当中我们使用了 iosDrvInstall() 函数向 IO 子系统注册我们的驱动,在wind中也提供了另外一个相反作用的函数iosDrvRemove()注销我们的驱动。该函数调用原型如下 :
\lstset{language=C}
\begin{lstlisting}
STATUS iosDrvRemove 
 ( 
 int drvnum, /* driver to remove, returned by iosDrvInstall()*/ 
 BOOL forceClose /* if TRUE, force closure of open files */ 
); 
\end{lstlisting}

该函数的第二个参数指定是否强制进行卸载,并将所有与此驱动有关的文件描述符关闭。如果强制关闭,则 IO 子系统将遍历系统文件描述符表,检查每个描述符对应结构中的驱动号是否等于要卸载驱动的驱动号,如果相同,则调用这个驱动的 close 实现函数进行关闭,同时释放文件描述符表中该表项,此时用户层的文件句柄将自动失去功效,如果用户其后使用这个文件描述符,将直接得到一个错误返回。

除了驱动卸载函数之外,我们的驱动初始化时还向USBD层进行了注册,在卸载的时候也应该注销USBD层的注册,同时注销动态注册的回调函数并从系统的设备表当中删除掉该设备。之后还应该对该驱动占用的所有其他的系统资源进行释放。

至此,我们已经完成了该特定需求下的USB口转串口驱动程序的所有组成部分的设计和实现。

\subsection{通用多设备驱动的实现}
多设备支持的驱动程序在初始化的时候与特定需求单设备支持的驱动初始化的过程是不一样的,需要支持多设备,首先需要在识别设备后给每一个设备分配一个设备名,并加入到系统设备表当中,然后需要给每一个设备初始化自己的设备结构体多设备下的驱动运行流程如\autoref{fig:MDev-Drv-diagram}所示。

\begin{figure}[p, !h]
\centering
  \begin{subfigure}[b]{1.0\textwidth}
  \includegraphics[width=\textwidth]{./graphics/MDev-Drv-Diagram-a.pdf}
  \caption{}\label{fig:MDevice-Driver-diagram-a}
  \end{subfigure}
  ~
  \begin{subfigure}[b]{1.0\textwidth}
  \includegraphics[width=\textwidth]{./graphics/MDev-Drv-Diagram-b.pdf}
  \caption{}\label{fig:MDevice-Driver-diagram-b}
  \end{subfigure}
\caption{多设备驱动运行流程图}\label{fig:MDev-Drv-diagram}
\end{figure}


与特定需求单设备的驱动相比,这里驱动程序的变化主要在驱动的初始化和读写函数部分,同时设备自定义的数据结构会更加复杂,需要保存更多与具体的设备相关的信息。

\subsubsection{设备的自定义结构体}
多设备的驱动自定义结构体的定义如下所示:
\lstset{language=C}
\begin{lstlisting}
typedef struct cp210x_dev
{
  DEV_HDR cp210xDevHdr; /*must be first field*/
  LINK 	devHdrLink; /*linked list of  devhdr structs*/
  UINT16 numOpen;
  USBD_NODE_ID nodeId; /*device nodeID*/
  UINT16 configuration; 
  UINT16 interface; /*a interface of this device*/
  UINT16 interfaceAltSetting;
  UINT16 vendorId;
  UINT16 productId;
  BOOL connected;
  
  int trans_len;
  USBD_PIPE_HANDLE outPipeHandle; /* USBD pipe handle for bulk OUT pipe*/
  USB_IRP	outIrp; /*IRP to monitor output to device*/
  BOOL outIrpInUse;
  UINT32 outErrors; /*TRUE while IRP is outstanding*/
  UINT8 trans_buf[64];
  UINT16  outEpAddr;  
  
  USBD_PIPE_HANDLE inPipeHandle;
  USB_IRP inIrp;
  BOOL inIrpInUse;
  UINT8 inBuf[64];
  UINT32 inErrors;
  UINT16 inEpAddr;
  
  char *writeBuf;
  int writeFront;
  int writeRear;
  char *readBuf;
  int readFront;
  int readRear;
} CP210X_DEV, *pCP210XDEV;
\end{lstlisting}
	在这个结构体当中我们增加了每个设备的读写缓冲区的指针和各个缓冲区的头尾指针。同时使用了一个链表devHdrLink来链接在系统上的由该驱动支持的设备,每次检测到新设备时我们可以通过将新添加的设备增加到这个链表当中,之后可以通过nodeId来从多个设备中定位我们的设备是否存在,之后我们可以给每一个设备分配一个设备名。部分代码如下所示:
\lstset{language=C}
\begin{lstlisting}
...
 usbListLinkProt(&devListHdr,(pVOID)pCp210xDev,(pLINK)&pCp210xDev->devHdrLink,LINK_TAIL,cp210xMutex);
...

LOCAL pCP210XDEV findDevHdr(USBD_NODE_ID nodeId)
{
	pCP210XDEV pCp210xDev =  usbListFirst(&devListHdr);

	while(pCp210xDev != NULL)
	{
		if(pCp210xDev->nodeId == nodeId)
			break;
		pCp210xDev = usbListNext(&pCp210xDev->devHdrLink);
	}
	return pCp210xDev;
}
\end{lstlisting}

\subsubsection{驱动注册和设备创建}

	比较单设备驱动初始化(如\autoref{fig:SDevice-Driver-diagram-a}所示)和多设备驱动初始化(如\autoref{fig:MDevice-Driver-diagram-a}所示),我们可以看出在驱动注册过程中两者的区别,在单设备驱动的初始化中我们先完成设备结构体的创建并定好一个设备名,之后直接将其加入到系统的设备表当中,即使此时没有设备连接。而在多设备的驱动初始化当中我们是在驱动的回调函数中驱动识别完了设备之后再完成设备结构体的创建和加入系统设备表的,这种方式是通用的设备驱动常采用的方式。
	在设备创建时我们会通过判断已连接设备的个数来决定当前设备所采用的设备名,部分代码如下:

\lstset{language=C}
\begin{lstlisting}
LOCAL int getCp210xDeviceNum(CP210X_DEV *pCp210xDev)
{
...
  for (int index=0; index < CP210X_MAX_DEVICE; index++)
    if (pCp210xDevArray[index] == NULL){
      pCp210xDevArray[index] = pCp210xDev;
      return (index);
    }
...
}
	
LOCAL STATUS cp210xAttachCallback(USBD_NODE_ID nodeId, UINT16 attachAction,UINT16 configuration,UINT16 interface,UINT16 deviceClass,UINT16 deviceSubClass, UINT16 deviceProtocol)
{
  ...
  cp210xUnitNum = getCp210xDeviceNum(pCp210xDev);
  sprintf (cp210xName, "%s%d", CP210X_NAME,cp210xUnitNum);
  if(iosDevAdd(&pCp210xDev->cp210xDevHdr,cp210xName,cp210xDrvNum) != OK)
  ...
}
\end{lstlisting}

\subsubsection{设备读写}
	对于通用多设备的写操作,与设定设备的操作不同的是设备连接上时,没有一个自动发送缓冲区的数据的过程,此时设备没有连接上也不可能往缓冲区中写入数据。其基本流程如下:	
	
	\begin{enumerate}
	\item 将数据拷贝到输出循环缓冲区当中,若缓冲区已满则等待。
	\item 判断设备是否仍然处于连接状态。
	\item 若处于连接状态,那么是否有数据正在发送当中。若有数据正在发送,则等待。
	\item 若没有数据在发送,则触发发送数据的操作。
	\item 返回发送的字节数。
	\end{enumerate}
	
	对于数据发送的逻辑控制,使用了VxWoks的同步和互斥信号量。首先在写入数据的时候需要进行互斥写,因为此时设备有可能正在从缓冲区当中取数据进行输出操作,那么这是写入输出缓冲区就需要等待,否则可能会造成缓冲区的混乱,造成输出结果与输入数据不一致。当设备输出从缓冲区拷贝完成之后就会释放互斥信号量,此时写入操作就可以往输出缓冲区中写入数据。
	
	
	对于通用多设备的读操作,我们会事先创建好一个用来读取数据的USB IRP在这个IRP中我们注册一个回调函数,USB的中断会由USBD层来替我们进行管理,当有数据到来时USBD层会调用我们注册的回调函数来通知我们。在驱动程序初始化完成之后我们就会启动listenForInput()这个函数来注册一个USBD的通知过程,一个接收IRP使用完成之后,我们需要重新注册一次,因为每一次的IRP都是单次有效的,所以在cp210xIrpCallback()中我们接受完这一次的IRP的数据之后, 需要新建另一个IRP重新启动下一次的listenForInput()过程。部分关键代码如下:
\lstset{language=C}
\begin{lstlisting}

LOCAL STATUS listenForInput(CP210X_DEV *pCp210xDev)
{
..
  pIrp->userPtr = pCp210xDev;
  pIrp->userCallback = cp210xIrpCallback;
  pIrp->timeout = USB_TIMEOUT_NONE;
  pIrp->transferLen = 64;
  if(usbdTransfer (cp210xHandle, pCp210xDev->inPipeHandle, pIrp) != OK)
...
}

LOCAL void cp210xIrpCallback(pVOID p)
{
...
	if(pIrp == &pCp210xDev->inIrp && pCp210xDev->connected == TRUE)
	{
		if(pIrp->result == OK)
			copy_to_readBuf(pCp210xDev);
			
		if (pIrp->result != S_usbHcdLib_IRP_CANCELED)
			listenForInput(pCp210xDev);
	}
}

\end{lstlisting}




其他的部分如设备的控制、设备打开/关闭、设备卸载函数与单设备下的相比不需要做改变即可完成,此时操作的设备就是我们使用设备名打开的那个设备,IO子系统会将设备名映射到该设备所对应地驱动。









\section{小结}
	本章首先介绍了我们的USB口转串口驱动的设计想法,包括USB口转串口所使用转换器的选择,驱动程序所需要实现的模块,每个模块的功能是什么。接下来对所选择的转换器CP2102的开发进行了介绍,对VxWorks中USB的开发进行了介绍,最后根据我们的设计方案对驱动程序的每一个部分进行了实现。






%\subsection{第二层}\label{sec:1}
%\subsubsection{第三层}\label{sec:1}
%测试测试测试测试测试测试测试测试测试测试测试测试。
%\footnote{\label{footnote:1}脚注}

\section{字体}

普通\textbf{粗体}\emph{斜体}

\hei{黑体}\kai{楷体}\fangsong{仿宋}

\section{公式}

单个公式,公式引用:\autoref{eq:1}。
\begin{equation}
 c^2 = a^2 + b^2 \label{eq:1}
\end{equation}

多个公式,公式引用:\autoref{eq:2},\autoref{eq:3}。

\begin{subequations}
\begin{equation}
  F = ma \label{eq:2}
\end{equation}
\begin{equation}
  E = mc^2 \label{eq:3}
\end{equation}
\end{subequations}

\section{罗列环境}

\begin{enumerate}
    \item 第一层\label{item:1}
    \item 第一层
    \begin{enumerate}
        \item 第二层\label{item:2}
        \item 第二层
        \begin{enumerate}
            \item 第三层\label{item:3}
            \item 第三层
        \end{enumerate}
    \end{enumerate}
\end{enumerate}

\begin{description}
    \item[解释环境]  解释内容
\end{description}



\clearpage

\section{代码环境}

\begin{lstlisting}[language=python]
import os

def main():
    '''
    doc here
    '''
    print 'hello, world' # Abc
    print 'hello, 中文' # 中文
\end{lstlisting}

\section{定律证明环境}

\begin{definition}\label{def:1}
这是一个定义。
\end{definition}
\begin{proposition}\label{proposition:1}
这是一个命题。
\end{proposition}
\begin{axiom}\label{axiom:1}
这是一个公理。
\end{axiom}
\begin{lemma}\label{lemma:1}
这是一个引理。
\end{lemma}
\begin{theorem}\label{theorem:1}
这是一个定理。
\end{theorem}
\begin{proof}\label{proof:1}
这是一个证明。
\end{proof}

\section{算法环境}

\begin{algorithm}[H]
\SetAlgoLined
\KwData{this text}
\KwResult{how to write algorithm with \LaTeX2e }
initialization\;\label{alg_line:1}
\While{not at end of this document}{
read current\;
\eIf{understand}{
go to next section\;
current section becomes this
 one\;
}{
go back to the beginning of current section\;
}
}
\caption{How to write algorithms}\label{alg:1}
\end{algorithm}

\section{表格}
表格见\autoref{tab:1}。

\begin{table}[!h]
\centering
\caption{一个表格}\label{tab:1}
\begin{tabular}{|c|c|}
\hline
a & b \\
\hline
c & d \\
\hline
\end{tabular}
\end{table}
\section{图片}
图片见\autoref{fig:1}。图片格式支持eps,png,pdf等。多个图片见\autoref{fig:2},分开引用:\autoref{fig:2-1},\autoref{fig:2-2}。

\begin{figure}[!h]
\centering
\includegraphics[width=.4\textwidth]{hust-title.pdf}
\caption{hust-title}\label{fig:hust-title}
\end{figure}

\begin{figure}[!h]
\centering
  \begin{subfigure}[b]{0.3\textwidth}
  \includegraphics[width=\textwidth]{./data/VxWorks-driver-structure.pdf}
  \caption{VxWorks-driver-structure}\label{fig:2-1}
  \end{subfigure}
  ~
  \begin{subfigure}[b]{0.3\textwidth}
  \includegraphics[width=\textwidth]{fig-example.pdf}
  \caption{图片2}\label{fig:2-2}
  \end{subfigure}
\caption{多个图片}\label{fig:2}
\end{figure}

\section{参考文献示例}
这是一篇中文参考文献\cite{徐媛媛2003嵌入式实时操作系统的设备驱动};这是一篇英文参考文献\cite{9787508342894};同时引用\cite{9780124467422,bamboosilk}。

\section[\textbackslash{}autoref 测试]{\texttt{\textbackslash{}autoref} 测试}

\begin{description}
  \item[公式] \autoref{eq:1}
  \item[脚注] \autoref{footnote:1}
  \item[项] \autoref{item:1},\autoref{item:2},\autoref{item:3}
  \item[图] \autoref{fig:1}
  \item[表] \autoref{tab:1}
  \item[附录] \autoref{appendix:1}
  \item[章] \autoref{chapter:1}
  \item[小节] \autoref{sec:1},\autoref{sec:2},\autoref{sec:3}
  \item[算法] \autoref{alg:1},\autoref{alg_line:1}
  \item[证明环境] \autoref{def:1},\autoref{proposition:1},\autoref{axiom:1},\autoref{lemma:1},\autoref{theorem:1},\autoref{proof:1}
\end{description}

% backmatter用于表示论文的正文结束
\backmatter

%ack 环境用于致谢页面
\begin{ack}
致谢正文。
\end{ack}

% bibliography用于生成参考文献。
\bibliography{ref/myref.bib}

% appendix环境用于附录环境。即可以将附录置于appendix环境当中。如:
% \begin{appendix}
%  <content>
% \end{appendix}

% 直接使用\appendix 则表明后文均为附录。如:
% \appendix
%  <content> 
\appendix

% publications环境用于已经发表了的论文的页面,一般用于附录当中,使用上同enumerate环境
\begin{publications}
    \item 论文1
    \item 论文2
\end{publications}

\chapter{这是一个附录}\label{appendix:1}
附录正文。


\end{document}

\endinput
%%
%% End of file `hustthesis-zh-example.tex'.
