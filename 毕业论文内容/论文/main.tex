
% !TeX program = xelatex

%使用hustthesis这个模板。
% draft版本的正文页包括页眉(“华中科技大学xx学位论文”)、页眉修饰线(双线)、页脚(页码)和页脚修饰线(单线)。
% final版本的正文页不包括页眉、页眉修饰线和页脚修饰线,仅包 含页脚(页码)。如果不指定,默认设置为final。

% degree用来指定论文的种类
% language用来指定论文语言。特别的,如果设定为english-draft,将会剔除论文中的所有中文内容,这有利于在未安装中文字体的环境中使用。如果不指定,默认设置为 chinese。


\documentclass[format=draft,language=chinese,degree=master]{hustthesis}
%\documentclass[format=final,language=chinese,degree=master]{hustthesis}

%使用下面的这两个宏包来生成带书签和超链接的PDF文件。
%\usepackage[bookmarks,bookmarksopen,bookmarksdepth=2]{hyperref}
%\usepackage[pdftex,CJKbookmarks=true,colorlinks=true]{hyperref} %LaTeX Error: Option clash for package hyperref

\stuno{M201672905}
\schoolcode{10487}
\title{基于Vxworks的调试通道的设计与实现}{A design and implementation of debug channel based on Vxworks.}
\author{郑松}{Joe}
\major{计算机应用技术}{Computer Applications Technology}
\supervisor{张杰\hspace{1em}讲师}{Instructor JieZhang}
\date{2018}{3}{26}
%\date{\today}

\zhabstract{
    数据传输是现代通讯过程中的一个重要环节,在数据的传输过程中,不仅仅要求数据传输的准确率要高,而且要求速度快,连接方便。传统的RS232串口通讯和并口通讯都存在传输速度低,扩展性性差、安装麻烦等缺点,而基于USB接口的数据传输系统能够较好的解决这些问题。目前USB接口以其传输速率高、即插即用、支持热插拔等优点,逐步成为PC机的标准接口。

	本文中的数据传输系统采用了USB接口进行上位机与下位机之间的数据通讯。下位机采用的是VxWorks实时操作系统。

}
\zhkeywords{实时操作系统,设备驱动,USB口转串口,VxWorks,CP2103}

\enabstract
{
    This is a \LaTeX{} template example file. This template is used in written thesis for Huazhong Univ. of Sci. \& Tech.

    This template is published under LPPL v1.3 License.

}
\enkeywords
{\LaTeX{}, Huazhong Univ. of Sci. \& Tech., Thesis, Template}

\begin{document}

% frontmatter用于设定论文的状态、改变样式,其具体使用见简单示例。
% frontmatter用在文档最开始,表明文档的前言部分(如封面,摘要,目录等)的开 始。
\frontmatter

% maketitle的作用和makecover的作用相同,用于生成封面和版权页面
\maketitle

% makeabtract用于生成中英文的摘要页面。
\makeabstract

% tableofcontents用于生成目录
\tableofcontents

% listoffigures 和 listoftables 分别用于生成图片和表格索引,可以根据要求在论文的前言中使用或者是在附录中使用
%\listoffigures
%\listoftables

% mainmatter表示论文正文的开始。
\mainmatter
\clearpage

\chapter{绪论}
\section{课题背景以及意义}
	当今,在嵌入式领域,嵌入式技术(Embedded Technology)已经成为了新的技术热点,随着嵌入式系统的不断发展和应用,针对不同的嵌入式软件的开发也越来越受到重视。在嵌入式软件的设计当中,由于嵌入式独有的特点,其调试、分析一直是一个费时费力的工作。一个好的调试器可以给嵌入式软件开发人员带来很大的帮助,使其达到事半功倍的效果,快速完成软件开发过程中的调试分析过程、软件运行过程中日志信息的定位等工作。
		
	目前国内外已有几十种商业化嵌入式操作系统可以供选择,如VxWorks、uc/OS-II、Windows Embedded CE、RTLinux、和“女蜗Hopen”等。其中VxWorks以其良好的可靠性和卓越的实时性被广泛地应用在通信、军事、航空、航天等高精尖技术及实时性要求极高的领域中,如卫星通讯、军事演习、弹道制导、飞机导航等。而Linux操作系统则是完全开源的,在全世界拥有几十万的开源项目,目前主流的Android及嵌入式设备都采用Linux操作系统。
	
	在我国VxWorks大量的应用于我国的军事、国防工业当中,通常在进行VxWorks应用程序的开发或者是将Linux下的应用程序移植到VxWorks中时都需要在程序中加入大量的调试信息,在程序的运行当中也需要输出一些日志信息,方便之后编程人员对程序运行过程中产生的问题进行具体的分析。VxWorks自身带有一个集成的测试开发调试环境Tronado,可以用它来完成程序的编辑、编译、调试、系统配置等工作。其带有的CrossWind调试器拥有一个驻留在主机端的命令行解释器WindSh和GDB命令行,但是由于中船重工实际使用中的限制,设备上并没有串口可供使用,而且设备并不具备进行现场调试的环境,他们希望能够在软件运行时直接将调试和日志信息输出保存之后进行一个事后分析的工作。
	
	因此,本论文基于中船重工的实际需求制作了一个基于VxWorks的调试通道。
	
			
\section{国内外概况}
	随着计算机技术的突飞猛进,实时系统无论是在技术上还是在应用领域当中都取得了辉煌的成就,尤其是在最近的二三十年当中,随着物联网的兴起,各种智能设备都需要安装嵌入式操作系统,导致嵌入式操作系统的发展愈加迅猛。而嵌入式实时操作系统属于嵌入式操作系统中定位更加精准的一类,其应用场景更加的专业化。

实时嵌入式系统广泛应用在通信、航天、航空等
关键型任务控制领域内。实时嵌入式系统是指以计
算机技术为基础 ,以运用为目的的专用系统。系统
对硬件和软件都有严格要求 ,硬件上对外观尺寸、内
部可靠性及指令系统都有很高要求 ,软件在可靠性、
实时性等方面也具有严格要求。美国风河公司的
VxWorks 操作系统 ,因具有抢占式调度、中断延迟
小、系统内核可剪裁等特点 ,在嵌入式应用领域内占
据重要地位。在嵌入式系统应用中 ,出于成本、尺
寸、功能等方面考虑 ,广泛应用定制硬件 ,这样就要
求用户自己开发硬件的驱动程序。驱动程序开发是
系统开发中的重要部分 ,驱动程序的性能、可靠性制
约着应用系统的性能和可靠性。设备驱动本身与操作系统的相关性特别密切 ,因此驱动程序开发不仅
要求开发者对操作系统有深入的了解 ,还要对硬件
体系具有相当的了解 ,所以难度较大。
	
	目前国内在使用的RTOS有上百个,这些操作系统面向不同的专业领域,具有各自不同的特性。以下介绍几个具有代表性的操作系统:
\begin{itemize}
\item \textbf{uc/OS-II}
	
	是一个由Micrium公司提供的可移植、可固化、可裁剪、抢占式多任务实时内核,在这个内核之上提供了最基本的系统服务,该内核适用于多种微处理器、微控制器和数字处理芯片。该实时操作系统内核的特点是仅仅包含了任务调度、任务管理、时间管理、内存管理、任务间的通信和同步等基本功能,没有提供输入输出管理、文件系统、网络等额外的服务。但是由于uC/OS-II良好的可扩展性和源码开放性,这些功能完全可以由用户根据需要分别实现。
	
\item \textbf{Windows Embedded CE}

	由Microsoft开发出来的一个嵌入式实时操作系统。其内核提供内存管理、抢先多任务和中断处理功能。内核上面是图形用户界面GUI和桌面应用程序。在GUI的内部运行着所有的应用程序,其内核具有32000个处理器的并发处理能力,每个处理器有2GB虚拟内存寻址空间,同时还能够保持系统的实时响应\cite{WindowsEmbeddedCE6.0}。但其缺点是很难实现产品的定制,而且并不具备真正的实时性能,没有足够的多任务支持能力。其主要的应用场景为互联网协议机顶盒、全球定位系统、无线投影仪以及各种工业自动化、消费电子、以及医疗设备等。
	
\item \textbf{RTLinux}
	
	由美国墨西哥理工学院开发的嵌入式实时操作系统,其特殊之处在于开发者并没有针对实时操作系统的特性而重写Linux内核,而是将标准的Linux核心作为实时核心的一个进程,同用户的实时进程一起进行调度。这样对Linux内核的改动非常小,并充分利用了Linux下现有的丰富的软件资源。RTLinux的优点在于:与Linux一样,RTLinux是开放源码的操作系统,在网上较易获得所需的资料和技术支持,使用者可以根据自己的需要进行修改。其主要应用领域包括航天飞机的空间数据采集、科学仪器监控和电影特技图像处理等。
	
\item \textbf{Vxworks}

	由美国Wind River System公司推出的一个实时操作系统,并提供了一套实时操作系统开发环境Tornado,提供了丰富的调试、仿真环境和工具。VxWorks具有良好的持续发展能力、高性能的微内核以及友好的用户开发环境。它支持广泛的网络通信协议、并能够根据用户的需求进行组合,其开放式的结构和工业标准的支持,使得开发者只需要做最少量的工作即可设计出有效的适合于不同的用户要求的系统。因为VxWorks良好的可靠性和卓越的实时性,其广泛的被运用于通信、军事、航天等高精尖和实时性要求极高的领域当中。	
\end{itemize}	
		
	VxWorks作为一款强实时性、高可靠性的操作系统,在我国广泛的运用在军工、航空航天、通信等部门。VxWorks的集成开发调试环境为Tornado,使用该开发环境可以帮助编程人员轻松的完成程序的编辑、编译、调试、系统配置等工作。Tornado拥有一整套完整的面向嵌入式系统的开发和调试工具,包括C和C++远程级调试器、目标和工具管理、系统目标跟踪、内存使用分析和自动配置,所有工具都能够很方便的同时运行,很容易增加扩展和交互式开发。Tornado的调试器包含有GDB命令行接口和WindSh工具,能够很好的进行应用程序的现场开发和调试。但是对于调试信息、日志信息的事后分析却没有提供解决办法且该工具要基于RS-232串口来使用,而现在大多数的设备都已不再配置RS-232串口。
	
	对USB口转串口的设计通常可以采用两种方案,一种是以CY7C68013芯片为代表,自己从底层的固件开始,进行彻底而全面的系统开发,这种方案的成本和开发难度都很大,通常都不会使用这种方案。另外一个方案是采用类似于CP2102等专用的双向USB口转串口芯片来进行设计,这种方案简单实用,只需要对芯片的功能进行了解和应用即可,无需深入开发。因此我们在此会选择CP2102芯片来进行调试通道的设计。
	

\section{论文的主要内容和组织结构}	
	研究目标:在嵌入式实时操作系统VxWorks上实现一个能够满足程序的调试信息输出的通道,主要包括两个部分:一个满足特定要求的、实用的USB转串口驱动程序,一个上层的日志传输接口封装程序和标准输出重定向接口封装程序。\\
 本文共分为六章,各个章节的具体安排如下:
 
 第一章为绪论,主要介绍了本课题的研究背景和意义、国内外的发展状况以及本文的内容的安排。
 
 第二张介绍了进行调试通道的开发所需要了解的系统知识,主要包括VxWorks系统及驱动开发的知识、USB开发的相关知识,最后给出了一个调试通道的总体设计。
 
 第三章介绍了我们使用CP2102模块开发的相关知识和VxWorks下的USB口转串口驱动的具体实现,包括特定需求下的单设备驱动和多设备支持的驱动
 
 第四章主要介绍了应用层的接口封装部分,主要包括Log接口的设计,标准输出重定向接口的设计,以及PC客户端的协议解析部分。
 
 第五章主要内容是系统的功能测试部分。
 
 最后在结束语部分对整个的工作进行了总结,指出了本次的工作的不足之处,并对下一步的工作进行了展望。 

\

%\subsection{第二层}\label{sec:1}
%\subsubsection{第三层}\label{sec:1}
%测试测试测试测试测试测试测试测试测试测试测试测试。
%\footnote{\label{footnote:1}脚注}

\section{字体}

普通\textbf{粗体}\emph{斜体}

\hei{黑体}\kai{楷体}\fangsong{仿宋}

\section{公式}

单个公式,公式引用:\autoref{eq:1}。
\begin{equation}
 c^2 = a^2 + b^2 \label{eq:1}
\end{equation}

多个公式,公式引用:\autoref{eq:2},\autoref{eq:3}。

\begin{subequations}
\begin{equation}
  F = ma \label{eq:2}
\end{equation}
\begin{equation}
  E = mc^2 \label{eq:3}
\end{equation}
\end{subequations}

\section{罗列环境}

\begin{enumerate}
    \item 第一层\label{item:1}
    \item 第一层
    \begin{enumerate}
        \item 第二层\label{item:2}
        \item 第二层
        \begin{enumerate}
            \item 第三层\label{item:3}
            \item 第三层
        \end{enumerate}
    \end{enumerate}
\end{enumerate}

\begin{description}
    \item[解释环境]  解释内容
\end{description}



\clearpage
\chapter{实时操作系统VxWorks}
\section{概述}
\subsection{实时操作系统}
实时操作系统(Real Time Operation System)是整个实时系统的核心。POSIX1003.1标准为RTOS下了一个简单的定义:RTOS是能够在有限的响应时间内为应用提供所要求级别服务的操作系统\cite{Renard20081003}。这个系统能够对任何时间要求苛刻的事件服务,能够在正确的时间内做正确的事情。实时系统按照实时的效果可以分为软实时和硬实时,一个好的实时操作系统能够使你的系统时钟满足要实现的需求,即使在系统的负荷很重的情况下。

现代实时操作系统基于多任务和任务间通信的互补概念。多任务环境允许将实时应用程序构建为一组独立任务,每一个任务都有自己的执行线程和一组系统资源。任务间通信设施允许这些任务同步并进行协调其活动。在VxWorks中,任务间通信工具从快速信号量到消息队列,从管道到网络透明套接字。

	实时系统的另外一个关键设施是硬件中断处理,因为中断处理时通知系统发生外部时间的常用机制。为了尽可能快地响应中断,VxWorks中的中断服务例程(ISR)运行自己的特定上下文中,这个上下文在任何其他的任务上下文之外\cite{Wind2003VxWorks}。

操作系统的核心是内核。内核控制着计算机系统上的所有硬件和软件资源,在必要的时候给应用程序分配硬件资源,并执行相应的操作命令。

	内核的主要功能为以下的四种:

\bullet 系统的内存管理:不仅可以管理服务器上的可用物理内存,还可以创建和管理虚拟内存。

\bullet 软件程序管理:内核控制着系统上运行着的所有程序。

\bullet 硬件设备管理:任何需要操作系统与之进行通信的设备都需要在内核的代码当中加入其驱动程序代码。驱动程序代码相当于应用程序和硬件设备间的中间人,允许内核和设备之间交换数据。

\bullet 文件系统管理:不同于其他的一些操作系统,Linux内核支持通过不同类型的文件系统从硬盘当中读写数据。除了自有的诸多文件系统之外,Linux还支持从其他的操作系统(如windows)采用的文件系统中读写数据。内核必须在编译时就加入对所有可能用到的文件系统的支持。Linux内核采用虚拟文件系统(Virtual File System,VFS)作为和每一个文件系统交互的接口。这为Linux内核同任何类型文件系统通信提供了一个标准的接口,当每个文件系统都被挂载和使用时,VFS将信息都缓存在内存当中。

VxWorks是一种基于微内核技术的实时操作系统。


\section{代码环境}

\begin{lstlisting}[language=python]
import os

def main():
    '''
    doc here
    '''
    print 'hello, world' # Abc
    print 'hello, 中文' # 中文
\end{lstlisting}

\section{定律证明环境}

\begin{definition}\label{def:1}
这是一个定义。
\end{definition}
\begin{proposition}\label{proposition:1}
这是一个命题。
\end{proposition}
\begin{axiom}\label{axiom:1}
这是一个公理。
\end{axiom}
\begin{lemma}\label{lemma:1}
这是一个引理。
\end{lemma}
\begin{theorem}\label{theorem:1}
这是一个定理。
\end{theorem}
\begin{proof}\label{proof:1}
这是一个证明。
\end{proof}

\section{算法环境}

\begin{algorithm}[H]
\SetAlgoLined
\KwData{this text}
\KwResult{how to write algorithm with \LaTeX2e }
initialization\;\label{alg_line:1}
\While{not at end of this document}{
read current\;
\eIf{understand}{
go to next section\;
current section becomes this one\;
}{
go back to the beginning of current section\;
}
}
\caption{How to write algorithms}\label{alg:1}
\end{algorithm}

\section{表格}
表格见\autoref{tab:1}。

\begin{table}[!h]
\centering
\caption{一个表格}\label{tab:1}
\begin{tabular}{|c|c|}
\hline
a & b \\
\hline
c & d \\
\hline
\end{tabular}
\end{table}
\section{图片}
图片见\autoref{fig:1}。图片格式支持eps,png,pdf等。多个图片见\autoref{fig:2},分开引用:\autoref{fig:2-1},\autoref{fig:2-2}。

\begin{figure}[!h]
\centering
\includegraphics[width=.4\textwidth]{fig-example.pdf}
\caption{一个图片}\label{fig:1}
\end{figure}

\begin{figure}[!h]
\centering
  \begin{subfigure}[b]{0.3\textwidth}
  \includegraphics[width=\textwidth]{fig-example.pdf}
  \caption{图片1}\label{fig:2-1}
  \end{subfigure}
  ~
  \begin{subfigure}[b]{0.3\textwidth}
  \includegraphics[width=\textwidth]{fig-example.pdf}
  \caption{图片2}\label{fig:2-2}
  \end{subfigure}
\caption{多个图片}\label{fig:2}
\end{figure}

\section{参考文献示例}
这是一篇中文参考文献\cite{徐媛媛2003嵌入式实时操作系统的设备驱动};这是一篇英文参考文献\cite{9787508342894};同时引用\cite{9780124467422,bamboosilk}。

\section[\textbackslash{}autoref 测试]{\texttt{\textbackslash{}autoref} 测试}

\begin{description}
  \item[公式] \autoref{eq:1}
  \item[脚注] \autoref{footnote:1}
  \item[项] \autoref{item:1},\autoref{item:2},\autoref{item:3}
  \item[图] \autoref{fig:1}
  \item[表] \autoref{tab:1}
  \item[附录] \autoref{appendix:1}
  \item[章] \autoref{chapter:1}
  \item[小节] \autoref{sec:1},\autoref{sec:2},\autoref{sec:3}
  \item[算法] \autoref{alg:1},\autoref{alg_line:1}
  \item[证明环境] \autoref{def:1},\autoref{proposition:1},\autoref{axiom:1},\autoref{lemma:1},\autoref{theorem:1},\autoref{proof:1}
\end{description}

% backmatter用于表示论文的正文结束
\backmatter

%ack 环境用于致谢页面
\begin{ack}
致谢正文。
\end{ack}

% bibliography用于生成参考文献。
\bibliography{ref/myref.bib}

% appendix环境用于附录环境。即可以将附录置于appendix环境当中。如:
% \begin{appendix}
%  <content>
% \end{appendix}

% 直接使用\appendix 则表明后文均为附录。如:
% \appendix
%  <content> 
\appendix

% publications环境用于已经发表了的论文的页面,一般用于附录当中,使用上同enumerate环境
\begin{publications}
    \item 论文1
    \item 论文2
\end{publications}

\chapter{这是一个附录}\label{appendix:1}
附录正文。


\end{document}

\endinput
%%
%% End of file `hustthesis-zh-example.tex'.
