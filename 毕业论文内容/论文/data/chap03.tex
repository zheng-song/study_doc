\chapter{驱动程序的设计和实现}

\section{设备驱动概述}
	通常为了实现与应用程序的平台无关性,操作系统都会为应用程序提供一套标准的接口,VxWorks也不例外,这样就可以通过调整底层驱动或者是接近驱动那部分的操作系统中间层来提高应用层开发的效率,避免重复编码。在我们的通用操作系统(如Mac OS、Linux、Windows)当中,通常会将这套应用层的接口标准从操作系统中独立出来,专门以标准库的形式存在,这样可以屏蔽操作系统之间的差异,增强应用程序的平台无关性。
	
	VxWorks中也对应用层提供了一套标准的文件操作接口,实际上与GPOS提供接口类似,我们将其称作为标准I/O 库,VxWorks下由ioLib.c文件提供。ioLib.c文件提供如下标准接口函数:creat()、open()、unlink()、remove()、close()、rename()、read()、write()、ioctl()、lseek()、readv()、writev()等\cite{BSP开发人员指南},VxWorks与通用操作系统有很大的一个不同点是:VxWorks不区分用户态和内核态,用户层可以直接对内核函数进行调用,而无需使用陷阱指令之类的机制,以及存在使用权限上的限制。因此VxWorks提供给应用层的接口无需通过外围库的方式,而是直接以内核文件的形式提供。
	
	设备驱动时直接控制设备操作的那部分程序,也是设备上层的一个软件接口。设备驱动程序的功能完成软件层对硬件的访问,实际上从软件工程的角度来说就是介于软件和硬件之间实现软件层标准接口的程序。软件层访问硬件必须要通过调用驱动程序。所以驱动程序不能自动执行,只能被系统或者程序调用。
	
	应用程序必须要通过驱动程序才能够与硬件设备进行通信,而驱动程序的编写与操作系统的关系密不可分。设备驱动程序在操作系统中如何存在、如何与操作系统的其它部分相联系、如何与操作系统的其他部分相联系、如何为用户提供服务都是操作系统的设计人员在设计操作系统时制定的,系统已经为驱动程序制定好了一个框架,无论驱动程序的开发人员以何种方式控制设备,他们所开发的驱动程序都是以预先设计好的方式存在、与操作系统其他部分相联系和为用户提供服务的。将这种由操作系统的设计人员指定的设备驱动程序结构定义为驱动程序的外部结构,而由于驱动开发人员在开发设备驱动时采用的具体策略不同导致的不同的驱动程序结构称为驱动程序的内部结构。驱动程序的外部结构决定了操作系统的I/O体系结构,驱动程序的内部结构决定了不同的设备驱动方式。

	在VxWorks系统中,在控制器权转到设备驱动程序之前,用户的请求进行尽可能少的处理。VxWorks I/O系统的角色更像是一个转接开关,负责将用户请求转接到合适的驱动例程上。每一个驱动都能够处理原始的用户请求,到最合适它的设备上。另外,驱动程序开发者也可以利用高级别的库例程来实现基于字符设备或者块设备的标准协议。因此,VxWorks的I/O系统具有两方面的优点:一方面使用尽可能少的使用驱动相关代码就可以为绝大多数设备写成标准的驱动程序,另一方面驱动程序开发者可以在合适的地方使用非标准的程序自主的处理用户请求。


\subsection{设备驱动的功能以及组成部分}
\subsubsection{设备驱动程序结构}
	VxWorks 的I/O框架由ioLib.c 文件提供,但ioLib.c文件提供的函数仅仅是一个最上层的接口,并不能完成具体的用户请求,而是将请求进一步向其他内核模块进行传递,位于ioLib.c模块之下的模块就是iosLib.c。我们将ioLib.c 文件称为上层接口子系统,将iosLib.c文件称为I/O 子系统,注意二者的区别。上层接口子系统直接对用户层可见,而I/O 子系统则一般不可见(当然用户也可以直接调用iosLib.c 中定义的函数,但一般需要做更多的封装,且违背了内核提供的服务层次),其作为上层接口子系统与下层驱动系统的中间层而存在。
	

	

\begin{figure}[!h]
\centering
\includegraphics[width=.4\textwidth]{VxWorks-driver-structure.pdf}
\caption{VxWorks驱动内核层次结构}\label{fig:VxWorks内核驱动层次结构}
\end{figure}

	I/O 子系统在整个驱动层次中起着十分重要的作用,其对下管理着各种类型的设备驱动。换句话说,各种类型(包括网络设备)的设备都必须向I/O 子系统进行注册方可被内核访问。所以在I/O 子系统这一层次,内核维护着三个十分关键的数组用以对设备所属驱动、设备本身以及当前系统文件句柄进行管理。

	需要指出的是,VxWorks文件系统在内核驱动层次中实际上是作为块设备驱动层次中的一个中间层而存在的,其向I/O 子系统进行注册,而将底层块设备驱动置于自身的管理之下以提高数据访问的效率。在这些文件系统中,dosFs 和rawFs 是最常用的两种文件系统类型,在VxWorks早期版本就包含对这两种文件系统的支持。

\subsubsection{驱动程序管理表}
\subsubsection{驱动程序管理步骤}







\section{USB转串口设备驱动程序的实现}
在应用程序可以与一个设备通信之前,主机需要知道设备支持哪一些传输类型和终端,主机也必须要分配一个地址给设备。主机通过一个被称为列举的信息交换来完成这些工作。以下叙述USB列举的基本过程:
\begin{enumerate}
\item \hei{USB设备与主机系统的交互}

	MCU对USB芯片进行初始化:设置内部时钟,选择内部连接方式以及是否开通DMA传输等;设定设备的工作模式,并设备本设备的初始地址为0(USB规范指明当设备接入PC的时候都由初始为0的地址对主机进行响应,之后再由主机分配一个地址给USB设备,设备接收到分配地址命令后,再更改自己的地址,并一直通过这个地址完成后续的通信)。
	完成初始化工作之后,MCU将使能USB接口,主机系统将因此检测到一个新的USB接入而很快与设备进行握手,获取设备的基本信息,并完成一些列的对设备的配置,其过程为:
	\begin{enumerate}
	\item USB上电使能后,主机会向USB设备发送GET DEVICE DESCRIPTOR的命令,之后主机会收到设备发出的设备描述符,随即为设备分配一个空闲地址,并向设备发送SET ADDRESS的命令,这时设备通过地址0发送一个长度为0的数据包予以应答,然后根据主机的要求更改自身的地址,而且这以后的数据交换都将会通过这个新的地址来进行。
	\item 完成地址设备之后,主机将会发送GET CONFIGURATION DESCRIPTOR,USB规定当主机发出该命令符的时候,设备必须要同时返回配置所包含的所有接口和接口所包含的所有端点的描述符。
	\item 主机获取到USB设备的描述符、配置描述符并进行了地址设置之后,设备与主机的握手初步完成,之后会将该设备加入到设备列表当中。
	\end{enumerate}
	
	\item \hei{USB设备与驱动程序的交互}
	
USB设备的驱动与传统意义上的硬件驱动不完全相同,他并不与硬件直接通信,而是以创建和发送URB请求块的形式把命令传递给操作系统所提供的USB总线驱动程序,由总线驱动程序来完成与硬件的直接交互。和主机类似,驱动会首先创建和发送请求得到该设备的DEVICE DESCRIPTOR的URB,并将获取到的信息存储在专用的数据结构当中,接着驱动为了得到完整的设备配置,必须要通过总线驱动发送两次GET CONGFIGURATION DESCRIPTOR命令得到设备。获取设备的配置信息之后,驱动在启动设备之前还要发送SET CONFIGURATION、SET INTERFACE命令。通过以上的几个步骤便可以完成驱动与USB设备过程。	
\end{enumerate}

	当设备接入到PC的USB接口的时候,在固件和操作系统的支持下会对设备进行枚举操作,
	
	由于我们的USB驱动程序不能够支持处理中断,所以只能够用查询的方式来连续接收数据。可以使用两种方式实现设备的查询,一是使用计时器,另外就是使用系统线程。在我们的USB口转串口驱动程序中,我们使用两个线程实现数据的接收操作。在第一次设置PID过滤参数之后,开启一个系统线程,它会不停的从IN端口读取数据,然后写入到驱动程序中开辟的缓冲区当中;


\subsection{特定需求的单设备支持}

驱动程序各模块的实现:
\begin{enumerate}
\item  cp210xDevInit模块

	

\end{enumerate}


\subsubsection{USB设备硬件的初始化}


\subsection{通用的多设备支持}