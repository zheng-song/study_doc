\chapter{应用层程序接口封装}
	从VxWorks6.x开始引入了RTP(VxWorks Real Time Process Project)模式,这种模式的优点是应用程序之间互相独立、互不影响,而且增加了内核的稳定性,缺点是由于“内核态”与“用户态”的内存拷贝,其执行效率有所降低,随着CPU速度越来越快,这点效率的牺牲已经越来越不重要。相比较于传统的DKM(downloadable kernel module project),RTP适合多个团队独立运作,然后汇总联试,这种模式除了全局函数不能在shell里直接调用外,其对应用程序几乎不做任何约束,原有的DKM工程代码稍作修改即可正常运行。内核变化较大,需要添加较多的组件,内存需要较好的划分,为保持应用程序直接调用函数调试的习惯,需要封装接口供用户使用。

\section{Log协议的设计}

我们为调试信息制定的协议格式为:\backslash 03\backslash 03<L=日志级别;PN=任务ID;P=任务名;F=文件名;N=行号;T=时间>contents\backslash 04\backslash 04\\
其中各部分的含义如下:

\begin{itemize}
\item \textbf{\backslash 03\backslash 03}:表示自定义的Log协议的数据包的开始;

\item \textbf{<}:表示自定义的Log协议数据包头部的开始;

\item \textbf{L}:表示日志的级别,我们在此将日志分为五个级别:
	\begin{itemize}
	\item e:表示error;
	
	\item w:表示warning;
	
	\item i:表示info;
	
	\item d:表示debug;
	
	\item o:表示其他信息。
	\end{itemize}
	
\item \textbf{PN}:此处的内容是输出该条调试信息的任务的任务ID;

\item \textbf{P}:此处的内容是输出该条调试信息的任务的任务名;

\item \textbf{F}:此处的内容是输出该条调试信息的任务所在的文件名;

\item \textbf{N}:此处的内容是该调试信息语句所在的文件的行号;

\item \textbf{T}:此处的内容是这条调试信息被输出时候的系统时间;

\item \textbf{>}:表示自定义的Log协议数据包头部的结束;

\item \textbf{contents}:这个部分是调试信息的正文部分。

\item \textbf{\backslash 04 \backslash 04}:表示自定义的 Log 协议数据包的结束。

\end{itemize}\\


\section{标准输出重定向接口的设计}
	由于标准输出的重定向无法在RTP模式和task模式下使用同一种方法来实现,于是我们使用了两种方法来分别实现RTP模式和task模式下的标准输出重定向。
\subsection{RTP模式下标准输出重定向}
	RTP模式下的标准输出重定向流程如\autoref{fig:rtp-printf-reset}所示。部分关键代码如下:
\lstset{language=C}
\begin{lstlisting}
int ResetStdOut(int usb_serial)
{
  ...
	
  _init_fd();
  if(usb_serial == 1)
  {
    if(dup2(log_fd,STD_OUT) < 0)
    {
      printf("can not reset STDOUT to /hust_use_serial\n");	  
	  return -1;
	}
  }
  else
  {
    if (dup2(STDOUT_FD,STD_OUT) < 0)
    {
	  printf("can not reset STDOUT to /hust_use_serial\n");
	  return -1;
	}
  }
  
  ...
}
\end{lstlisting}

\begin{figure}[!h]
\centering
\includegraphics[width=0.7\textwidth]{./graphics/rtp-printf-reset.pdf}
\caption{RTP模式下标准输出重定向流程图}\label{fig:rtp-printf-reset}
\end{figure}


在RTP模式下使用dup2函数来实现标准输出的重定向,dup2函数的原型为:
\lstset{language=C}
\begin{lstlisting}
int dup2(int oldfd, int newfd);
\end{lstlisting}\\
dup2()用于复制描述符oldFd到newFd的,其中oldFd是要被复制的文件描述符,newFd是制定的新文件描述符,如果newFd已经打开,它将首先被关闭。如果newFd等于oldFd,dup2会返回newFd,但是不会关闭它。函数调用成功时会返回新的文件描述符,所返回的新的描述子与参数oldFd给定的描述符字引用同一个打开的文件,即共享同一个系统打开文件表项。函数调用失败时会返回-1并设置errno。

\subsection{task模式下标准输出重定向}
在task模式下无法使用dup()/dup2()函数来进行标准输出的重定向,在task模式下VxWorks有专用的标准输出接口ioTaskStdSet(),我们在此模式下只能使用这个借口来实现重定向,task模式下的标准输出重定向如\autoref{fig:task-printf-reset}所示。部分关键代码如下:
\lstset{language=C}
\begin{lstlisting}
int ResetStdOut(int usb_serial)
{
  ...
	
  _init_fd();
  if(usb_serial == 1)
  {
    ioTaskStdSet(0,STD_OUT,log_fd);
  }
  else
  {
    ioTaskStdSet(0,STD_OUT,STDOUT_FD);
  }
  
  ...
}
\end{lstlisting}

\begin{figure}[!h]
\centering
\includegraphics[width=0.7\textwidth]{./graphics/task-printf-reset.pdf}
\caption{task模式下标准输出重定向流程图}\label{fig:task-printf-reset}
\end{figure}

ioTaskStdSet()是VxWorks专门用来进行任务级的重定向的函数。其函数原型为:
\lstset{language=C}
\begin{lstlisting}
void ioTaskStdSet(int taskId, int stdFd, int newFd);
\end{lstlisting}\\
在VxWorks中每一个任务都有一个数组taskStd,用于表明这个任务的标准输入、标准输出、标准错误,函数ioTaskStdSet()的功能就是将特定任务的标准描述符重定向到newFd,newFd需要是一个文件或者设备的描述符。第一个参数taskId表示需要进行重定向的任务的ID(ID为0表示该任务本身),第二个参数是需要被重定向的某个标准描述符(0,1,2),第三个参数是需要重定向到的文件描述符。该函数没有返回值。


\section{Log接口的设计}
	Log接口函数用于完成标准格式的log的输出,使用时只需要调用LogE()、LogW()、LogI()、LogD()、LogO(),这几个接口均为宏定义,定义在usb\_ logWrite.h当中,在使用时需要包含该头文件,作用是获取log协议所需要的部分信息,其代码如下所示:
\lstset{language=C}
\begin{lstlisting}
#define LogE(format, ...) usb_logWrite('e',__FILE__,__LINE__,format,##__VA_ARGS__)

#define LogD(format, ...) usb_logWrite('d',__FILE__,__LINE__,format,##__VA_ ARGS__)

#define LogI(format, ...) usb_logWrite('i',__FILE__,__LINE__,format,##__VA_ ARGS__)

#define LogW(format, ...) usb_logWrite('w',__FILE__,__LINE__,format,##__VA_ ARGS__)

#define LogO(format, ...) usb_logWrite('o',__FILE__,__LINE__,format,##__VA_ ARGS__)

extern int usb_logWrite(char level,char *fileName, int lineNum, const char * format, ...);
\end{lstlisting}
LogE(),LogW(),LogD(),LogO,LogI()均由usb\_ logWrite()函数来实现,usb\_ logWrite()函数实现真正的完整的协议封装和调用驱动发送的过程,usb\_ logWrite()完成协议头部信息的获取,包括日志的级别,发送该日志的进程号和进程名,打印该日志的文件的文件名,该日志在文件中所处的行号。并将这些信息封装在所定义的头部格式当中。最后将用户需要输出的信息放入协议的数据部分,并添加结束标志,然后调用驱动程序将该数据包发送出去。usb\_ logWrite()的部分关键代码如下所示:
\lstset{language=C}
\begin{lstlisting}
int usb_logWrite(char level,char *fileName, int lineNum, const char * format, ...)
{
  ...
   
  struct timespec tp;
  struct tm timeBuffer;
  time_t nowSec;
  char datetime[64];
  clock_gettime(CLOCK_REALTIME,&tp);
  nowSec = tp.tv_sec;
  localtime_r(&nowSec,&timeBuffer);
  timeLen = strftime(datetime,64,"%Y/%m/%d %H:%M:%S",&timeBuffer);
  sprintf(datetime+timeLen,".%3.3ld",tp.tv_nsec/1000000L);

  _init_fd();
  if(fileName != NULL) 
  {
    char *rf = strrchr(fileName, '/');
	if(rf != NULL) fileName = rf+1;
  }

  logWriteBuf[0]=0x03;
  logWriteBuf[1]=0x03;
  n = snprintf(&logWriteBuf[2],LOG_BUF_SIZE-2,"<L=%c;PN=%d;P=%s;F=%s;N=%d;T=%s>",level,Id,name,fileName,lineNum,datetime);
  n+=2;
  va_list argList;
  va_start(argList,format);
  m = vsnprintf(logWriteBuf+n,LOG_BUF_SIZE-n,format,argList);
  va_end(argList);

  if(m <= 0) 
  {
	m = snprintf(logWriteBuf+n,LOG_BUF_SIZE-n, "format error\n");
  }
  n += m;
  
  if(n > LOG_BUF_SIZE-2) n = LOG_BUF_SIZE-2;
  logWriteBuf[n++] = 0x04;
  logWriteBuf[n++] = 0x04;

  write(log_fd, logWriteBuf, n);
  return n;
}
\end{lstlisting}

usb\_ logWrite()函数的实现在RTP模式和task模式之下是一样的,在task模式下只需包含usb\_ logWrite.h头文件即可,在RTP模式下需要包含usb\_ logWrite.h和usb\_ logWrite.c两个文件。


\section{Windows下的日志分析工具}
	
	由于日志分析工具并不是我们本次论文的介绍重点,此处我们只介绍调试信息的接收部分协议的解析相关的内容,对于其他的部分不做详细介绍;日志分析工具为Windows下使用QT开发的界面程序,其结构如\autoref{fig:日志分析工具结构图}所示,主界面如\autoref{fig:RoutonLog} 所示。
\begin{figure}[!h]
\centering
\includegraphics[width=1.0\textwidth]{./graphics/routonLog-system-structure.pdf}
\caption{日志分析工具结构图}\label{fig:日志分析工具结构图}
\end{figure}
\begin{figure}[!h]
\centering
\includegraphics[width=1.0\textwidth]{./graphics/routonLogNoRun.pdf}
\caption{主机端日志分析工具界面}\label{fig:RoutonLog}
\end{figure}\\
\textbf{串口读取流程}
日志界面的串口读取流程图如\auotref 所示,当用户打开串口后,主窗口中的串口读取函数会对串口中的数据进行非堵塞地读取,并按协议格式进行解析、显示和存盘。对于非协议格式的数据,则按照普通标准输出重定向过来的数据进行显示。此时在日志信息显示框中只会将信息显示在详细内容部分,日志级别、进程号、文件名、行号等内容均为空白。对于按照协议格式发送的信息,会按照信息的级别以不同的底色进行显示。
\begin{figure}[!h]
\centering
\includegraphics[width=.6\textwidth]{./graphics/routonTTYRead.pdf}
\caption{日志界面串口读取流程图}\label{fig:串口读取流程图}
\end{figure}

