\chapter{绪论}
\section{课题背景以及来源}
	嵌入式系统目前不断发展和壮大,使用嵌入式的场景越来越多,同时对嵌入式系统进行软件开发和移植工作也越来越多,由于每种嵌入式系统都有不同的软件接口和特性。所以在不同的嵌入式系统上进行软件开发和移植工作要根据其特性使用不同的方法,在我们所熟悉的通用操作系统(Linux、Windows、MacOS)下进行软件设计和开发时它们通常都会有现成的集成开发环境和调试工具。
	然而在嵌入式软件的开发和移植过程当中,由于嵌入式设备所独有的特点,对嵌入式系统上的软件的调试、分析一直是一个费时费力的工作。一个好的调试、分析工具可以给嵌入式软件开发人员带来很大的帮助,使其达到事半功倍的效果,快速完成软件开发过程中的调试、分析,软件运行过程中调试信息的定位等工作。
		
	目前国内外已有几十种商业化嵌入式操作系统可以供选择,如VxWorks、uc/OS-II、Windows Embedded CE、RTLinux、和“女蜗Hopen”等。其中VxWorks因为其可靠性和实时性被广泛地应用对系统的实时性要求很高的领域当中,如:通信、航天、军事等\cite{刘小军2008基于}。
	在进行VxWorks应用程序的开发或者是将Linux下的应用程序移植到VxWorks中时都需要在程序中加入大量的调试信息,在程序的正常运行当中也需要输出一些日志信息,方便之后对程序运行过程中产生的问题进行具体的分析。在VxWorks的集成开发环境Tornado当中有一个驻留在主机端的命令行解释器WindSh和GDB命令行,但是这个工具是位于主机上的,需要通过通过网口或者串口将控制命令传输到嵌入式设备上,然后再将输出信息传输回主机上进行显示。然而目前的大多设备都已不再将RS-232的串口作为必须的标准接口,而是大量的使用USB接口,因此在调试信息的传输过程中只能够使用USB接口。
	
	
	本次基于VxWorks的调试通道的设计来源于中船重工的实际需求,由于实际的生产环境的限制导致他们无法使用VxWorks集成开发环境当中提供的调试工具,我们基于VxWorks开发出一个小巧易用的调试信息的传输通道,方便它们将应用程序的调试信息传输出来,以进行一个事后的分析工作,对于底层的信息传输我们使用USB口转串口来实现,USB口转串口的转换器使用CP2102模块来实现,并在此基础上设计了给应用层使用的调试接口。
	
		
			
\section{国内外概况}
	嵌入式系统在近年发展迅猛,随着物联网的兴起,其更加广泛的应用于各种智能设备当中。而嵌入式实时操作系统作为嵌入式操作系统中定位更加精准的一类,其应用场景更加的专业化\cite{解月江2004VxWorks设备驱动技术研究}\cite{VxWorks嵌入式实时操作系统的结构研究},对系统的可靠性、实时性等方面有严格要求。
	目前国内外在使用的嵌入式实时操作系统有上百个,这些操作系统面向不同的专业领域,具有各自不同的特性。以下介绍几个具有代表性的操作系统:
\begin{itemize}
\item \textbf{uc/OS-II}
	
	由Micrium公司开发的一个开源的实时操作系统内核,其具有可移植、可固化、可裁剪等优点\cite{Wu2008Implementation}。在uc/OS-II提供的是一个微内核,在内核当中只包含了一些基本的服务,如任务的管理、任务间通信、时间管理、内存管理、中断处理等\cite{马增炜2011基于}\cite{Zhang2010Design},这有利于保证内核的精简、高效。对于其他的复杂的功能如文件系统、网络服务等需要用户根据自己的实际需求自己来实现。
	
	
\item \textbf{Windows Embedded CE}

	由Microsoft开发出来的一个嵌入式实时操作系统。其内核也会提供基本的服务,如:内存管理、抢先多任务和中断处理功能等,除了常用的基本服务之外它的内核也提供了诸如网络服务、文件系统等复杂的功能,这导致其内核的体积比较大,不够灵活。
	Windows Embedded CE在内核上面还实现了图形用户界面,图形用户界面的存在导致其资源消耗较大\cite{徐媛媛2003嵌入式实时操作系统的设备驱动}\cite{谢强2007基于}。Windows Embedded CE的内核虽然具有并发处理能力,但是其并不具备真正的硬实时性能,这导致该系统主要应用于工业自动化、消费电子、以及医疗设备等实时性要求不高的领域。
	
\item \textbf{RTLinux}
	
	由美国墨西哥理工学院开发的一个开源嵌入式实时操作系统,RTLinux没有针对实时操作系统的特性重写Linux内核,它使用了一种比较巧妙的方式来实现实时性,它通过在Linux内核和硬件之间增加一个虚拟层,实现了一个与Linux内核分离的、小巧的、时间上可预测的实时内核,使得运行其中的实时进程能够满足实时性的要求\cite{郭春生2002硬实时操作系统},这样的实现能够利用Linux下现有的丰富的软件资源\cite{Zhu2004RTLinux}。
\item \textbf{Vxworks}

	由美国Wind River System公司开发的一个实时操作系统,其内核提供可裁剪、可移植等特点,其具有抢占式调度、多任务优先级、强实时性、高可靠性等特点\cite{李立志2003实时操作系统}\cite{陈洋2007VxWorks},开发者可以对其内核提供的各种功能根据自己的需要进行裁剪,最小能裁剪到8K,这让开发者能够设计出有效的适合于不同的用户要求的系统\cite{谢强2007基于}\cite{徐媛媛2003嵌入式实时操作系统的设备驱动}。同时它还提供了一套集成开发环境Tornado,在开发环境当中提供了丰富的调试工具和仿真环境。	
\end{itemize}	
		

	VxWorks广泛的运用在我国的军工、航空航天、通信等部门\cite{陈洋2007VxWorks}\cite{张鹏2007基于}。VxWorks的集成开发调试环境为Tornado,使用该开发环境可以帮助编程人员轻松的完成程序的编辑、编译、调试、系统配置等工作\cite{嵌入式实时操作系统VxWorks及其开发环境Tornado}\cite{Tronado}。Tornado的调试器包含有GDB命令行接口和WindSh工具,能够很好的进行应用程序的现场开发和调试。但是对于调试信息、日志信息的事后分析却没有提供解决办法且该工具要基于RS-232串口来使用,而现在大多数的设备都已不再配置RS-232串口,只配置有USB接口,而VxWorks上并没有实现好的USB口转串口的驱动程序,只实现了标准的USB协议栈和串口驱动,因此我们需要自己在VxWorks下实现一个USB口转串口驱动。
	
	对于USB口转串口的转换器,国内外通常都会采用两种方案:一种是以CY7C68013芯片为代表,自己从底层的硬件和固件开始,进行彻底而全面的系统开发,这种方案的成本和开发难度都很大,通常都不会使用这种方案。另外一个方案是采用类似于CP2102等专用的双向USB口转串口芯片来进行设计,这种方案简单实用,只需要对芯片的功能进行了解和应用即可,无需深入开发\cite{Yao2009Design}\cite{Zhou2002The}。因此我们在此会选择CP2102芯片来进行调试通道的设计。	
	
	

\section{论文的主要工作和组织结构}	
	主要工作:在嵌入式实时操作系统VxWorks上实现一个能够满足程序的调试信息输出的通道,主要包括两个部分:一个将USB总线技术和RS-232接口相结合,设计出一个满足特定要求的、实用的USB转串口驱动程序;另一个是设计给应用层调用的日志传输接口封装程序和标准输出重定向接口封装程序。\\
 本文共分为六章来进行描述,对每一个章节我们做了如下的安排:
 
 第一章为绪论部分,主要描述了本次的课题的背景和来源、国内外的发展状况以及本文的结构安排。
 
 第二章介绍了首先介绍对于我们的调试通道的总体设计,然后介绍了调试通道的开发所需要了解的系统知识和关键技术,主要包括VxWorks系统及驱动开发的知识、USB技术相关知识。
 
 第三章介绍了USB口转串口驱动程序的设计和实现,包括驱动程序程序当中对于缓冲区和信号量的设计,我们使用CP2102模块开发和VxWorks下的USB开发的内容,然后给出了USB口转串口驱动的具体实现,包括特定需求下的单设备驱动和多设备支持的驱动
 
 第四章主要介绍了应用层的接口封装部分,主要包括Log接口的设计,标准输出重定向接口的设计,以及PC客户端的协议解析部分。
 
 第五章主要内容是系统的功能测试部分,我们进行了整体测试和各个部分的功能测试。
 
 最后在结束语部分对整个的工作进行了总结,指出了本次的工作的不足之处,并对下一步的工作进行了展望。 

