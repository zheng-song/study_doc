\chapter{绪论}
\section{课题背景以及意义}
	当今,在嵌入式领域,嵌入式技术(Embedded Technology)已经成为了新的技术热点。嵌入式系统的最典型的特色是它同人们的日常生活紧密相关。小到MP3、PDA等微型数字化设备。大到信息家电、智能电视、车载GPS等形形色色运用了嵌入式技术的电子产品。目前各种新型嵌入式设备在数量上已经远远超过了通用计算机。在嵌入式设备发展的30多年的历史中,嵌入式技术从来没有像现在这样风靡,人类也从来没有像现在这样享受嵌入式技术带来的便利。

	目前国内外已有几十种商业化嵌入式操作系统可以供选择,如VxWorks、uc/OS-II、Windows Embedded CE、RTLinux、和“女蜗Hopen”等。其中VxWorks以其良好的可靠性和卓越的实时性被广泛地应用在通信、军事、航空、航天等高精尖技术及实时性要求极高的领域中,如卫星通讯、军事演习、弹道制导、飞机导航等。而Linux操作系统则是完全开源的,在全世界拥有几十万的开源项目,目前主流的Android及嵌入式设备都采用Linux操作系统。
	
	在我国VxWorks大量的应用于我国的军事、国防工业当中,通常在进行VxWorks应用程序的开发或者是将Linux下的应用程序移植到VxWorks中时都需要输出大量的调试信息,这样可以方便编程人员对系统进行分析、调试。同时基于应用程序的可移植性的考虑,VxWorks在其内核Wind中也支持 POSIX 1003.1b的规定和1003.1中有关基本系统调用的规定,包括:过程初始化、文件与目录、I/O 初始化、语言服务、目录处理;而且 VxWorks 还支持 POSIX 1003.1b的实时扩展,主要包括:异步 I/O、记数信号量、消息队列、信号、内存管理和调度控制等\cite{Wind2003VxWorks}。当然有些POSIX接口也有VxWorks下的实现重新实现版本,比如定时器、二进制信号量、消息队列等。VxWorks下专用的实现版本和POSIX兼容的实现版本在性能上有一些差异,POSIX的接口主要是为简化Linux下程序的移植而设立的。对于没有POSIX接口的部分则必须将程序修改为VxWorks下的接口,对于修改了的代码在VxWorks下运行可能会出现各种问题,本文为了给出一个事后对程序进行分析的方法,于是设计了这个基于VxWorks下的调试通道。
	
	目前开发人员碰到使用新的硬件方案,厂家基本上不提供VxWorks的Bsp,只支持Linux。这个给开发人员增加了很大的开发成本,所有的程序跟模块都要重新设计,以适应新的系统,增加了开发周期。本文设计一种方案,可以很方便地将VxWorks的应用程序搬迁到Linux上,为厂家缩短开发周期,减少程序搬迁风险,大大降低了开发成本。本文先介绍VxWorks的技术原理,然后分析技术实现细节,最后拿一个应用程序作为例子。 
		
\section{国内外概况}
	随着计算机技术的突飞猛进,实时系统无论是在技术上还是在应用领域当中都取得了辉煌的成就,尤其是在最近的二三十年当中,随着物联网的兴起,各种智能设备都需要安装嵌入式操作系统,导致嵌入式操作系统的发展愈加迅猛。而嵌入式实时操作系统属于嵌入式操作系统中定位更加精准的一类,其应用场景更加的专业化。
	
\subsection{现有的RTOS}
	目前国内在使用的RTOS有上百个,这些操作系统面向不同的专业领域,具有各自不同的特性。以下介绍几个具有代表性的操作系统:
\begin{itemize}
\item \textbf{uc/OS-II}
	
	是一个由Micrium公司提供的可移植、可固化、可裁剪、抢占式多任务实时内核,在这个内核之上提供了最基本的系统服务,该内核适用于多种微处理器、微控制器和数字处理芯片。该实时操作系统内核的特点是仅仅包含了任务调度、任务管理、时间管理、内存管理、任务间的通信和同步等基本功能,没有提供输入输出管理、文件系统、网络等额外的服务。但是由于uC/OS-II良好的可扩展性和源码开放性,这些功能完全可以由用户根据需要分别实现。
	
\item \textbf{Windows Embedded CE}

	由Microsoft为开发各类功能强大的信息设备而开发出来的一个嵌入式实时操作系统。其内核提供内存管理、抢先多任务和中断处理功能。内核上面是图形用户界面GUI和桌面应用程序。在GUI的内部运行着所有的应用程序,其内核具有32000个处理器的并发处理能力,每个处理器有2GB虚拟内存寻址空间,同时还能够保持系统的实时响应\cite{WindowsEmbeddedCE6.0}。Windows Embedded CE以多种方式将一个虚拟的桌面计算机置于掌上或放置于口袋中,可以看做是Windows98/NT的微缩版。但是从技术的角度而言,Windows Embedded CE并不能构称得上是一个优秀的RTOS,首先其很难实现产品的定制,其次它占用过多的RAM而且并不具备真正的实时性能,没有足够的多任务支持能力。其主要的应用场景为互联网协议机顶盒、全球定位系统、无线投影仪以及各种工业自动化、消费电子、以及医疗设备等。
	
\item \textbf{RTLinux}
	
	由美国墨西哥理工学院开发的嵌入式实时操作系统,其特殊之处在于开发者并没有针对实时操作系统的特性而重写Linux内核,而是将标准的Linux核心作为实时核心的一个进程,同用户的实时进程一起进行调度。这样对Linux内核的改动非常小,并充分利用了Linux下现有的丰富的软件资源。RTLinux的优点在于:与Linux一样,RTLinux是开放源码的操作系统,使用者可以根据自己的需要进行修改,同时由于其开源,在网上较易获得所需的资料和技术支持。其主要应用领域包括航天飞机的空间数据采集、科学仪器监控和电影特技图像处理等。
	
\item \textbf{Vxworks}

	由美国Wind River System公司推出的一个实时操作系统,并提供了一套实时操作系统开发环境Tornado,提供了丰富的调试、仿真环境和工具。VxWorks具有良好的持续发展能力、高性能的微内核以及友好的用户开发环境。它支持广泛的网络通信协议、并能够根据用户的需求进行组合,其开放式的结构和工业标准的支持,使得开发者只需要做最少量的工作即可设计出有效的适合于不同的用户要求的系统。因为VxWorks良好的可靠性和卓越的实时性,其广泛的被运用于通信、军事、航天等高精尖和实时性要求极高的领域当中。	
	 
\end{itemize}

	Linux操作系统经过几十年的发展,其拥有大量的开发人员,成千上万的开源项目,成为了开发人员的首选。在移动领域借助Android与IOS平分天下。在嵌入式领域,近年来CPU主频的发展迅猛,CPU主频的增加克服了Linux分时操作系统的实时缺陷,使得基于Linux的实时操作系统的精度已经完成可以满足大部分实时领域的要求,特别在民用领域。但是在强实时的工业控制,航天,军用领域当中,仍然需要强实时的专用操作系统VxWorks。 	
	
	VxWorks操作系统是美国WindRiver公司于1983年推出的一种嵌入式实时操作系统(RTOS),是一个运行在目标机上的高性能、可裁剪的实时操作系统,是一个专门为嵌入式实时系统设计开发的操作系统,其具有良好的持续发展能力、高性能的内核以及友好的用户开发环境,为开发人员提供了高效的实时多任务调度、中断管理、实时的系统资源以及实时的任务间通信。在嵌入式实时操作系统领域当中占有一席之地。VxWorks支持X86、PowerPC、ARM等众多主流的处理器,且在各种CPU平台上提供了统一的编程接口和一致的运行环境。在军事、航空航天、工业控制、通信等高精尖以及实时性要求极高的领域当中,有着更加广泛而深入的应用。应用实例包括1997年火星表面登入的火星探测器、爱国者导弹、飞机导航、F-16、FA-18战斗机等。
	
\subsection{VxWorks下驱动程序的开发}
	
	VxWorks作为一款嵌入式操作系统,为了屏蔽硬件的具体操作细节,为上层应用程序提供统一的接口,引入了设备驱动程序的概念。设备驱动程序是用来直接控制设备,以完成设备应有的操作。操作系统通过驱动程序来对设备进行操作,属于一种软件接口。这样,应用程序开发使用统一的软件接口,使得开发人员可以专注于应用程序的开发,不用考虑底层的物理设备。
	
	VxWorks操作系统下的驱动程序在其开发上有自身规范,且对于不同的设备的驱动程序开发也表现出较大的差异。嵌入式设备的硬件平台千差万别,厂商不可能遇见所有的设备并提供相应的驱动程序,因此要实现VxWorks的跨平台移植,用户必须要根据硬件平台开发驱动程序。所以,开发VxWorks操作系统下外围设备的驱动程序具有很大的实际应用价值。
		
	虽然VxWorks作为实时嵌入式系统有着无比强大的性能表现,但是就目前的现状来看,VxWorks操作系统的平台成本很高,因为VxWorks是一个专用系统,使用这个系统的单位或公司都是处于军事,航天等特殊行业当中,这使得他们的开发经验和资源不能够拿出来共享,导致VxWorks的参考资料比较少,没有成熟活跃的社区支持,且VxWorks并不是一个开源的系统,需要花不菲的价格进行购买和售后支持,这些都提高了VxWorks的应用的开发门槛,使得一般的开发人员不能够方便的接入到对其的应用研究当中,增加了开发难度。

\section{论文的主要内容和组织结构}