\chapter{绪论}
\section{课题背景以及意义}
	随着嵌入式系统的不断发展和应用,针对不同的嵌入式系统的软件开发工作也越来越庞大而复杂。在传统的PC机上进行软件设计和开发时我们有许多的工具可以用来进行程序的调试和分析,然而在嵌入式软件的开发当中,由于嵌入式设备所独有的特点,对嵌入式系统上的软件的调试、分析一直是一个费时费力的工作。一个好的调试、分析工具可以给嵌入式软件开发人员带来很大的帮助,使其达到事半功倍的效果,快速完成软件开发过程中的调试、分析,软件运行过程中调试信息的定位等工作。
		
	目前国内外已有几十种商业化嵌入式操作系统可以供选择,如VxWorks、uc/OS-II、Windows Embedded CE、RTLinux、和“女蜗Hopen”等。其中VxWorks以其良好的可靠性和优越的实时性被广泛地应用在通信、军事、航空、航天等高精尖技术及实时性要求极高的领域中\cite{张鹏2007基于}。通常在进行VxWorks应用程序的开发或者是将Linux下的应用程序移植到VxWorks中时都需要在程序中加入大量的调试信息,在程序的运行当中也需要输出一些调试信息,方便之后对程序运行过程中产生的问题进行具体的分析。在VxWorks的集成开发环境Tornado当中有一个驻留在主机端的命令行解释器WindSh和GDB命令行,但是这个工具是位于主机上的,需要通过通过网口或者串口控制命令传输到嵌入式设备上,然后再将输出信息传输到主机上显示。然而目前的大多设备都已不再将RS-232的串口作为必须的标准接口,而是大量的使用USB接口,因此在调试信息的传输过程中只能够使用USB接口。
	
	
	本次基于VxWorks的调试通道的设计来源于中船重工的实际需求,在实际的生产环境当中他们无法使用VxWorks集成开发环境当中提供的调试工具,我们基于VxWorks开发出一个小巧易用的调试信息的传输通道,底层的信息传输使用USB口转串口来实现,USB口转串口的转换器使用CP2102模块来实现,并在此基础上设计了给应用层使用的调试接口。
	
		
			
\section{国内外概况}
	嵌入式系统在近年发展迅猛,随着物联网的兴起,其广泛的应用于各种智能设备当中。而嵌入式实时操作系统作为嵌入式操作系统中定位更加精准的一类,其应用场景更加的专业化\cite{解月江2004VxWorks设备驱动技术研究}\cite{VxWorks嵌入式实时操作系统的结构研究}。实时系统对系统的可靠性、实时性等方面有严格要求。
	目前国内在使用的嵌入式实时操作系统有上百个,这些操作系统面向不同的专业领域,具有各自不同的特性。以下介绍几个具有代表性的操作系统:
\begin{itemize}
\item \textbf{uc/OS-II}
	
	由Micrium公司开发的一个开源实时操作系统内核,其具有可移植、可固化、可裁剪、抢占式多任务等优点\cite{Wu2008Implementation}\cite{}。该实时操作系统内核仅仅包含了任务调度、任务管理、时间管理、内存管理、任务间的通信和同步等基本的系统服务,没有提供输入输出管理、文件系统、网络等额外的服务\cite{马增炜2011基于}\cite{Zhang2010Design}。对于其余的功能需要用户根据自己的实际需求由用户来实现。
	
\item \textbf{Windows Embedded CE}

	由Microsoft开发出来的一个嵌入式实时操作系统。其内核提供内存管理、抢先多任务和中断处理功能。内核上面是图形用户界面GUI和桌面应用程序,在GUI的内部运行着所有的应用程序\cite{徐媛媛2003嵌入式实时操作系统的设备驱动}\cite{谢强2007基于}。Windows Embedded CE的内核具有并发处理能力,但是其并不具备真正的硬实时性能,该系统主要的应用于工业自动化、消费电子、以及医疗设备等。
	
\item \textbf{RTLinux}
	
	由美国墨西哥理工学院开发的开源嵌入式实时操作系统,RTLinux没有针对实时操作系统的特性重写Linux内核,而是在Linux内核和硬件之间增加一个虚拟层,构成一个小的、时间上可预测的、与Linux内核分开的实时内核,使得其中运行的实时进程满足实时性要求\cite{郭春生2002硬实时操作系统}。这样在不改动Linux内核情况下利用了其现有的丰富的软件资源\cite{Zhu2004RTLinux}。RTLinux因为其i开源特性,使用者可以根据自己的需要对其进行自由的修改,且较容易获得所需的资料和技术支持。
	
\item \textbf{Vxworks}

	由美国Wind River System公司开发的一个实时操作系统,具有抢占式调度、中断延迟小、系统内核可剪裁等特点\cite{李立志2003实时操作系统}\cite{陈洋2007VxWorks},它提供了一套集成开发环境Tornado,在开发环境 当中提供了丰富的调试工具和仿真环境。VxWorks支持广泛的网络通信协议、并能够根据用户的需求进行组合,其开放式的结构和工业标准的支持,使得开发者只需要做最少量的工作即可设计出有效的适合于不同的用户要求的系统\cite{谢强2007基于}\cite{徐媛媛2003嵌入式实时操作系统的设备驱动}。	
\end{itemize}	
		

	VxWorks作为一款强实时性、高可靠性的操作系统,在我国广泛的运用在军工、航空航天、通信等部门\cite{陈洋2007VxWorks}\cite{张鹏2007基于}。VxWorks的集成开发调试环境为Tornado,使用该开发环境可以帮助编程人员轻松的完成程序的编辑、编译、调试、系统配置等工作\cite{嵌入式实时操作系统VxWorks及其开发环境Tornado}\cite{Tronado}。Tornado的调试器包含有GDB命令行接口和WindSh工具,能够很好的进行应用程序的现场开发和调试。但是对于调试信息、日志信息的事后分析却没有提供解决办法且该工具要基于RS-232串口来使用,而现在大多数的设备都已不再配置RS-232串口,只配置有USB接口,而VxWorks上并没有实现好的USB口转串口的驱动程序,只实现了标准的USB协议栈和串口驱动,因此我们需要自己在VxWorks下实现一个USB口转串口驱动。
	
	对于USB口转串口的转换器,国内外通常都会采用两种方案:一种是以CY7C68013芯片为代表,自己从底层的固件开始,进行彻底而全面的系统开发,这种方案的成本和开发难度都很大,通常都不会使用这种方案。另外一个方案是采用类似于CP2102等专用的双向USB口转串口芯片来进行设计,这种方案简单实用,只需要对芯片的功能进行了解和应用即可,无需深入开发\cite{Yao2009Design}\cite{Zhou2002The}。因此我们在此会选择CP2102芯片来进行调试通道的设计。	
	
	

\section{论文的主要工作和组织结构}	
	主要工作:在嵌入式实时操作系统VxWorks上实现一个能够满足程序的调试信息输出的通道,主要包括两个部分:一个将USB总线技术和RS-232接口相结合,设计出一个满足特定要求的、实用的USB转串口驱动程序;另一个是设计给应用层调用的日志传输接口封装程序和标准输出重定向接口封装程序。\\
 本文共分为六章,各个章节的具体安排如下:
 
 第一章为绪论,主要介绍了本课题的研究背景和意义、国内外的发展状况以及本文的内容的安排。
 
 第二张介绍了进行调试通道的开发所需要了解的系统知识,主要包括VxWorks系统及驱动开发的知识、USB开发的相关知识,最后给出了一个调试通道的总体设计。
 
 第三章介绍了我们使用CP2102模块开发的相关知识和VxWorks下的USB口转串口驱动的具体实现,包括特定需求下的单设备驱动和多设备支持的驱动
 
 第四章主要介绍了应用层的接口封装部分,主要包括Log接口的设计,标准输出重定向接口的设计,以及PC客户端的协议解析部分。
 
 第五章主要内容是系统的功能测试部分。
 
 最后在结束语部分对整个的工作进行了总结,指出了本次的工作的不足之处,并对下一步的工作进行了展望。 

\