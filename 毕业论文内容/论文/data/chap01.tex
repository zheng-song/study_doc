\chapter{绪论}
\section{课题背景以及意义}
	当今,在嵌入式领域,嵌入式技术(Embedded Technology)已经成为了新的技术热点,随着嵌入式系统的不断发展和应用,针对不同的嵌入式软件的开发也越来越受到重视。在嵌入式软件的设计当中,由于嵌入式独有的特点,其调试、分析一直是一个费时费力的工作。一个好的调试器可以给嵌入式软件开发人员带来很大的帮助,使其达到事半功倍的效果,快速完成软件开发过程中的调试分析过程、软件运行过程中日志信息的定位等工作。
		
	目前国内外已有几十种商业化嵌入式操作系统可以供选择,如VxWorks、uc/OS-II、Windows Embedded CE、RTLinux、和“女蜗Hopen”等。其中VxWorks以其良好的可靠性和卓越的实时性被广泛地应用在通信、军事、航空、航天等高精尖技术及实时性要求极高的领域中,如卫星通讯、军事演习、弹道制导、飞机导航等。而Linux操作系统则是完全开源的,在全世界拥有几十万的开源项目,目前主流的Android及嵌入式设备都采用Linux操作系统。
	
	在我国VxWorks大量的应用于我国的军事、国防工业当中,通常在进行VxWorks应用程序的开发或者是将Linux下的应用程序移植到VxWorks中时都需要在程序中加入大量的调试信息,在程序的运行当中也需要输出一些日志信息,方便之后编程人员对程序运行过程中产生的问题进行具体的分析。VxWorks自身带有一个集成的测试开发调试环境Tronado,可以用它来完成程序的编辑、编译、调试、系统配置等工作。其带有的CrossWind调试器拥有一个驻留在主机端的命令行解释器WindSh和GDB命令行,但是由于中船重工实际使用中的限制,设备上并没有串口可供使用,而且设备并不具备进行现场调试的环境,他们希望能够在软件运行时直接将调试和日志信息输出保存之后进行一个事后分析的工作。
	
	因此,本论文基于中船重工的实际需求制作了一个基于VxWorks的调试通道。
	
			
\section{国内外概况}
	随着计算机技术的突飞猛进,实时系统无论是在技术上还是在应用领域当中都取得了辉煌的成就,尤其是在最近的二三十年当中,随着物联网的兴起,各种智能设备都需要安装嵌入式操作系统,导致嵌入式操作系统的发展愈加迅猛。而嵌入式实时操作系统属于嵌入式操作系统中定位更加精准的一类,其应用场景更加的专业化。
	
	目前国内在使用的RTOS有上百个,这些操作系统面向不同的专业领域,具有各自不同的特性。以下介绍几个具有代表性的操作系统:
\begin{itemize}
\item \textbf{uc/OS-II}
	
	是一个由Micrium公司提供的可移植、可固化、可裁剪、抢占式多任务实时内核,在这个内核之上提供了最基本的系统服务,该内核适用于多种微处理器、微控制器和数字处理芯片。该实时操作系统内核的特点是仅仅包含了任务调度、任务管理、时间管理、内存管理、任务间的通信和同步等基本功能,没有提供输入输出管理、文件系统、网络等额外的服务。但是由于uC/OS-II良好的可扩展性和源码开放性,这些功能完全可以由用户根据需要分别实现。
	
\item \textbf{Windows Embedded CE}

	由Microsoft开发出来的一个嵌入式实时操作系统。其内核提供内存管理、抢先多任务和中断处理功能。内核上面是图形用户界面GUI和桌面应用程序。在GUI的内部运行着所有的应用程序,其内核具有32000个处理器的并发处理能力,每个处理器有2GB虚拟内存寻址空间,同时还能够保持系统的实时响应\cite{WindowsEmbeddedCE6.0}。但其缺点是很难实现产品的定制,而且并不具备真正的实时性能,没有足够的多任务支持能力。其主要的应用场景为互联网协议机顶盒、全球定位系统、无线投影仪以及各种工业自动化、消费电子、以及医疗设备等。
	
\item \textbf{RTLinux}
	
	由美国墨西哥理工学院开发的嵌入式实时操作系统,其特殊之处在于开发者并没有针对实时操作系统的特性而重写Linux内核,而是将标准的Linux核心作为实时核心的一个进程,同用户的实时进程一起进行调度。这样对Linux内核的改动非常小,并充分利用了Linux下现有的丰富的软件资源。RTLinux的优点在于:与Linux一样,RTLinux是开放源码的操作系统,在网上较易获得所需的资料和技术支持,使用者可以根据自己的需要进行修改。其主要应用领域包括航天飞机的空间数据采集、科学仪器监控和电影特技图像处理等。
	
\item \textbf{Vxworks}

	由美国Wind River System公司推出的一个实时操作系统,并提供了一套实时操作系统开发环境Tornado,提供了丰富的调试、仿真环境和工具。VxWorks具有良好的持续发展能力、高性能的微内核以及友好的用户开发环境。它支持广泛的网络通信协议、并能够根据用户的需求进行组合,其开放式的结构和工业标准的支持,使得开发者只需要做最少量的工作即可设计出有效的适合于不同的用户要求的系统。因为VxWorks良好的可靠性和卓越的实时性,其广泛的被运用于通信、军事、航天等高精尖和实时性要求极高的领域当中。	
\end{itemize}	
		
	VxWorks作为一款强实时性、高可靠性的操作系统,在我国广泛的运用在军工、航空航天、通信等部门。VxWorks的集成开发调试环境为Tornado,使用该开发环境可以帮助编程人员轻松的完成程序的编辑、编译、调试、系统配置等工作。Tornado拥有一整套完整的面向嵌入式系统的开发和调试工具,包括C和C++远程级调试器、目标和工具管理、系统目标跟踪、内存使用分析和自动配置,所有工具都能够很方便的同时运行,很容易增加扩展和交互式开发。Tornado的调试器包含有GDB命令行接口和WindSh工具,能够很好的进行应用程序的现场开发和调试。但是对于调试信息、日志信息的事后分析却没有提供解决办法且该工具要基于RS-232串口来使用,而现在大多数的设备都已不再配置RS-232串口。
	
	对USB口转串口的设计通常可以采用两种方案,一种是以CY7C68013芯片为代表,自己从底层的固件开始,进行彻底而全面的系统开发,这种方案的成本和开发难度都很大,通常都不会使用这种方案。另外一个方案是采用类似于CP2102等专用的双向USB口转串口芯片来进行设计,这种方案简单实用,只需要对芯片的功能进行了解和应用即可,无需深入开发。因此我们在此会选择CP2102芯片来进行调试通道的设计。
	


\section{论文的主要内容和组织结构}	
	研究目标:在嵌入式实时操作系统VxWorks上实现一个能够满足程序的调试信息输出的通道,主要包括两个部分:一个满足特定要求的高速的、实用的USB转串口驱动程序,一个上层的日志传输接口封装程序和标准输出重定向接口封装程序。\\
 本文共分为六章,各个章节的具体安排如下:
 
 第一章为绪论,主要介绍了本课题的研究背景和意义、国内外的发展状况以及本文的内容的安排。
 
 第二张介绍了进行调试通道的开发所需要了解的系统知识,主要包括VxWorks系统、UBS、串口和VxWorks的开发工具Tornado。
 
 第三章介绍了VxWorks下的驱动程序的开发和USB口转串口驱动的具体实现,包括特定需求下的单设备驱动和多设备支持的驱动
 
 第四章主要介绍了应用层的接口封装部分,主要包括Log接口的设计,标准输出重定向接口的设计,以及PC客户端的协议解析部分。
 
 第五章主要内容是系统的功能测试部分。
 
 最后在结束语部分对整个的工作进行了总结,指出了本次的工作的不足之处,并对下一步的工作进行了展望。 
