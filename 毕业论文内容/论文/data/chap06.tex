\chapter{总结与展望}
\section{全文总结}
	USB是一种新兴的外围接口标准,由于它众多的优良特性,且在现代PC中广泛使用,已经逐渐替代了其他的接口标准。本文选择了使用在VxWorks下使用USB口转串口的技术为基础,来实现一个调试通道。USB口转串口的硬件实现选择了目前市场上通用的CP2102芯片,因为芯片有良好的说明文档,并有大量的社区支持,对于驱动程序的编写更加方便。基于这个芯片我们在VxWorks上实现了一个针对该芯片的USB口转串口驱动。
	
	在本次的研究和开发的过程中,掌握了大量的关于USB接口和串口的相关知识,让我深深的体会到了USB驱动分层设计的好处与便利。要开发一个USB协议栈需要有很强的理论基础,明白各个层次之间的关系,上层应用程序与设备驱动之间的通信原理等。本次的设计是基于VxWorks操作系统之下的,之前从未接触过这类强实时性的嵌入式系统。在本次的开发过程中不断学习VxWorks系统的同时,也积累了大量的VxWorks下开发项目的实际经验。


\section{展望}

	通过本次的 VxWorks系统作为一个高性能的实时操作系统,成功的应用于很多大型的高尖端项目当中。在本次的发开过程当中我也只是学习到了VxWorks的冰山一角,对于其系统理论和实际应用的掌握还不全面。同时感受到了USB功能的强大和开发难度。但是其优异的性能一定能够使得其在今后的外围串行通信接口当中占有更加重要的地位。随着PC和消费电子类产品、数字化产品的不断增多和普及,用户对于数据带宽、数据存储容量、数据传输速率和使用等会有越来越高的要求,鉴于此,USB技术也在不断地发展和完善,USB OTG、type C等标准的颁布,更能使得USB的应用领域和实用场景强化。因此,研究USB协议和开发一些特定功能的USB接口将会成为应用USB技术的关键。
	
	至此,本次设计的调试通道的功能已经基本实现,能够满足实际的应用中的需求。由于时间和能力有限,对于本次的调试通道的设计还有很多的不足之处,对于VxWorks下的USB口转串口的驱动程序部分还有很多的可以改进、完善的部分。例如在本系统中没有完成对设备的流控的设置,因为串口的传输速率有限,远远小于USB口的传输速率,使得串口速率成为了调试通道中传输速率的瓶颈部分,也许可以选择更好的数据传输方式来设计此通道。 
	
	因为时间的关系,未能将研究工作进行的更加深入,对于不足之处深表遗憾。此外,论文当中难免会有考虑不周之处,恳请老师、同学批评指正。
