\chapter{总结与展望}

	USB是一种新兴的外围接口标准,由于它众多的优良特性,且在现代PC中广泛使用,已经逐渐替代了其他的接口标准。本文选择了使用在VxWorks下使用USB口转串口的技术为基础,来实现一个调试通道。USB口转串口的硬件实现选择了目前市场上通用的CP2102芯片,因为芯片有良好的说明文档,并有大量的社区支持,对于驱动程序的编写更加方便。基于这个芯片我们在VxWorks上实现了一个针对该芯片的USB口转串口驱动。
	
	在本次的研究和开发的过程中,掌握了大量的关于USB接口和串口的相关知识,让我深深的体会到了USB驱动分层设计的好处与便利。要开发一个USB协议栈需要有很强的理论基础,明白各个层次之间的关系,上层应用程序与设备驱动之间的通信原理等。本次的设计是基于VxWorks操作系统之下的,之前从未接触过这类强实时性的嵌入式系统。在本次的开发过程中不断学习VxWorks系统的同时,也积累了大量的VxWorks下开发项目的实际经验。

	通过本次的项目我了解到了VxWorks的强大之处,其作为一个高性能的实时操作系统,成功的应用于很多大型的高尖端项目当中。在本次的发开过程当中我也只是学习到了VxWorks的冰山一角,对于其系统理论和实际应用的掌握还不全面。同时感受到了USB功能的强大和开发难度之大。USB凭借其优异的性能一定会成为更多外设的通用标准,同时	
用户也会对USB的带宽、据传输速率提出越来越高的要求,因此以后的USB技术肯定会更加的强大和完善,随着USB OTG、type C等标准的颁布,USB的应用领域和实用场景进一步得到强化。
	
	至此,本次设计的调试通道的功能已经基本实现,能够满足实际的应用中的需求。由于时间和能力有限,对于本次的调试通道的设计还有很多的不足之处,对于VxWorks下的USB口转串口的驱动程序部分还有很多的可以改进、完善的部分。例如在本系统中没有完成对设备的流控的设置,因为串口的传输速率有限,远远小于USB口的传输速率,使得串口速率成为了调试通道中传输速率的瓶颈部分,也许可以选择更好的数据传输方式来设计此通道。 
	