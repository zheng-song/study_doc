\chapter{实时操作系统VxWorks}

\section{概述}

\subsection{实时操作系统}
	实时操作系统(Real Time Operation System,简称RTOS)是整个实时系统的核心。POSIX1003.1标准为RTOS下了一个简单的定义:RTOS是能够在有限的响应时间内为应用提供所要求级别服务的操作系统\cite{Renard20081003}。当外界事件或者是数据产生时,RTOS需要快速的进行处理,并且处理的结果又能够在规定的时间之内来控制生产过程或者对处理系统做出快速的响应,调度一切可利用的资源来完成实时任务。实时系统按照实时的效果可以分为软实时和硬实时,硬实时要求在规定的时间内必须完成操作,这是通过操作的在设计的时候就得到保证的;软实时只需要按照任务的优先级,尽可能快的完成任务即可。

一个实时操作系统的特征通常包括以下几点:
\begin{itemize}
\item \textbf{高精度计时系统} 

	计时精度是影响实时性的一个重要因素,在实时系统当中,经常需要精确确定实时地操作某个设备或执行某个任务,或精确的计算一个时间函数。这些不仅仅依赖于一些硬件提供的时钟精度,也依赖于实时操作系统的高精度计时功能。
\item \textbf{多级中断机制}

	中断是实时操作系统当中的一个关键设施,是用于通知系统发生外部事件的常用机制。一个实时操作系统通常需要处理多种外部信息或事件,但处理的紧迫程度有轻重缓急之分。有的必须立即作出反应,有的则可以延后处理。因此,需要建立多级中断嵌套处理机制,以确保对紧迫程度较高的实时事件进行及时响应和处理。
\item \textbf{实时调度机制} 

	实时操作系统不仅要及时响应实时事件中断,同时也要及时调度运行实时任务。但是,处理机调度并不能随心所欲的进行,因为涉及到两个进程之间的切换,只能在确保“安全切换”的时间点上进行,实时调度机制包括两个方面,一是在调度策略和算法上保证优先调度实时任务;二是建立更多“安全切换”时间点,保证及时调度实时任务。
\end{itemize}

	内核作为操作系统的核心,负责控制这计算机上的所有硬件和软件资源,在必要的时候给应用程序分配硬件资源,并执行相应的操作命令。内核的主要功能为以下四个:
\begin{itemize}
\item 系统内存管理
\item 软件程序管理
\item 硬件设备管理
\item 文件和网络系统管理
\end{itemize}

本次所需完成的调试通道正是基于一个目前业界有名的实时操作系统VxWorks。

\subsection{VxWorks简介}
	VxWorks操作系统是美国Wind River System公司于1983年推出的一个运行在目标机上的高性能、可裁剪的实时操作系统(RTOS),该系统专门为嵌入式实时系统领域而设计开发,其具有良好的持续发展能力、高性能的内核以及友好的用户开发环境,为开发人员提供了高效的实时多任务调度、中断管理、实时的系统资源以及实时的任务间通信。并且拥有多达1800个功能强大的应用程序接口\cite{嵌入式实时操作系统VxWorks及其开发环境Tornado}。VxWorks的系统结构如\autoref{fig:VxWorks系统结构}所示
\begin{figure}[!h]
\centering
\includegraphics[width=.7\textwidth]{./graphics/VxWorks-sys-structure.pdf}
\caption{VxWorks系统结构}\label{fig:VxWorks系统结构}
\end{figure}
	
	VxWorks采用微内核设计,支持多种硬件环境,包括X86、PowerPC、ARM等众多主流的处理器,同时还支持RISC、DSP技术,且在各种CPU平台上提供了统一的编程接口和一致的运行环境。VxWorks凭借着其优异的性能在军事、航空航天、工业控制、通信等高精尖以及实时性要求极高的领域当中,有着更加广泛而深入的应用。应用实例包括火星探测器、爱国者导弹、飞机导航、F-16、FA-18战斗机等\cite{嵌入式实时操作系统VxWorks及其开发环境Tornado}。自从对我国的销售解禁之后,VxWorks也大量的应用于我国的军事、国防工业当中。
	
	VxWorks操作系统由400多个相对独立的短小精炼的目标模块组成,用户可以根据自己的实际需求选择适当的模板来裁剪和配置系统,这样可以有效的保证系统的安全性和可靠性。系统的连接器可以按照应用的需要自动链接一些目标模块。这样,通过目标模块之间的按需组合,可以得到许多满足功能需求的应用。此外,VxWorks支持广泛的工业标准,如POSIX1003.1b实时扩展、ANSIC(浮点支持)和TCP/IP网络协议等。这种广泛的协议支持在主机和目标机之间提供了无缝的工作环境,任务可以通过网络箱其他系统的主机存取文件,即远程文件存取,也支持远程的过程调用。这些标准也促进了多种不同产品之间的互通性,提升了可移植性。由于其高度的灵活性,用户能够很容易的对该操作系统进行重新定制或作适当的开发,以满足自己的实际应用。	
\subsection{VxWorks原理}
	VxWorks操作系统以实时著称,系统代码组件化比较好,系统精简,产品的IMAGE可以做的非常小。VxWorks属于共享内存的操作系统,没有进程的概念,按照任务进行调度,每个任务是分配了堆栈,可以访问系统的所有内存空间,比较高效,但是也比较危险,很容易就会发生系统崩溃的情况。但是也就是因为这个特性,因此基于VxWorks开发的产品一般是单个应用产品,这个打包完可以以一个进程体现在Linux里面。 
	
	VxWorks系统主要如下几个组件,一般的应用程序也就用到如下几个模块:任务调度负责任务优先级,任务创建,任务删除等功能。任务通讯主要涉及队列管理,管道管理等。IO管理就是通用外设的输入输出管理。文件管理包含文件或者链接创建,修改,删除,检索等等。内存管理包含内存申请及内存释放。定时器管理就是定时器创建,定时器删除。网络通信包含套接字创建,udp/tcp通信,套接字删除,及配置管理。同步就是任务同步。互斥就是任务互斥,保护关键资源。
	
	在VxWorks操作系统的代码架构里面,一般写一个应用程序只需要涉及上面几个系统组件,由于VxWorks操作系统组件化非常好,这几个组件的耦合度非常低,每个组件对外提供都是单独的头文件,比如任务调度,其头文件为taskLib.h,任务通讯如果用的队列,那其头文件就是msgQLib.h,如果是定时器管理,那其头文件就是timerLib.h,因此也让程序移植提供了很大的便利性。有很多人认为,VxWorks跟Linux操作性的系统头文件差异化太大,因此移植难度成倍速增加,其实不然,就是由于VxWorks的高度组件化,让程序移植提供了很大的便利性。
	
	

\section{微内核wind}
	现代嵌入式操作系统的一个发展趋势是尽可能的从操作系统当中去掉冗余的服务,保证内核的高效、精简,只留下一个很小的内核,由用户进程来实现大多数操作系统的功能如文件服务、进程服务、存储服务等,即所谓的微内核操作系统,从而使得系统具有模块化、结构清晰、可移植性强和可裁剪性好的特点。在微内核操作系统当中,设备驱动程序通常作为核外可裁剪的部分存在。对于操作系统内核来说,核外驱动就相当于一个普通的应用程序,因此对于系统内核的影响很小,编写和调试都很方便,同时可以避免数据流在不同级间的拷贝。微内核的设计思想首先在CMU开发的MACH操作系统当中得到了成功的应用,并使得微内核结构成为操作系统研究的热点\cite{Black1992Microkernel}
	
	VxWorks的微内核wind提供的功能包括:任务管理、事件和异步信号服务、信号量服务、消息队列服务、内存管理、中断服务程序、时钟管理和定时服务、看门狗、异常处理等。微内核的设计减少了系统的开销,从而保证了对外部事件的快速、确定的反应。
\subsection{wind的多任务机制}
	现代实时系统基于多任务和任务间通信的互补概念。多任务环境允许将实时应用程序构建为一组独立任务,每个任务都有自己的执行线程和一组系统资源。任务间通信设施允许这些任务同步并进行通信以协调其活动。在VxWorks中,任务间的通信工具包括快速信号量、消息队列、管道、套接字。
	
	VxWorks下的任务即通用操作系统下所述的进程,是内核的基本运行单位,wind提供的多任务环境允许实时应用程序以一套独立任务的方式构筑,每一个任务拥有独立的执行线程和一套自己的系统资源。进程间的通信机制使得这些任务的行为同步、协调。内一个开启的任务都有一个任务控制的数据结构来记录当前的任务的状态,供内核管理调度,这个控制块简记为TCB。控制块里包含了当前的状态、优先级、要等待的事件或资源、任务程序码的起始地址、初始堆栈指针等任务的上下文。
	
	wind使用中断驱动和优先级的方式。它缩短了上下文转换的时间开销和中断的时间延迟。VxWorks中的任何例程都可以被启动为一个单独的任务,拥有它自己的上下文和堆栈,还有一些其他的任务机制可以使得任务挂起、继续、删除、延时或改变优先级。	

	VxWorks中的内存被统一编址资源共享十分方便;任务能够快速的共享系统的绝大部分资源,同时有自己独立的上下文。
	
	VxWorks的任务是VxWorks的最小的运行单元,每一个任务之间是共享内存的,可以互相访问,这个有点类似于Linux的线程的概念,而几个任务合起来的嵌入式产品就等同于一个进程。

\subsection{wind的任务调度}

	操作系统的进程调度时机基本上都会选择在同一个地方:从内核状态退出之时,VxWorks也一样。从内核状态退出的时机主要有两个渠道:系统调用和中断。	VxWorks与于通用操作系统的区别的关键点在于其必须能够迅速的响应外部中断,所以实时操作系统只是在响应速度上更快,而在响应时机上与通用操作系统并无区别。Vxworks区别于通用操作系统的另一个显著的特点是其划分运行级别,应用层程序可
以直接调用内核函数,而不需要通过软中断的方式从应用层进入到内核层。故 Vxworks 系统调用的概念与通用操作系统(如 Linux)有本质上的差别。
	
	
	VxWorks中应用软件的最小单位是任务,同时也是竞争资源的最小单位,每一个任务是程序的一个实例,程序可以由多个或单个函数构成,任务ID是操作系统管理时使用的唯一标志;任务被分为256个优先级,0最高,255最低;任务在被创建时就会被指定一个优先级,在运行时也可以通过taskPrioritySet()函数重新设定优先级。该函数的原型如下:
\lstset{language=C}
\begin{lstlisting}
STATUS taskPrioritySet
(
  int tid, /* task ID */
  int newPriority /* new priority */
) 
\end{lstlisting}
	
	
	
	在VxWorks中任务调度是基于优先级的,而且是可抢占式的调度方式,这样才能够区分实际情况下的不同状态的处理级别,对高优先级的情况进行优先响应。	VxWorks下对于应用层任务,推荐使用100-250之间的优先级,驱动层任务可以使用51-99之间的优先级。
	VxWorks中任务的状态通常有四种,这四种状态可以通过相应的函数控制其相互转化:
\begin{itemize}
\item \textbf{就绪态:}任务正在等待CPU资源,该状态下以优先级为序排列任务。
\item \textbf{休眠态:}任务正在等待除CPU资源之外的其他资源,当竞争使用某个资源,而资源当前不可得,就进入这个状态。
\item \textbf{延迟态:}任务正在等待一定时间的延时。当调用taskDelay 函数让任务延迟一段固定的时间时,任务所处的状态,此时任务无资格竞争使用CPU。
\item \textbf{悬置态:}任务无法执行,主要是用于调试一种状态,这种状态仅影响任务的执行而不影响任务的转换。
\end{itemize}

\begin{figure}[!h]
\centering
\includegraphics[width=.9\textwidth]{./graphics/vxworks-task-shift-diagram.pdf}
\caption{任务状态转换图}\label{fig:VxWorks状态转换图}
\end{figure}

	下面的这张内核管理的\autoref{fig:wind任务调度}展示了内核中构建了四种状态队列,通过TCB中任务状态的变化,内核对其进行调度,并安保就绪的任务加入到就绪队列等候CPU资源。

\begin{figure}[!h]
\centering
\includegraphics[width=.9\textwidth]{./graphics/vxworks-task-scheduling-diagram.pdf}
\caption{wind任务调度}\label{fig:wind任务调度}
\end{figure}


\subsection{wind任务间通信}
	VxWorks的内核提供了三种任务间通信机制(不包括信号机制):信号量,消息队列,管道。这三种机制无一例外的都是在使用共享物理内存机制,只不过这块共享的内存由内核在进行管理,任务无法直接进行访问,而必须通过内核提供的接口函数进行访问,这就提供了一种保护和管理机制,使得任务间通信安全有序的进行。

\begin{enumerate}
	\item \textbf{信号量}
	
	信号量的主要用途是互斥和同步。互斥主要保护资源,即某个时刻只允许只有一个任务在使用该资源。同步则是任务间协同完成某一项共同工作的机制,典型的例子就是生产者和消费者进程,消费者平时处于等待状态,等待生产者完成其资源的生成,而一旦资源产生完毕,此时生产者就会触发同步信号量,让消费者的任务启动(即唤醒)其处理资源的工作。
	
	基于各种资源的不同使用方式,VxWorks信号量机制具体的提供了三种信号量:通用信号量;互斥信号量;资源计数信号量。通用信号量既可用于同步也可用于资源计数,此时资源数通常为 1(当资源数为 1 时,也可以称之为互斥)。互斥信号量针对在使用过程中一些具体问题(如优先级反转)做了优化,更好的服务于任务间互斥需求;资源计数信号量用于资源数较多,同时可供多个任务使用的场合。
	\item \textbf{消息队列}
	
	消息队列内核实现上实际是一个结构数组,数组大小和数组中元素的容量在创建消息队列时被确定。消息队列是 Vxworks 内核提供的任务间传递较多信息的一种机制,不过这种机制存在的很大的局限性,即每个消息的最大长度是固定的。Vxworks内核提供的消息机制在创建消息队列时就必须指定单个消息的最大长度以及消息的数量,在消息队列成功创建后,这些参数都是固定不变的。
	
	\item \textbf{管道}
	
	管道相比消息队列提供了一种更为流畅的任务间信息传递机制。消息队列对于每个消息的大
小存在限制,而且必须将信息分批打包,而管道可以像文件那样进行读写,是一种流式消息
机制。管道在底层实现上是一种更为直接的共享物理内存机制。信号量和消息
队列还需要对传递的数据进行某种方式的封装,则管道不会传递信息做任何包装,直接分配
一块连续的内存空间作为任务间信息交互的中转站。传统意义上,管道分为两种:命名管道
和非命名管道。通常任务间通信使用的一般都是命名管道。非命名管道使用在线程意义上,
如父子进程,进程关系密切,且某些变量存在继承关系,如文件描述符,一般无法使用在两
个执行路线完全不同的任务之间。
\end{enumerate}		

\textbf{任务间特殊的通信机制--信号:} 信号不是信号量,二者不是一个概念。信号量是一种任务间互斥和同步机制,而信号则用于通知一个任务某个事件的发生。一个信号产生后,对应任务暂停当前执行流程,转而去执行一个特定的函数进行处理。信号机制有些类似于中断,也可以将信号看作是一种用户层提供的软件中断机制(区别于 CPU 本身提供的软件中断指令)。这种用户层软件中断机制相比硬件中断和中断指令而言具有较长的延迟时间,即从信号产生到信号的处理之间大多将经历较长的延迟。Vxworks 下信号处理有些特别,当一个任务接收到一个信号时,在这个任务下一次被调度运行之时进行信号的处理(即调用相关信号处理函数)。事实上,由于 Vxworks 在退出内核函数时都会进行任务调度,故一个任务,无论是否是当前执行任务,都将在被调度运行时执行信号的处理。

	通用操作系统上对于某些信号将不允许用户修改其默认处理函数,如 SIGKILL,SIGSTOP,然而 Vxworks 操作系统中可以对任何信号的处理函数都可以进行更换。
	
	运行环境也提供了有效的任务间通信机制,允许独立的任务在实时系统中与其行动相协调。
开发者在开发应用程序时可以使用多种方法:用于简单数据共享的共享内存、用于单 CPU异常事件的信号等。为了控制关键的系统资源,提供了三种信号灯:二进制、计数、有优先级继承特性的互斥信号灯。

\subsection{I/O系统}
	通常为了实现与应用程序的平台无关性,操作系统都会为应用程序提供一套标准的接口,VxWorks也不例外,这样就可以通过调整底层驱动或者是接近驱动那部分的操作系统中间层来提高应用层开发的效率,避免重复编码。在我们的通用操作系统(如Mac OS、Linux、Windows)当中,通常会将这套应用层的接口标准从操作系统中独立出来,专门以标准库的形式存在,这样可以屏蔽操作系统之间的差异,增强应用程序的平台无关性。
	
	VxWorks中也对应用层提供了一套标准的文件操作接口,实际上与GPOS提供接口类似,我们将其称作为标准I/O 库,VxWorks下由ioLib.c文件提供。ioLib.c文件提供如下标准接口函数:creat()、open()、unlink()、remove()、close()、rename()、read()、write()、ioctl()、lseek()、readv()、writev()等\cite{BSP开发人员指南},VxWorks与通用操作系统有很大的一个不同点是:VxWorks不区分用户态和内核态,用户层可以直接对内核函数进行调用,而无需使用陷阱指令之类的机制,以及存在使用权限上的限制。因此VxWorks提供给应用层的接口无需通过外围库的方式,而是直接以内核文件的形式提供。对于一般的设备而言,remove()接口是不需要实现的,\autoref{tab:IO接口函数表} 是七个驱动程序接口函数的简单描述。

\begin{table}[!h]
\centering
\begin{tabular}{|c|c|c|}
\hline
{\hei{接口名称}} & {\hei{函数原型}} & {\hei{描述}}\\ 
\hline
{open} & {open(filename,flags,mode)} & {打开文件}\\
\hline
{create} & {create(filename,flags)} & {创建文件}\\
\hline
{read} & {read(fd,\& buf,nBytes)} & {从文件中读取}\\
\hline 
{write} & {write(fd,\& buf,nBytes)} & {向文件中写入}\\ 
\hline
{ioctl} & {ioctl(fd,command,arg)} & {其他控制命令}\\
\hline
{close} & {close(fd)} & {关闭文件}\\
\hline
{remove} & {remove(filename)} & {移除文件}\\
\hline
\end{tabular}
\caption{IO接口函数表}\label{tab:IO接口函数表}
\end{table}


\section{集成开发环境Tornado}
	Tornado是嵌入式实时领域里最新一代的开发调试环境,提供了高效明晰的图形化的实时应用开发平台,它包括一整套完整的面向嵌入式系统的开发和调试工具。Tornado采用的是主机—目标机的交叉开发模型,应用程序在主机的Windows环境下编译链接生成可执行文件,下载到目标机,通过主机上的目标服务器与目标机上的代理程序的通信完成对应用程序的调测、分析。这些工具包括C和C++远程级调试器、目标和工具管理、系统目标跟踪、内存使用分析和自动配置,所有工具都能够很方便的同时运行,很容易增加扩展和交互式开发。

	典型的主机开发系统通常有比较大的RAM、硬盘空间、打印机以及其他外部设备,而典型的目标机系统只有仅能满足实时应用的资源,除此以外可能还有比较少量的用于测试和调试的额外资源。Tornado开发环境的基本优点是应用模块不需要链接到运行系统库。Tornado直接装载重定位的目标模块,通过每个模块里的符号表来动态解析外部符号的引用,解析符号表是由运行在目标机上的目标服务器来完成的。Tornado在开发过程中会把目标模块的大小减到最小,这样可以缩短开发周期。主机端驻留的shell和调试器也能够调用和测试独立的应用程序或者是完整的任务。

	Tornado是为开发VxWorks应用程序提供的集成开发环境,其中包含有工程管理软件,可以将用户自己的代码和VxWorks的核心有效的结合起来,可以按照用户的需要裁剪配置VxWorks内核;
Tornado开发系统包含有三个高度集成的部分:
\begin{itemize}
\item \hei{运行在宿主机和目标机上的强有力的交叉开发工具和实用程序;}
\item \hei{运行在目标机上的高性能、可裁剪的实时操作系统VxWorks;}
\item \hei{连接宿主机和目标机的多种通信方式,如以太网,串口线,ICE或ROM仿真器等。} 
\end{itemize}
板级支持包BSP(Board Support Package)作为VxWorks系统的主要组成部分,对各种板子的硬件功能提供了同一的软件接口,它包括硬件初始化、中断的产生和处理、硬件时钟和计时器管理、局域和总线内存地址映射、内存分配等。每一个板级支持包包括一个ROM启动(Boot ROM)或其他启动机制。
Tornado主机集成开发环境中的主要工具为:
\begin{itemize}
\item 编辑器:源码编辑
\item 工程管理器
\item 编译器
\item WindSh:一个驻留在主机端的命令行解释器,提供从主机端控制所有运行系统的接口
\item CrossWind调试器:一个远程源码级的调试器,该调试器控制窗口综合了GDB命令行接口和WindSh工具。
\item 浏览器:是一个系统对象的查看器,可以监视目标机的状态。
\item WindView逻辑分析器:一个动态可视化工具,可以提供上下文切换的情况,事件和有关测量对象的信息。
\item VxSim仿真器:用来模拟目标操作系统。
\end{itemize}


\section{USB简介}
	USB(Universial Serial Bus,通用串行总线)是这十几年来应用在PC领域的最新型的接口技术,出现的契机是为了为了解决日益增加的PC外设与有限的主板插槽之间的矛盾,其实现是由一些PC大厂商(Microsoft、Intel等)定制出来的,自从1995年在Comdex上展出以来至今已广泛地被各个PC厂家所支持。目前已经在各类外部设备中都广泛的采用USB接口。USB接口标准目前有三种:USB1.1,USB2.0和USB3.0。USB接口应用如此的广泛是由其独特和实用的特性决定的。USB的体系结构如图\autoref{fig:USB体系结构}所示
\begin{figure}[!h]
\centering
\includegraphics[width=1.0\textwidth]{./graphics/usb-structure.pdf}
\caption{USB总线拓扑结构}\label{fig:USB体系结构}
\end{figure}

	USB的物理层拓扑为星型结构,包括三个部分:USB主机(Host)、USB集线器(Hub)、USB设备(Device)。
	
\begin{enumerate}
\item \textbf{USB主机}

	USB的体系结构只允许系统中有一个主机,主计算机系统的USB接口称之为USB主控制器。主控制器可以是硬件、固件或软件的联合体,其控制着总线上所有USB设备和所有集线器的数据通信过程。所有的数据传输都是由USB的主机端发起的,而且如果USB主机嵌入在一个计算机系统中,在数据的传输过程中也不需要计算机的CPU参与传输工作。USB主机通常具有以下的功能:
	\begin{itemize}
	\item 检测USB设备的插拔动作(通过根集线器来实现)
	\item 管理 USB 主机和USB设备之间的控制流
	\item 管理USB主机和USB设备之间的数据流
	\item 收集USB主机的状态和USB设备的动作信息
	\end{itemize}
\item \textbf{USB集线器}
	
	根集线器是集成在主机系统当中的一个特殊集线器,他可以提供一个或者跟多的接入口。在即插即用的USB体系结构当中,集线器是一种很重要的设备。其极大的简化了USB的复杂性,而且以很低的价格和易用性提供了设备的健壮性,一个集线器通常会包括两个部分:集线控制器和集线放大器。集线放大器是一种在上游端口和下游端口之间的协议控制开关。集线控制器提供接口寄存器用于和主机之间的通信,允许主机对其特定状态和控制命令进行设置,并监视和控制其端口。一种市场常见的集线器如\autoref{fig:USB集线器}所示
\begin{figure}[!ht]
\centering
\includegraphics[width=.5\textwidth]{./graphics/usb-hub.pdf}
\caption{USB集线器}\label{fig:USB集线器}
\end{figure}

每个集线器的下游端口都可以连接另外的集线器或功能部件,最大的连接能力是127,集线器可以检测每一个下游端口的设备的安装或移除,并可以对下游端口的设备分配资源,每一个下有端口都有独立的能力,不论是高速或者是低俗设备均可连接。集线器可以将低俗和高速端口的信号分开。

\item \textbf{USB设备}

	USB设备是USB总线系统的重要组成部分,它们以从属的方式与USB主机进行通信,并受USB主机的控制。USB主机端提供的协议软件通过和USB设备通信获得设备的信息,并给设备提供驱动程序,相比USB主机而言,USB设备只能够被动的应答,按照USB主机的要求接受或者发送数据。USB设备通过以下的属性来完成主机的要求:
	\begin{itemize}
	\item \textbf{描述符}\\
	USB协议为USB设备定义了一套描述设备功能和属性的固定结构的描述符,通过这些描述符向USB主机汇报设备的各种属性,主机通过对这些描述符的访问对设备进行识别、配置并为其提供相应的客户端驱动程序。运行于USB协议栈上层的客户端驱动程序通过这些信息正确的访问设备并与其进行通信,以实现即插即用的目的。
	\item \textbf{类}\\
	USB协议支持许多的外围设备,为了正确的驱动这些设备,USB主机端要为这些设备提供符合USB协议的驱动程序,称为客户端驱动。同时为了避免客户端程序过多,协议通过归纳将设备划分为不同的设备类,把功能相近的设备归为一类,主机端只需要提供类驱动程序便可以驱动大多数的USB设备。	
	\item \textbf{功能(Function)/接口(Interface)}\\
	在USB协议中,Function被定义为具有某种能力的设备,即相当于传统的单一功能设备。随着USB设备应用的发展,物理上几种不同的Function可以组成一台设备,只要设备的接口具有某种独立的能力,就称为一个Function。Function是从功能角度来说的,从设备的角度来说,它又被称为Interface。
	\item \textbf{端点(Endpoint)}\\
	端点层的各个子模块是USB设备与USB主机逻辑上的数据传输的最基本单元,即最基本的通信点,因而端点层的每一个逻辑模块被称为端点(Endpoint)。每一个端点都关联一个相应的端点号和数据传输方向(IN/OUT)。具有相同端点号和不同传输方向的通信点表示不同的端点,端点0被USB规范保留用作设备枚举和配置过程中的数据传输端点,与端点0对应的管道是默认管道,设备的所有端点共享端点0。在一个USB系统当中,USB主机对特定设备功能的访问是通过使用不同的端点号,否则,USB主机将不能对相应的设备进行正确的访问。由于在系统运行时,不同的设备配置有着互斥性,因而在不同的配置描述中,端点号是可以重复的。
	\item \textbf{管道}\\
	设备端点与USB主机所形成的具有特定数据传输特性(如数据传输格式、传输带宽、传输方向等)的数据通道,被称为管道。管道的物理介质就是USB系统中的数据线。端点层的USB设备与USB主机之间的数据传输都是基于管道进行的。
	\item \textbf{设备地址}\\
	USB主机的客户端驱动程序通过描述符来区分不同的设备,而USB主机控制器通过设备地址来区分设备。设备地址共有7位,表示理论上系统可以同时连接127个USB设备,但是在实际中,由于USB总线带宽的限制,这么多设备不可能同时的工作。USB主机负责为USB系统中的设备分配不同的设备地址,用来表示哪个设备的同时还要指名设备的端点号,表示使用哪个管道。
	\end{itemize}	
	
\end{enumerate}



\subsection{USB2.0总线结构和机制}
	USB的主要有优点有:
\begin{enumerate}
\item 使用方便,支持热插拔:连接外设不必再打开主机箱,而且方便携带,允许在任何时候USB外设的热插拔,而且不必关闭主机的电源;USB外设没有需要用户选择的设置;例如端口地址和中断请求线;自动配置外设,当用户连接USB外设到一个正在运行的系统时,系统会自动的检测外设,载入合适的驱动软件(如果有的话),并将其初始化;以上的特点使得USB外设符合便携易用的潮流。
\item 速度快:目前设备上使用大多数是USB 2.0的接口,最新生产的设备都配备了USB3.0的接口。USB2.0接口的理论速度可以达到480Mbps(即60MB/s),并且可以向下兼容USB1.1。USB3.0接口的理论速度可以达到5.0Gbps(即640MB/s),并提供USB2.0的兼容。
\item 连接灵活:一个USB主控制器理论上可以连接127个设备
\end{enumerate}

	一个主机和设备的建档连接需要一些列的层次和实体之间的交互,USB接口层在主机和设备间提供物理、信号包的关联,USB设备层表示USB系统程序实现对一个设备进行的总的USB操作,功能层适当匹配的客户服务程序层提供附加的功能给主机。设备和功能层当中各自有逻辑通信,但是实际的USB数据传输是通过USB总线接口层来实现的\cite{USB开发手册}\cite{圈圈教你玩USB}。
	
	USB2.0规范规定了USB传输的四种方式,每种方式有各自的用途\cite{USB总线接口开发指南}:
\begin{itemize}
\item \hei{控制传输:}

	控制传输是每一个设备必须要具有的传输方式,也是四种传输方式中过程最复杂的部分,他使得主机能够从设备列出的范围中读取和选择配置和其他设置,也能够发送自定义请求来为任何目的而发送和接收数据,在配置和列举的过程中起到重要的作用。
\item \hei{批量传输:}
	
	批量传输是为了处理传输速率不是很关键的情况,一般用于打印机和扫描仪。
\item \hei{中断传输:}

	中断传输是为了那些要快速实现主机和设备的交互而准备的,比如适用于鼠标和键盘。
\item \hei{等时传输:}
	
	等时传输用于必须要按照一个常数传输数据的情况,比如一个需要被实时播放的视频/音频数据流。
\end{itemize}
所有的传输都是由事物组成,事物又由包组成,而包包含一个包识别器(PID)、CRC和其他的信息,这些层次关系在USB2.0中有详细介绍。


\subsection{VxWorks上的USB驱动程序结构}




\section{串口通信}
	在生产实践和科学研究当中经常需要在上位机和下位机中进行数据和控制的传输,这种数据、控制传输都需要通过串口来进行,而现在生产的设备中大多都不再配置串口,仅仅保留了USB接口,于是需要通过一些手段将USB转换为串口,这时就出现了一些专用的USB转UART芯片,CP2102就是这样一种芯片。CP2102与其他同类型的芯片相比具有功耗更低、体积更小、集成度更高(仅需少量外部元件)、价格更低等优点。因此我们此次选择这个芯片作为调试通道的设计当中使用的芯片。

	计算机与计算机之间或者是计算机与终端设备之间的通信方式有串口通信,并行通信以及网络通信等。使用串行通信方式的优点是成本低,使用线路少,而且可以解决由于不同的线路特性造成的问题,因此在远距离传输方面也是一种不错的选择。对于特性相异的设备要使用串行通信来连接时,必须保证双方所使用的的标准接口是一致的。一般的串口标准有RS-232、RS-485等

\subsubsection{RS232与TTL}
	我们通常见到的串口有两种的物理标准,D型9针(对应于RS-232标准)插头和4针(对应于TTL标准)的杜邦头,4针的通常也会有第五根引脚--3.3V电源引脚。RS232电平标准用正负电压来表示逻辑状态,在RS-232标准中正电压是0,负电压是1。TTL电平标准使用高低电平来表示逻辑状态,TTL标准中低电平是0,高电平是1。使用TTL标准进行连接时,一般只会使用GND、RXD、TXD引脚,不会使用VCC或3.3V的电源线,避免与目标设备上的供电冲突。
	PL2303、CP2102芯片是USB转TTL串口的芯片,用USB来扩展串口(TTL)电平
	
 
\subsection{CP2102简介}
	CP2102是SILICON LABORATORIES推出的USB与RS232接口转换芯片,是一种高度集成的USB-UART桥接器,提供一个使用最小化的元件和PCB空间实现RS232转USB的简便的解决方案。CP2102芯片包含有一个USB2.0全速功能控制器,EEPROM,USB收发器,振荡器和带有全部的调制解调器控制信号的异步串行数据总线(UART),CP2102将全部的部件集成在一个5mm*5mm MLP-28封装的IC当中\cite{CP2102},cp2102的电路框图如\autoref{fig:cp2102电路框图}所示。

\begin{figure}[!h]
\centering
\includegraphics[width=1.0\textwidth]{./graphics/cp2102-circuit-diagram.pdf}
\caption{cp2102电路框图}\label{fig:cp2102电路框图}
\end{figure}

	CP2102是一个USB/RS232双向转换芯片,一方面可以从主机接收USB数据并将其转换为RS232信息格式流发送给外设,另外一方面可以从RS232外设接收数据转换为USB数据格式传送回主机,这些工作都会由芯片自动完成。使用时我们只需要将数据通过USB的数据包发送给CP2102芯片即可,芯片会自动进行解析和控制。
	
	在CP2102的内部有内置的和计算机进行通信的USB协议栈,在设备驱动的支持下,PC会将该设备识别为一个虚拟的串口。此时用户可以按照通用串口的方式来使用这个虚拟的串口。	
\begin{enumerate}
\item CP2102的USB功能控制器和收发器:CP2102的USB功能控制器是一个符合USB2.0协议的全速器件,这个器件负责管理USB和UART之间的所有数据传输以及由USB主控制器发出的命令请求和用于控制UART功能的命令。
\item 异步串行数据总线(UART)接口:CP2102的UART接口包括TXD(发送)和RXD(接收)数据信号以及RTS,CTS,DSR,DTR,DCD和RI控制信号。ART支持RTS/CTS,DSR/DTR和X-on/X-Off握手。且支持编程使UART支持各种数据格式和波特率。ART的数据格式和波特率的编程是在PC的COM口配置期间进行的。可以使用的数据格式和波特率见\autoref{CP2102可配置参数}。
\item 内部EEPROM:CP2102内部集成了一个EEPROM用于存储设备原始制造商定义的USB供应商ID、产品ID、产品说明、电源参数、器件版本号和器件序列号等信息\cite{CP2102}。USB配置数据的定义是可选的,如果EEPROM没有被OEM的数据所填充的话,则设备会自动的使用一组默认的数据如\autoref{CP2102DefaultConfigure}所示。
\end{enumerate}


\begin{table}[!h]
\centering
\begin{tabular}{|c|c|}
\hline
{数据位} & {5,6,7,8} \\
\hline
{停止位} & {1,1.5,2} \\
\hline
{校验位} & {无校验,偶校验,奇校验,标志校验,间隔校验} \\
\hline
{波特率} & \tabincell{c}{600,1200,2400,4800,7200,9600,14400,16000,19200,28800,\\ 38400,51200,56000,57600,64000,76800,115200,128000,158600,\\ 230400,250000,256000,4608000,576000,921600}\\
\hline
\end{tabular} 
\caption{CP2102可配置参数}\label{CP2102可配置参数}
\end{table}

\begin{table}[!h]
\centering
\begin{tabular}{|c|c|}
\hline
{\hei{Name}} & {\hei{Value}} \\
\hline
{Vendor ID} & {10C4h} \\
\hline
{Product ID} & {EA60h} \\
\hline
{Power Descriptor(attributes)} & \tabincell{c}{80h}\\
\hline 
{Power Descriptor(Max Power)} & {32h} \\
\hline
{Release Number} & {0100h} \\
\hline
{Serial Number} & {0001(63 characters maximum)} \\
\hline
{Product Description String} & \tabincell{c}{"CP2102 USB to UART Bridge Controller”(126 characters maximum)"} \\
\hline
\end{tabular}
\caption{CP2102默认配置表}\label{CP2102DefaultConfigure}
\end{table}




\subsection{CP2102应用}
	使用CP2102开发RS232转USB具有电路简单,运行可靠,成本低廉的优点,通常在使用CP2102进行串口扩展的时候所需要的外部器件是非常少的,仅仅需要2-3个去耦电容即可,在silicon给出的文档当中已经帮我们给出了一个最简单的连接电路图,如\autoref{fig:cp2102电路框图}所示,在REGIN的输入端加入一个去耦电容0.1uF与1.0uF并联。供电的电源由计算机的USB口提供,加上三只保护管即可。电路使用CP2102UART总线上的TXD/RXD两个引脚,其余的引脚都悬空。此次我们使用的是购买的已经制作好的CP2102模块如\autoref{fig:cp2102模块正反面}所示:
\begin{figure}[h]
\centering
  \begin{subfigure}[b]{0.4\textwidth}
  \includegraphics[width=\textwidth]{./graphics/cp2102Front.pdf}
  \caption{CP2102模块正面}\label{fig:cp2102Front}
  \end{subfigure}
  ~
  \begin{subfigure}[b]{0.4\textwidth}
  \includegraphics[width=\textwidth]{./graphics/cp2102Rear.pdf}
  \caption{CP2102模块反面}\label{fig:cp2102Rear}
  \end{subfigure}
\caption{CP2102模块正反面}\label{fig:cp2102模块正反面}
\end{figure}
	




