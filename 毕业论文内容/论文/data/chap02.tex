\chapter{实时操作系统VxWorks}

\section{概述}

\subsection{实时操作系统}
	实时操作系统(Real Time Operation System,简称RTOS)是整个实时系统的核心。POSIX1003.1标准为RTOS下了一个简单的定义:RTOS是能够在有限的响应时间内为应用提供所要求级别服务的操作系统\cite{Renard20081003}。当外界事件或者是数据产生时,RTOS需要快速的进行处理,并且处理的结果又能够在规定的时间之内来控制生产过程或者对处理系统做出快速的响应,调度一切可利用的资源来完成实时任务。实时系统按照实时的效果可以分为软实时和硬实时,硬实时要求在规定的时间内必须完成操作,这是通过操作的在设计的时候就得到保证的;软实时只需要按照任务的优先级,尽可能快的完成任务即可。

一个实时操作系统的特征通常包括以下几点:
\begin{itemize}
\item \textbf{高精度计时系统} 

	计时精度是影响实时性的一个重要因素,在实时系统当中,经常需要精确确定实时地操作某个设备或执行某个任务,或精确的计算一个时间函数。这些不仅仅依赖于一些硬件提供的时钟精度,也依赖于实时操作系统的高精度计时功能。
\item \textbf{多级中断机制}

	中断是实时操作系统当中的一个关键设施,是用于通知系统发生外部事件的常用机制。一个实时操作系统通常需要处理多种外部信息或事件,但处理的紧迫程度有轻重缓急之分。有的必须立即作出反应,有的则可以延后处理。因此,需要建立多级中断嵌套处理机制,以确保对紧迫程度较高的实时事件进行及时响应和处理。
\item \textbf{实时调度机制} 

	实时操作系统不仅要及时响应实时事件中断,同时也要及时调度运行实时任务。但是,处理机调度并不能随心所欲的进行,因为涉及到两个进程之间的切换,只能在确保“安全切换”的时间点上进行,实时调度机制包括两个方面,一是在调度策略和算法上保证优先调度实时任务;二是建立更多“安全切换”时间点,保证及时调度实时任务。
\end{itemize}

	内核作为操作系统的核心,负责控制这计算机上的所有硬件和软件资源,在必要的时候给应用程序分配硬件资源,并执行相应的操作命令。内核的主要功能为以下四个:
\begin{itemize}
\item 系统内存管理
\item 软件程序管理
\item 硬件设备管理
\item 文件和网络系统管理
\end{itemize}

本次所需完成的调试通道正是基于一个目前业界有名的实时操作系统VxWorks。

\subsection{VxWorks简介}
	VxWorks操作系统是美国Wind River System公司于1983年推出的一个运行在目标机上的高性能、可裁剪的实时操作系统(RTOS),该系统专门为嵌入式实时系统领域而设计开发,其具有良好的持续发展能力、高性能的内核以及友好的用户开发环境,为开发人员提供了高效的实时多任务调度、中断管理、实时的系统资源以及实时的任务间通信。并且拥有多达1800个功能强大的应用程序接口\cite{嵌入式实时操作系统VxWorks及其开发环境Tornado}。VxWorks的系统结构如\autoref{fig:VxWorks系统结构}所示
\begin{figure}[!h]
\centering
\includegraphics[width=.7\textwidth]{./graphics/VxWorks-sys-structure.pdf}
\caption{VxWorks系统结构}\label{fig:VxWorks系统结构}
\end{figure}
	
	VxWorks采用微内核设计,支持多种硬件环境,包括X86、PowerPC、ARM等众多主流的处理器,同时还支持RISC、DSP技术,且在各种CPU平台上提供了统一的编程接口和一致的运行环境。VxWorks凭借着其优异的性能在军事、航空航天、工业控制、通信等高精尖以及实时性要求极高的领域当中,有着更加广泛而深入的应用。应用实例包括火星探测器、爱国者导弹、飞机导航、F-16、FA-18战斗机等\cite{嵌入式实时操作系统VxWorks及其开发环境Tornado}。自从对我国的销售解禁之后,VxWorks也大量的应用于我国的军事、国防工业当中。
	
	VxWorks操作系统由400多个相对独立的短小精炼的目标模块组成,用户可以根据自己的实际需求选择适当的模板来裁剪和配置系统,这样可以有效的保证系统的安全性和可靠性。系统的连接器可以按照应用的需要自动链接一些目标模块。这样,通过目标模块之间的按需组合,可以得到许多满足功能需求的应用。此外,VxWorks支持广泛的工业标准,如POSIX1003.1b实时扩展、ANSIC(浮点支持)和TCP/IP网络协议等。这种广泛的协议支持在主机和目标机之间提供了无缝的工作环境,任务可以通过网络箱其他系统的主机存取文件,即远程文件存取,也支持远程的过程调用。这些标准也促进了多种不同产品之间的互通性,提升了可移植性。由于其高度的灵活性,用户能够很容易的对该操作系统进行重新定制或作适当的开发,以满足自己的实际应用。	
\subsection{VxWorks原理}
	VxWorks操作系统以实时著称,系统代码组件化比较好,系统精简,产品的IMAGE可以做的非常小。VxWorks属于共享内存的操作系统,没有进程的概念,按照任务进行调度,每个任务是分配了堆栈,可以访问系统的所有内存空间,比较高效,但是也比较危险,很容易就会发生系统崩溃的情况。但是也就是因为这个特性,因此基于VxWorks开发的产品一般是单个应用产品,这个打包完可以以一个进程体现在Linux里面。 
	
	VxWorks系统主要如下几个组件,一般的应用程序也就用到如下几个模块:任务调度负责任务优先级,任务创建,任务删除等功能。任务通讯主要涉及队列管理,管道管理等。IO管理就是通用外设的输入输出管理。文件管理包含文件或者链接创建,修改,删除,检索等等。内存管理包含内存申请及内存释放。定时器管理就是定时器创建,定时器删除。网络通信包含套接字创建,udp/tcp通信,套接字删除,及配置管理。同步就是任务同步。互斥就是任务互斥,保护关键资源。
	
	在VxWorks操作系统的代码架构里面,一般写一个应用程序只需要涉及上面几个系统组件,由于VxWorks操作系统组件化非常好,这几个组件的耦合度非常低,每个组件对外提供都是单独的头文件,比如任务调度,其头文件为taskLib.h,任务通讯如果用的队列,那其头文件就是msgQLib.h,如果是定时器管理,那其头文件就是timerLib.h,因此也让程序移植提供了很大的便利性。有很多人认为,VxWorks跟Linux操作性的系统头文件差异化太大,因此移植难度成倍速增加,其实不然,就是由于VxWorks的高度组件化,让程序移植提供了很大的便利性。
	
	

\section{微内核wind}
	现代嵌入式操作系统的一个发展趋势是尽可能的从操作系统当中去掉冗余的服务,保证内核的高效、精简,只留下一个很小的内核,由用户进程来实现大多数操作系统的功能如文件服务、进程服务、存储服务等,即所谓的微内核操作系统,从而使得系统具有模块化、结构清晰、可移植性强和可裁剪性好的特点。在微内核操作系统当中,设备驱动程序通常作为核外可裁剪的部分存在。对于操作系统内核来说,核外驱动就相当于一个普通的应用程序,因此对于系统内核的影响很小,编写和调试都很方便,同时可以避免数据流在不同级间的拷贝。微内核的设计思想首先在CMU开发的MACH操作系统当中得到了成功的应用,并使得微内核结构成为操作系统研究的热点\cite{Black1992Microkernel}
	
	VxWorks的微内核wind提供的功能包括:任务管理、事件和异步信号服务、信号量服务、消息队列服务、内存管理、中断服务程序、时钟管理和定时服务、看门狗、异常处理等。微内核的设计减少了系统的开销,从而保证了对外部事件的快速、确定的反应。
\subsection{wind的多任务机制}
	现代实时系统基于多任务和任务间通信的互补概念。多任务环境允许将实时应用程序构建为一组独立任务,每个任务都有自己的执行线程和一组系统资源。任务间通信设施允许这些任务同步并进行通信以协调其活动。在VxWorks中,任务间的通信工具包括快速信号量、消息队列、管道、套接字。
	
	wind提供的多任务环境允许实时应用程序以一套独立任务的方式构筑,每一个任务拥有独立的执行线程和一套自己的系统资源。进程间的通信机制使得这些任务的行为同步、协调。内一个开启的任务都有一个任务控制的数据结构来记录当前的任务的状态,供内核管理调度,这个控制块简记为TCB。控制块里包含了当前的状态、优先级、要等待的事件或资源、任务程序码的起始地址、初始堆栈指针等任务的上下文。
	
	wind使用中断驱动和优先级的方式。它缩短了上下文转换的时间开销和中断的时间延迟。VxWorks中的任何例程都可以被启动为一个单独的任务,拥有它自己的上下文和堆栈,还有一些其他的任务机制可以使得任务挂起、继续、删除、延时或改变优先级。	

	
	VxWorks的任务是VxWorks的最小的运行单元,每一个任务之间是共享内存的,可以互相访问,这个有点类似于Linux的线程的概念,而几个任务合起来的嵌入式产品就等同于一个进程。

\subsection{wind的任务调度}

	VxWorks中应用软件的最小单位是任务,同时也是竞争资源的最小单位。任务名是提供给用户设计软件使用的;ID是操作系统管理时使用的唯一标志;任务被分为256个优先级,0最高,255最低;VxWorks中的内存被统一编址资源共享十分方便;任务能够快速的共享系统的绝大部分资源,同时有自己独立的上下文。
	
	在VxWorks中任务调度是基于优先级的,而且是可抢占式的调度方式,这样才能够区分实际情况下的不同状态的处理级别,对高优先级的情况进行优先响应。	
	在操作系统中任务的状态通常有四种,VxWorks也不例外,这四种状态可以通过相应的函数控制其相互转化:
\begin{itemize}
\item \textbf{就绪态:}任务正在等待CPU资源。
\item \textbf{休眠态:}任务正在等待除CPU资源之外的其他资源。
\item \textbf{延迟态:}任务正在等待一定时间的延时。
\item \textbf{悬置态:}任务无法执行,主要是用于调试一种状态,这种状态仅影响任务的执行而不影响任务的转换。
\end{itemize}}
\begin{figure}[!h]
\centering
\includegraphics[width=.9\textwidth]{./graphics/vxworks-task-shift-diagram.pdf}
\caption{VxWorks状态转换图}\label{fig:VxWorks状态转换图}
\end{figure}

	下面的这张内核管理的图展示了内核中构建了四种状态队列,通过TCB中任务状态的变化,内核对其进行调度,并安保就绪的任务加入到就绪队列等候CPU资源。



\subsection{wind线程同步}
	运行环境也提供了有效的任务间通信机制,允许独立的任务在实时系统中与其行动相协调。
开发者在开发应用程序时可以使用多种方法:用于简单数据共享的共享内存、用于单 CPU
的多任务间信息交换的消息队列和管道、套接口、用于网络通信的远程过程调用、用于处理
异常事件的信号等。为了控制关键的系统资源,提供了三种信号灯:二进制、计数、有优先
级继承特性的互斥信号灯。


\section{集成开发环境Tornado}
\subsection{tadj}

\section{VxWorsks上的USB协议栈}
	USB(Universial Serial Bus,通用串行总线)是这十几年来应用在PC领域的最新型的接口技术,出现的契机是为了为了解决日益增加的PC外设与有限的主板插槽之间的矛盾,其实现的原理是由一些PC大厂商(Microsoft、Intel等)定制出来的,自从1995年在Comdex上展出以来至今已广泛地被各个PC厂家所支持。目前已经在各类外部设备中都广泛的采用USB接口。USB接口标准目前有三种:USB1.1,USB2.0和USB3.0。USB接口应用如此的广泛是由其独特和实用的特性决定的。USB的体系结构如图\autoref{fig:USB体系结构}所示
\begin{figure}[!h]
\centering
\includegraphics[width=.4\textwidth]{./graphics/USB-structure.png}
\caption{USB-structure}\label{fig:USB体系结构}
\end{figure}

\subsection{USB2.0总线结构和机制}
	USB的主要有优点有:
\begin{enumerate}
\item 使用方便,支持热插拔:连接外设不必再打开主机箱,而且方便携带,允许在任何时候USB外设的热插拔,而且不必关闭主机的电源;USB外设没有需要用户选择的设置;例如端口地址和中断请求线;自动配置外设,当用户连接USB外设到一个正在运行的系统时,系统会自动的检测外设,载入合适的驱动软件(如果有的话),并将其初始化;以上的特点使得USB外设符合便携易用的潮流。
\item 速度快:目前设备上使用大多数是USB 2.0的接口,最新生产的设备都配备了USB3.0的接口。USB2.0接口的理论速度可以达到480Mbps(即60MB/s),并且可以向下兼容USB1.1。USB3.0接口的理论速度可以达到5.0Gbps(即640MB/s),并提供USB2.0的兼容。
\item 连接灵活:一个USB主控制器理论上可以连接127个设备
\end{enumerate}

	一个主机和设备的建档连接需要一些列的层次和实体之间的交互,USB接口层在主机和设备间提供物理、信号包的关联,USB设备层表示USB系统程序实现对一个设备进行的总的USB操作,功能层适当匹配的客户服务程序层提供附加的功能给主机。设备和功能层当中各自有逻辑通信,但是实际的USB数据传输是通过USB总线接口层来实现的\cite{USB开发手册}\cite{圈圈教你玩USB}。
	
	USB2.0规范规定了USB传输的四种方式,每种方式有各自的用途\cite{USB总线接口开发指南}:
\begin{itemize}
\item \hei{控制传输:}

	控制传输是每一个设备必须要具有的传输方式,也是四种传输方式中过程最复杂的部分,他使得主机能够从设备列出的范围中读取和选择配置和其他设置,也能够发送自定义请求来为任何目的而发送和接收数据,在配置和列举的过程中起到重要的作用。
\item \hei{批量传输:}
	
	批量传输是为了处理传输速率不是很关键的情况,一般用于打印机和扫描仪。
\item \hei{中断传输:}

	中断传输是为了那些要快速实现主机和设备的交互而准备的,比如适用于鼠标和键盘。
\item \hei{等时传输:}
	
	等时传输用于必须要按照一个常数传输数据的情况,比如一个需要被实时播放的视频/音频数据流。
\end{itemize}
所有的传输都是由事物组成,事物又由包组成,而包包含一个包识别器(PID)、CRC和其他的信息,这些层次关系在USB2.0中有详细介绍。


\subsection{VxWorks上的USB驱动程序结构}


\section{CP2012模块}
在生产实践和科学研究当中经常需要在上位机和下位机中进行数据和控制的传输,这种数据、控制传输都需要通过串口来进行,而现在生产的设备中大多都不再配置串口,仅仅保留了USB接口,于是需要通过一些手段将USB转换为RS232,这时就出现了一些专用的USB转UART芯片,CP2102就是这样一种芯片。CP2102与其他同类型的芯片相比具有功耗更低、体积更小、集成度更高(仅需少量外部元件)、价格更低等优点。因此我们此次选择这个芯片作为调试通道的设计当中使用的芯片。
\subsection{CP2102简介}
	CP2102是SILICON LABORATORIES推出的USB与RS232接口转换芯片,是一种高度集成的USB-UART桥接器,提供一个使用最小化的元件和PCB空间实现RS232转USB的简便的解决方案。CP2102芯片包含有一个USB2.0全速功能控制器,EEPROM,USB收发器,振荡器和带有全部的调制解调器控制信号的异步串行数据总线(UART),CP2102将全部的部件集成在一个5mm*5mm MLP-28封装的IC当中\cite{CP2102},cp2102的电路框图如\autoref{fig:cp2102电路框图}所示。

\begin{figure}[!h]
\centering
\includegraphics[width=1.0\textwidth]{./graphics/cp2102-circuit-diagram.pdf}
\caption{cp2102电路框图}\label{fig:cp2102电路框图}
\end{figure}

	CP2102是一个USB/RS232双向转换芯片,一方面可以从主机接收USB数据并将其转换为RS232信息格式流发送给外设,另外一方面可以从RS232外设接收数据转换为USB数据格式传送回主机,这些工作都会由芯片自动完成。使用时我们只需要将数据通过USB的数据包发送给CP2102芯片即可,芯片会自动进行解析和控制。
	
	在CP2102的内部有内置的和计算机进行通信的USB协议栈,在设备驱动的支持下,PC会将该设备识别为一个虚拟的串口。此时用户可以按照通用串口的方式来使用这个虚拟的串口。	
\begin{enumerate}
\item CP2102的USB功能控制器和收发器:CP2102的USB功能控制器是一个符合USB2.0协议的全速器件,这个器件负责管理USB和UART之间的所有数据传输以及由USB主控制器发出的命令请求和用于控制UART功能的命令。
\item 异步串行数据总线(UART)接口:CP2102的UART接口包括TXD(发送)和RXD(接收)数据信号以及RTS,CTS,DSR,DTR,DCD和RI控制信号。ART支持RTS/CTS,DSR/DTR和X-on/X-Off握手。且支持编程使UART支持各种数据格式和波特率。ART的数据格式和波特率的编程是在PC的COM口配置期间进行的。可以使用的数据格式和波特率见\autoref{CP2102可配置参数}。
\item 内部EEPROM:CP2102内部集成了一个EEPROM用于存储设备原始制造商定义的USB供应商ID、产品ID、产品说明、电源参数、器件版本号和器件序列号等信息\cite{CP2102}。USB配置数据的定义是可选的,如果EEPROM没有被OEM的数据所填充的话,则设备会自动的使用一组默认的数据如\autoref{CP2102DefaultConfigure}所示。
\end{enumerate}


\begin{table}[!h]
\centering
\begin{tabular}{|c|c|}
\hline
{数据位} & {5,6,7,8} \\
\hline
{停止位} & {1,1.5,2} \\
\hline
{校验位} & {无校验,偶校验,奇校验,标志校验,间隔校验} \\
\hline
{波特率} & \tabincell{c}{600,1200,2400,4800,7200,9600,14400,16000,19200,28800,\\ 38400,51200,56000,57600,64000,76800,115200,128000,158600,\\ 230400,250000,256000,4608000,576000,921600}\\
\hline
\end{tabular} 
\caption{CP2102可配置参数}\label{CP2102可配置参数}
\end{table}

\begin{table}[!h]
\centering
\begin{tabular}{|c|c|}
\hline
{\hei{Name}} & {\hei{Value}} \\
\hline
{Vendor ID} & {10C4h} \\
\hline
{Product ID} & {EA60h} \\
\hline
{Power Descriptor(attributes)} & \tabincell{c}{80h}\\
\hline 
{Power Descriptor(Max Power)} & {32h} \\
\hline
{Release Number} & {0100h} \\
\hline
{Serial Number} & {0001(63 characters maximum)} \\
\hline
{Product Description String} & \tabincell{c}{"CP2102 USB to UART Bridge Controller”(126 characters maximum)"} \\
\hline
\end{tabular}
\caption{CP2102默认配置表}\label{CP2102DefaultConfigure}
\end{table}




\subsection{CP2102应用}
	使用CP2102开发RS232转USB具有电路简单,运行可靠,成本低廉的优点,通常在使用CP2102进行串口扩展的时候所需要的外部器件是非常少的,仅仅需要2-3个去耦电容即可,在silicon给出的文档当中已经帮我们给出了一个最简单的连接电路图,如\autoref{fig:cp2102电路框图}所示,在REGIN的输入端加入一个去耦电容0.1uF与1.0uF并联。供电的电源由计算机的USB口提供,加上三只保护管即可。电路使用CP2102UART总线上的TXD/RXD两个引脚,其余的引脚都悬空。此次我们使用的是购买的已经制作好的CP2102模块如\autoref{fig:cp2102模块正反面}所示:
\begin{figure}[h]
\centering
  \begin{subfigure}[b]{0.4\textwidth}
  \includegraphics[width=\textwidth]{./graphics/cp2102Front.pdf}
  \caption{CP2102模块正面}\label{fig:cp2102Front}
  \end{subfigure}
  ~
  \begin{subfigure}[b]{0.4\textwidth}
  \includegraphics[width=\textwidth]{./graphics/cp2102Rear.pdf}
  \caption{CP2102模块反面}\label{fig:cp2102Rear}
  \end{subfigure}
\caption{CP2102模块正反面}\label{fig:cp2102模块正反面}
\end{figure}
	




